\section{Container}

  \subsection{iterator}
    iterator和generator(带yield的函数)有些类似,区别在于内存的使用,iterator会生成所有的值并存在内存里,
    generator则按需生成值,节省内存。
    \begin{codeblock}[language=python, caption={iterator}]
      i_nums = iter([1, 2, 3, 4, 5]) #从iterable生成iterator
      ite = map(lambda x: x * x, range(10), [i for i in range(20)])
      ite = filter(lambda x: x % 2 == false, range(10))
      #这两个函数里第二个及更后面的参数可以是iterator,range,tuple,list,string等可以迭代的内容
      #map会对遍历过程中的每个元素进行function操作,filter只会留下condition为True的元素
      ite = zip('abcd', range(4), [i*i for i in range(4)])
      #zip返回的迭代器是各可循环容器的元素组成的元组,遍历直到有容器被遍历完
      print(next(result)); print(next(result)) #逐项查找
      for value in ite:
        print(value) #遍历迭代器
      #注意,迭代器在遍历后会变为空,这一点与list,range,tuple不同。
      #迭代器转为list等其他变量类型时会将其遍历一遍,因此同样会变空。
    \end{codeblock}

  \subsection{list}
    \begin{codeblock}[language=python, caption={basic operations of list}]
      #创建列表
      empty_list = []; empty_list = list()
      student1=["Hermione","Harry","Ron"]
      student2=["Draco","Padma"]
      student_score=list(range(10)) #[range(10)]会创建一个只有range(10)一个元素的列表
      student_score=[(letter, num) for letter in 'abcd' for num in range(4)]
      score = [(letter, num) for letter, num in zip('abcd', range(4))] #注意这两个是不同的
      student_score = list(map(lambda n: n*n, range(10)))
      #对列表元素进行操作
      student="Potter"
      student1.append(student) #在列表的末尾追加
      student1.insert(2, student) #在指定index处插入
      student2.extend(student1) #列表的拼接,这样可以比+更快
      del student[-1] #删除最后一个元素
      student_popped=student1.pop(1) #删除指定index处的元素,默认删除最后一个
      student_popped=student1.remove('Padma') #寻找第一个指定元素并删除,找不到会报错
      student_reversed=student1.reverse() #反转元素顺序
      seq_Ron=student.index("Ron") #查找第一个匹配的元素并输出,找不到会报错
      #列表拼接、扩展、排序、遍历
      student1=student1+student2
      student3=sorted(student1); student1.sort() #sorted是一个function,sort是一个method
      student3=sorted(student1, key=str.lower(), reverse=True) #不区分大小写倒序排序
      #sorted函数会对每个元素进行key的操作,按照返回值的顺序,对原数据进行排序
      student1.sort(reverse=True) #True stands for down
      if "Hermione" in student1:
        print("Hermione is in student1")
      for a in student1: #遍历列表
      for index, item in enumerate(student1):
      for index, student in enumerate(student1, start=1): #这里index可以从1开始编号
      students = ', '.join(student1) #将list中的元素合并成一个字符串,以逗号为分隔符
      #列表统计
      number=student1.count("Padma") #统计元素出现的数量
      ind=student2.index("Padma") #找到元素首次出现的位置
      min_one, max_one = min(student_score), max(student_score)
      total=sum(student_score) #列表求和
    \end{codeblock}

  \subsection{tuple}
    元组是不可变的序列,但可以重新赋值
    \begin{codeblock}[language=python, caption={basic operation of tuple}]
      #元组创建
      empty_tuple = (); empty_tuple = tuple() #创建空元组
      a=tuple(rage(10,20,3)); a=(); a=(1,2,3)
      a_tup = sorted(a) #tuple没有sort method
    \end{codeblock}

  \subsection{set}
    Python的set是无序集合,这与C++不同。
    \begin{codeblock}[language=python, caption={basic operation of set}]
      students=[
        {"name":"Hermione","house":"Gryffindor"},
        {"name":"Harry","house":"Gryffindor"},
        {"name":"Ron","house":"Gryffindor"},
        {"name":"Draco","house":"Slytherin"},
        {"name":"Padma","house":"Ravenclaw"},
      ]
      houses=set() #创建空集合,注意houses={}会创建一个字典
      houses.pop() #删除第一个元素,注意顺序是随机的
      houses.clear() #清空集合
      for student in students:
        houses.add(student["house"])
      for house in sorted(houses):
        print(house)
      if "Slytherin" in houses:
        print("Slytherin is in houses.")
    \end{codeblock}

  \subsection{dict}
    \begin{codeblock}[language=python, caption={basic operation of dictionary}]
      #创建字典
      students1={
        "Harry":"Gryffindor",
        "Ron":"Gryffindor",
        "Draco":"Slytherin",
        "Padma":"Ravenclaw",
      }
      name=["Harry","Ron","Draco","Padma"]
      house=["Gryffindor","Gryffindor","Slytherin","Ravenclaw"]
      student1=dict(zip(name,sign)) 
      student1=dict(n: s for n, s in zip(name,sign))
      student1=dict((("Harry","Gryffindor"),("Ron","Gryffindor"),
          ("Draco","Slytherin"),("Padma","Ravenclaw"))) #这三句的效果相同
      student1=dict(Harry="Gryffindor",Ron="Gryffindor",Draco="Slytherin",Padma="Ravenclaw")
      student_empty=dict.fromkeys(name) #创建值为空的字典
      #访问,排序,删除,加入,遍历
      ron_house=student1["Ron"] #访问字典,找不到会报错
      ron_house=student1.get("Ron") #访问字典,找不到会返回None,可以通过这种方法避免报错
      david_house=student1.get("David", "Not Found") #访问字典,找不到返回Not Found
      s_house = sorted(student1) #根据key来排序
      ron_house=student1.pop("Ron") #删除元素并返回值
      del student1["Ron"] #删除元素
      student1["Hermione"]="Gryffindor" #加入元素
      student1.update({"Hermione":"Gryffindor"}) #合并两字典
      for key,value in student1.items() #遍历字典
      for key in student1.keys(); for key in student1 #遍历键数组
      for value in student1.values() #遍历值数组
    \end{codeblock}

  \subsection{String}
    \subsubsection{编码解码}
      \begin{codeblock}[language=python, caption={create a string}]
        raw_string = r'\tTab' #这里的\textbackslash t不会被替换
        name="Harry"; text='Hello world!'
        byte=name.encode('GBK') #编码,返回编码的16进制值
        name1=byte.decode('GBK') #解码
      \end{codeblock}

    \subsubsection{常用操作}
      \begin{codeblock}[language=python, caption={basic operations of string}]
        length=len(name) #返回字符串长度
        size=len(name.encode()) #返回内存大小
        text.split(' ') #切分字符串
        strnew=string.join(house) #合并字符串
        sub='ab'
        str_replace=sub.replace('a', 'c') #字符串替换
        num=string.count(sub) #检索子字符串出现次数
        start=string.find(sub) #子字符串首次出现的索引,若没有出现,返回-1
        str_lower=str.lower() #转为小写,不会改变原字符串
        str_upper=str.upper() #转为大写
        str_strip=str.strip(['@']) #删去字符串左右两侧的' ',\textbackslash t,\textbackslash r,\textbackslash n等(默认),以及'@'(可选)
        str_lstrip=str.lstrip(['@']) #删去字符串左侧的特定字符
        str_rstrip=str.rstrip(['@']) #删去字符串右侧的特定字符
        str_filled = str.ljust(10, fillchar=' ') #字符串左对齐,右侧填充至10个字符
        str_filled = str.rjust(10, fillchar=' ') #字符串右对齐
        str_filled = str.center(10, fillchar='*') #字符串中间对齐
      \end{codeblock}

    \subsubsection{格式化字符串}
      格式化字符串也就是先制定一个模板,并在模板中以占位符为变量留下位置。一种实现方式是与C语言相似的占位符,
      更适合python的做法是使用format函数或者f字符串。
      \begin{codeblock}[language=python, caption={formatting string1}]
        '%[-][+][0][m][.n]格式字符'%exp #这是模板
        template = '编号:%09d\t 公司名称: %s\t官网: http://www.%s.com'
        print(template%(7, '百度', 'baidu'))
      \end{codeblock}

      可选项如下
      \begin{itemize}
        \item -:左对齐,正数前面无负号,负数前面有负号
        \item +:右对齐,正数前面加正号,负数前面加负号
        \item 0:右对齐,正数前面无负号,负数前面加负号,用0补足位数
        \item m:占有宽度
        \item .n:小数位数
      \end{itemize}

      \begin{table}[H]
        \centering
        \caption{placeholder}
        \label{tab:placeholder}
        \begin{tabular}{cccc}
          \toprule[1.5pt]
          格式字符 & 说明 & 格式字符 & 说明 \\
          \midrule
          \%s & 字符串 & \%c & 单个字符 \\
          \%x & 十六进制整数 & \%o & 八进制整数 \\
          \%f~~\%F & 浮点数 & \%d~~\%i & 十进制整数 \\
          \%\% & 字符\% & \%e & 指数(基底为e) \\
          \bottomrule[1.5pt]
        \end{tabular}
      \end{table}
  
      format函数使用模板
      \begin{codeblock}[language=python, caption={format string}]
        '{[index][:[fill][align][sign][#][width][.precision][type]]}'.format(item1, item2)
      \end{codeblock}

      \begin{itemize}
        \item index:可选,表示该位置填入的是format的那个参数(0-base)
        \item fill:可选,填充字符
        \item align:可选,<左对齐,>右对齐,\^{}内容居中
        \item sign:可选,+-,用法与占位符选项相同
        \item \#:可选,进制前缀显示,也可以选`,'表示添千位分隔符
        \item width:可选,字段宽度
        \item .precision:可选,小数位数
        \item type:可选,指定类型
      \end{itemize}

      \begin{table}[H]
        \centering
        \caption{type options}
        \label{tab:type options}
        \begin{tabular}{cccc}
          \toprule[1.5pt]
          格式字符 & 说明 & 格式字符 & 说明 \\
          \midrule
          S & 字符串类型(默认) & b & 十进制整数转为二进制 \\
          D & 十进制整数 & o & 十进制整数转为八进制 \\
          C & 按ASCII码转为字符 & x~~X & 十进制整数转为十六进制 \\
          e~~E & 转为科学计数法 & f~~F & 转为浮点数,默认6位小数 \\
          g~~G & 自动切换 & \% & 转为百分数 \\
          \bottomrule[1.5pt]
        \end{tabular}
      \end{table}

      下面给出一些例子
      \begin{codeblock}[language=python, caption={examples of format string}]
        number, name, url = 7, "百度", "baidu"
        template='编号:{:0>9s}\t公司名称:{:s}\t官网:http://www.{:s}.com'
        context=template.format(number, name, url)
        context_f=f'编号:{number}\t公司名称:{name}\t官网:http://www.{url}.com'

        f_template='编号:{f_number}\t公司名称:{f_name}\t官网:http://www.{f_url}.com'
        f_context = f_template.format(f_number = number, f_name = name, f_url = url)

        context='编号:{2}\t公司名称:{0}\t官网:http://www.{1}.com'.format(name, url, number)

        #下面假设有company变量,他有三个成员,分别是name、url和number
        context='编号:{0.number}\t公司名称:{0.name}\t官网:http://www.{0.url}.com'.format(company)
      
        company = {'name': 'baidu', 'number': 7, 'url': 'baidu'}
        context = '编号:{number}\t公司名称:{name}\t官网:http://www.{url}.com'.format(**company)
        context = 'This function ran with args: {}, and kwargs: {}'.format(args, kwargs)
        #这里\{\}中的内容可以是表达式或其他类型的数据(如list,dict)
      \end{codeblock}

  \subsection{collection}
    collection包含了dict,list,str等Container,同时提供了一些额外的数据结构,相当于C++标准库。
    \begin{codeblock}[language=python, caption={namedtuple}]
      import collection

      #namedtuple可以通过给tuple中每个变量加名称增加代码的可读性。
      Color = collection.namedtuple('Color', ['red', 'green', 'blue'])
      color = Color(55, 155, 255); print(color[0], color.red)

      #deque双端队列,可以从两端添加和删除元素
      collection.deque([1, 2, 3])

      #Counter计数器,计算可迭代对象中元素的频率
      collection.Counter(['a', 'b', 'c', 'a', 'b', 'b'])

      #OrderedDict有序字典,记住元素插入顺序
      collection.OrderedDict([('a', 1), ('b', 2)])
      
      #defaultdict可以为字典中没有的元素添加默认值,默认值类型为int
      dd = collection.defaultdict(int); print(dd['key']) #输出0 
      
      #heapq是优先级队列,用最小堆实现
      heap = collection.heapq.heapify([10, 17, 50, 7, 30, 24, 27, 45, 15, 5, 36, 21])
      collection.heappush(heap, 13) #插入元素,heap会以满二叉堆形式存储
      print(heapq.heappop(heap)) #弹出堆顶元素并返回
    \end{codeblock}

  \subsection{SQLite}
    sqlite3是python标准库中的数据库模块。sqlite3中数据类型有5种,分别是NULL,INTEGER,REAL(float),
    TEXT,BLOB。
    \begin{codeblock}[language=python, caption={SQLite}]
      import sqlite3

      #创建并连接数据库
      conn = sqlite3.connect(':memory:') #将数据存在RAM中,每次关闭都会清空
      conn = sqlite3.connect('my_database.db') #将数据存在文件中

      c = conn.cursor() #创建指针,数据库中的内容将由cursor完成

      c.execute("""CREATE TABLE IF NOT EXISTS students( #创建名为students的表,并创建3个column
                  first_name text, #不能创建已经存在的column否则会报错
                  last_name text,
                  id integer
                  )""") #IF NOT EXISTS是可选的

      #向表中添加数据
      c.execute("INSERT INTO students VALUES ('Amy', 'Elisabeth', 11)")
      c.execute("INSERT INTO students VALUES ('{}', '{}', {})".format( #用f-string格式化
          student1.first_name, student1.last_name, student1.id))
      c.execute("INSERT INTO students VALUES (?, ?, ?)", #可以用一个tuple简化操作
          (student1.first_name, student1.last_name, student1.id))
      c.execute("""INSERT INTO students VALUES 
              (:first_name, :last_name, :pay)""", #可以用dict使操作更清晰
          {'first_name': student1.first_name, 
          'last_name': student1.last_name, 
          'id': student1.id})
      #从表中读取数据,*表示提取全部
      c.execute("SELECT * FROM students WHERE last_name='Elisabeth'")
      c.execute("SELECT * FROM students WHERE last_name=?", ('Elisabeth',)) #1个元素也要加`,'
      c.execute("SELECT * FROM students WHERE last_name=:last", {'last': 'Elisabeth'}) 
      get_data = c.fetchone() #从select得到的内容中读取1行
      get_data = c.fetchmany(5) #从select得到的内容中读取5行,返回一个列表
      get_data = c.fetchall() #从select得到的内容中读取所有行
      
      conn.commit() #需要提交cursor的操作
      conn.close() #关闭数据库

      #with语句可以让连接,提交等操作更加简便,下面给出两个例子
      conn = sqlite3.connect('my_database.db') #将数据存在文件中
      with conn: 
          c.execute("INSERT INTO students VALUES ('Amy', 'Elisabeth', 11)")
          #离开with代码块时自动执行conn.commit()

      with sqlite3.connect('my_database.db') as conn:
          c = conn.cursor
          c.execute("INSERT INTO students VALUES ('Amy', 'Elisabeth', 11)")
          #离开with代码块时自动执行conn.commit()和conn.close()
    \end{codeblock}
