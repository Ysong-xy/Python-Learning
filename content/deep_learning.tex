\section{deep learning}

  \subsection{pytorch基本操作}
    \subsubsection{数据操作}
      \begin{codeblock}[language=python, caption={basic manipulation}]
        import torch

        x = torch.arange(12, dtype=torch.float32) #创建1至12的向量
        x = torch.zeros((3, 4)) #零张量
        x = torch.zeros_like(y) #创建同形状的零张量
        x = torch.ones((3, 4)) #全为1的张量
        x = torch.ones_like(y) #全为1的张量,形状同y
        x_repeated = torch.repeat_interleave(input, repeats, dim=None)
            #input为要进行重复的张量,repeats为重复次数(可以是整数或与input某维度长度相同的张量)
            #dim指定在哪个维度上重复元素,若未指定,input会被展平
        x_repeated = x.repeat(2, 1, 1) #在第0维重复2变,其它维不变
        x = torch.randn(3, 5) #均值为0,标准差为1的高斯分布 
        x = torch.tensor([[1, 3, 5, 6], [6, 9, 3, 5]]) #由list创建
        x = torch(2.0) #创建一个标量

        print(x.shape); print(x.numel()) #获取矩阵形状和总大小
        print(len(x)) #获取向量的维度或矩阵列向量的维度
        x = x.reshape(3, 4); x = x.reshape(3, 2, -1) #改变矩阵的形状,行优先
        x = x.unsqueeze(0) #在第0维插入一个新的维度
        x = x.permute((1, 0)) #将坐标轴顺序调换为(1, 0)
        A = torch.cat(x, y, dim=0) #将两个列数相同的张量纵向拼接
        B = torch.cat(x, y, dim=1) #将两个行数相同的张量横向拼接
        A = torch.stack(x, y, dim=0) #将两个形状相同的矩阵堆叠,产生新维度dim0

        print(x[2, 3]); x[2, 3] = 9 #通过行列索引访问和修改
        y = x[torch.tensor([1, 3, 5])] #通过一个tensor作为index取值
        y = x[[0, 1], [1, 0]] #取出(0, 1)和(1, 0)两个元素
        x[0: 2, :] = 12 #将前两行全部改成12
        x = x[None, :] #在axis=0处加入一个维度
        x.fill_(10.0) #用10.0填充整个张量,可用于修改0-dim tensor

        A = x.numpy(dtype=float); B = torch.tensor(A) #torch和numpy转换
        A = x.detach().numpy() #detach可以丢掉当前计算图,不参与梯度计算
        x.detach_() #原地操作丢掉当前计算图
        A = torch.tensor(data_frame.to_numpy()); B = pd.DataFrame(A.numpy()) #torch和DataFrame转换
        val = a.item(); val = float(a); val = int(a) #大小为1的张量转标量
        x = x.type(y.dtype) #将x改为y的类型
        x_long = x.long() #修改数据类型为long

        print(x + y, x - y, x * y, x / y, x ** y) #同形状对应元素运算,不同形状会广播
        print(x == y, x > y) #构建二元张量
        x[:] = exp(x) #求对应元素的e指数,用切片表示法可以节省内存

        x_sum = x.sum(); x_sum = x.sum(axis=[0, 1]) #求和,生成单元素张量
        x_sum_axis0 = A.sum(axis=0) #沿轴0降维,逐列求和,这里axis参数的轴会在降维后消失
        sum_x = x.sum(axis=1, keepdims=True) #沿轴1求和降维,但是轴数不变,可以广播
        x_mean_axis0 = A.mean(axis=0) #沿轴0降维,逐列求均值
        x_cumsum = torch.cumsum(x, dim=0) #计算张量在指定维度上的累积和,dim=0即沿行累加
        x_max = x.max() #返回一个tensor,最大值
        x_argmax = x.argmax() #返回一个tensor,最大值对应的index

        sorted_tensor, indices = torch.sort(x, dim=1, descending=True)
        #dim默认为-1,indices为排序后的张量中每个元素在原张量中dim维上的index
      \end{codeblock}

      注意,直接赋值不会重新分配内存
      \begin{codeblock}[language=python, caption={equal and clone}]
        a = torch.arange(10)
        b = a; print(a is b, a == b) #返回True. True
        b = a.clone(); print(a is b, a == b) #返回False, True
      \end{codeblock}

    \subsubsection{读写文件}
      \begin{codeblock}[language=python, caption={File I/O}]
        import torch
        from torch import nn
        from torch.nn import functional as F

        x = torch.arange(4); y = torch.ones(3, 2)
        torch.save([x, y], 'x-file') #写入文件
        x2, y2 = torch.load('x-file') #读取文件

        mydict = {'x': x, 'y': y} #以字典形式读写
        torch.save(mydict, 'mydict')
        mydict2 = torch.load('mydict')

        net = Net() #nn.module的子类
        torch.save(net.state_dict(), 'net.params')
        clone = Net(); clone.load_state_dict(torch.load('net.params'))
        clone.eval()
      \end{codeblock}

    \subsubsection{线性代数}
      \begin{codeblock}[language=python, caption={linear algebra}]
        A = A.T #矩阵转置
        result = torch.dot(x, y) #求向量点积
        y = torch.mv(A, x) #求矩阵-向量积
        C = torch.mm(A, B) #求矩阵-矩阵积
        C = torch.matmul(A, B) #矩阵乘法,可以处理多种输入形状,与\@ 基本相同
        C = A @ B #矩阵乘法,在处理向量时会尝试将向量作为矩阵进行矩阵乘法,而matmul会返回内积
        l2 = torch.norm(x) #求向量的$L_2$泛数
        l1 = torch.abs(x).sum() #求向量的$L_1$泛数
      \end{codeblock}

    \subsubsection{微分}
      \begin{codeblock}[language=python, caption={differentiation}]
        x = torch.arange(4.0); x.requires_grad_(True) #注意这里必须是float向量而不是int向量
        x = torch.arange(4.0, requires_grad=True) #两句等效,创建储存梯度的空间
        y = 2 * torch.dot(x, x) #标量公式,不能是向量或张量,可以用sum()
        y.backward() #调用反向传播函数
        print(x.grad) #输出梯度向量
        x.grad.zero_() #在默认情况下,PyTorch会累加梯度,我们需要清除之前的值
        with torch.no_grad(): #在这个上下文中所有的操作不会计算梯度

        y = x * x; u = y.detach() #求微分时将y是为一个常数,丢弃计算图中y的信息
        z = u * x; z.sum().backward()
        print(x.grad) #x.grad == u,全部为True
        x.grad.zero_()
      \end{codeblock}

    \subsubsection{概率}
      \begin{codeblock}[language=python, caption={Probability and Statistics}]
        import torch
        from torch.distributions import multinomial
        from d2l import torch as d2l
        
        fair_probs = torch.ones([6]) / 6 #概率向量
        sample_vec = multinomial.Multinomial(10, fair_probs).sample((500,)) 
            #进行500次测试,每次取10个样本,得到五行六列的矩阵

        print(dir(torch.distributions)) #查看所有的函数和类

        X = torch.normal(0, 1, (3, 5)) #生成正态分布数据,期望为0,标准差为1,3行5列
        X = torch.normal(0, 1, A.shape) 
      \end{codeblock}

  \subsection{线性神经网络}
    \subsubsection{线性回归}
      \begin{codeblock}[language=python, caption=]
        import torch
        from d2l import torch as d2l
        from torch import nn #neural network

        true_w = torch.tensor([2, -3.4])
        true_b = 4.2
        features, labels = d2l.synthetic_data(true_w, true_b, 1000) #初始数据集

        batch_size = 10
        data_iter = d2l.load_array((features, labels), batch_size) #将初始数据转换成迭代器

        net = nn.Sequential(nn.Linear(2, 1)) #权重矩阵的形状
        net[0].weight.data.normal_(0, 0.01) #初始化权重,更多初始化方法参考\href{https://pytorch.org/docs/stable/nn.init.html}{\underline{\texttt{torch.nn.init}}}
        nn.init.normal_(net[0].weight, mean=0, std=0.01) #这两句效果相同
        net[0].bias.data.fill_(0) #初始化偏置
        net = nn.Sequential(nn.LazyLinear(256), nn.ReLU(), nn.LazyLinear(10)) 
            #延迟初始化,只给定层的输出维数,在给定输入后根据输入的维数确定权重的shape并初始化

        loss = nn.MSELoss() #定义损失函数为$L_2$范数,更多损失函数参考\href{https://pytorch.org/docs/stable/nn.html#loss-functions}{\underline{\texttt{torch.nn loss-functions}}}
            #nn中的损失函数由reduction选项,可选'mean', 'none', 'sum'表示对损失值进行预操作,默认'mean'
            #注意MSELoss计算平方误差时不带系数1/2
        trainer = torch.optim.SGD(net.parameters(), lr=0.03) #设置优化算法为随机梯度下降

        num_epochs = 3 #迭代次数
        for epoch in range(num_epochs): #迭代训练
            for X, y in data_iter:
                l = loss(net(X), y)
                trainer.zero_grad()
                l.backward()
                trainer.step()
            l = loss(net(features), labels)

        w = net[0].weight.data #获取权重和偏置数据数据
        b = net[0].bias.data
        print(net[0].weight.grad) #获取梯度数据
        print(net[0].state_dict()) #获取一层的所有数据
        iter_para = net.parameters() #获取一个迭代器,包含所有的参数tensor
        iter_name_para = net.name_parameters() #获取迭代一,所有的(name, parameter)对
        print(net.eval()) #输出所有层的树,可视化
      \end{codeblock}

    \subsubsection{softmax回归}
      由于计算精度的影响softmax函数会出现上溢或者下溢的情况,解决这个问题的技巧是:
      \begin{itemize}
        \item 在计算softmax之前,先从所有$o_k$中减去$\max(o_k)$以解决上溢
        \item 将计算softmax和交叉熵结合在一起,保留softmax作为输出,并传递取对数的结果
              \[ \log(\hat{y}_j) = \log(\frac{\exp(o_j-\max{(o_k)})}{\sum_{k}\exp(o_k-\max{(o_k)})}) = o_j-\max(o_k)-\log(\sum_k\exp(o_k-\max(o_k))) \]
      \end{itemize}

      Pytorch的CrossEntropyLoss可以接受网络的未归一化输出$\mathbf{o}$,并在其内部进行上述操作,输出softmax的结果和交叉熵损失。

      \begin{codeblock}[language=python, caption={Softmax}]
        import torch
        from torch import nn
        from d2l import torch as d2l
        
        train_iter, test_iter = d2l.load_data_fashion_mnist(256) #加载Fashion-MNIST图片集
        
        net = nn.Sequential(nn.Flatten(), nn.Linear(784, 10)) #创建权重变量
        
        def init_weights(m):
            if type(m) == nn.Linear:
                nn.init.normal_(m.weight, std=0.01)
        
        net.apply(init_weights) #初始化权重
        
        loss = nn.CrossEntropyLoss(reduction='none')
        trainer = torch.optim.SGD(net.parameters(), lr=0.1)
        num_epochs = 10
        d2l.train_ch3(net, train_iter, test_iter, loss, num_epochs, trainer)
        d2l.plt.show()
      \end{codeblock}

    \subsubsection{多层向量机}
      \begin{codeblock}[language=python, caption={activation function}]
        import torch
        from d2l import torch as d2l

        x = torch.arange(-8.0, 8.0, 0.1, requires_grad=True)
        y = torch.relu(x) #ReLU函数,仅保留正值
        y = torch.sigmoid(x) #sigmoid压缩函数
        y = torch.tanh(x) #双曲正切函数也能将实数域压缩到(0, 1)
      \end{codeblock}

      \begin{codeblock}[language=python, caption={Multilayer}]
        import torch
        from torch import nn
        from d2l import torch as d2l

        net = nn.Sequential(nn.Flatten(), #将矩阵展平成向量
                            nn.Linear(784, 256), #可选bias=False,不带偏置
                            nn.ReLU(),
                            nn.Dropout(0.2), #标准暂退法正则化
                            nn.Linear(256, 10))

        def init_weights(m):
            if type(m) == nn.Linear:
                nn.init.normal_(m.weight, std=0.01)

        net.apply(init_weights);

        batch_size, lr, num_epochs = 256, 0.1, 10
        loss = nn.CrossEntropyLoss(reduction='none')
        trainer = torch.optim.SGD(net.parameters(), lr=lr)
        #parameters表示模型中的所有计算梯度的参数
        #这里可以设置$L_2$范数惩罚
        trainer = torch.optim.SGD(
            [{"params":net[0].weight, 'weight_decay':wd}, #给weight张量加上$L_2$范数惩罚
             {"params":net[0].bias}], #net是一个OrderedDict,每个元素代表一层
            lr = lr
        ])

        train_iter, test_iter = d2l.load_data_fashion_mnist(batch_size)
        d2l.train_ch3(net, train_iter, test_iter, loss, num_epochs, trainer)
        d2l.plt.show()
      \end{codeblock}

  \subsection{获取数据集}
    \subsubsection{线性回归数据集}
      \begin{codeblock}[language=python, caption={Get linear data}]
        def synthetic_data(w, b, num_examples):  #@save
            """生成y=Xw+b+噪声"""
            X = torch.normal(0, 1, (num_examples, len(w)))
            y = torch.matmul(X, w) + b
            y += torch.normal(0, 0.01, y.shape)
            return X, y.reshape((-1, 1))

        def data_iter(batch_size, features, labels): #加载迭代器
            num_examples = len(features)
            indices = list(range(num_examples))
            # 这些样本是随机读取的,没有特定的顺序
            random.shuffle(indices)
            for i in range(0, num_examples, batch_size):
                batch_indices = torch.tensor(
                    indices[i: min(i + batch_size, num_examples)])
                yield features[batch_indices], labels[batch_indices]
      \end{codeblock}

    \subsubsection{Fashion-MNIST图像分类数据集}
      Fashion-MNIST有10个类别,每个类别6000张图片的训练数据集和1000张图片的测试数据集。
      \begin{codeblock}[language = python, caption={Download Fashion-MNIST}]
        import torch
        import torchvision
        from torch.utils import data
        from torchvision import transforms
        from d2l import torch as d2l

        d2l.use_svg_display()

        def get_DataLoader_mnist():
            return 4

        def load_data_fashion_mnist(batch_size, resize=None):
            trans = [transforms.ToTensor()] #将图像由PIL类型转换为32位浮点数,并除以255归一
            if resize: #调整图像大小以保证大小统一
                trans.insert(0, transforms.Resize(resize))
            trans = transforms.Compose(trans)
            mnist_train = torchvision.datasets.FashionMNIST(
                root='../data', train=True, transform=trans, download=True
            )
            mnist_test = torchvision.datasets.FashionMNIST(
                root='../data', train=False, transform=trans, download=True
            )
            return (
                data.DataLoader(mnist_train, batch_size, shuffle=True, 
                                num_workers=get_dataloader_worker()),
                data.DataLoader(mnist_train, batch_size, shuffle=False, 
                                num_workers=get_dataloader_worker())
            ) #采用4个进程来读取数据
      \end{codeblock}

    \subsubsection{数据加载切分}
      \begin{itemize}
        \item torch.utils.data可以用于切分数据。
        \item torch.utils.data.DataLoader接收数据集(需要有\_\_len\_\_和\_\_getitem\_\_),并返回一个iterator
        \item torch.utils.data.Dataset是一个基类,数据集类可以是他的子类(当然也可以是list,torch.tensor等)
      \end{itemize}
      \begin{codeblock}[language=python, caption={DataLoader}]
        import torch
        import torch.utils.data

        # 自定义数据集
        class CustomDataset(torch.utils.data.Dataset):
            def __init__(self, data, labels):
                self.data = data
                self.labels = labels

            def __len__(self):
                return len(self.data)

            def __getitem__(self, index):
                # 返回指定索引的数据和标签
                return self.data[index], self.labels[index]

        # 创建数据和标签
        data = torch.randn(100, 3)  # 100 个样本,每个样本 3 个特征
        labels = torch.randint(0, 2, (100,))  # 100 个标签

        # 创建自定义数据集对象
        dataset = CustomDataset(data, labels)

        # 使用 DataLoader 加载自定义数据集
        dataloader = torch.utils.data.DataLoader(dataset, batch_size=10, shuffle=True)
      \end{codeblock}
