\section{Debug \& Tools}

  \subsection{doctest}
    doctest可以检查函数的输出,在代码注释样例中给出一组输入输出,若结果错误会报错。
    \begin{codeblock}[language=python, caption={doctest hello.py}]
      from operator import floordiv, mod

      def divide_exact(n, d):
        """Return the quotient and remainder of  dividing N by D
        >>> q, r = divide_exact(2013, 10)
        >>> q
        201
        >>> r
        2
        """
        return floordiv(n, d), mod(n, d)
    \end{codeblock}

    \begin{codeblock}[language=bash, caption={doctest bash}]
      python3 -m doctest hello.py
      **********************************************************************
      File "/mnt/d/Desktop/python/program file/test/test_basic/Pythonproject
      1/hello.py", line 8, in hello.divide_exact
      Failed example:
          r
      Expected:
          2
      Got:
          3
      **********************************************************************
      1 items had failures:
        1 of   3 in hello.divide_exact
      ***Test Failed*** 1 failures.
    \end{codeblock}

  \subsection{Try \& Assert}
    try,except用来处理可能出现报错的情况,并提供捕获错误的功能。

    常见的错误类型如下,他们都有一个共同的父类Exception
    \begin{table}[htb]
      \centering  
      \caption{Types of Error}
      \label{tab: types of error}
      \begin{tabular}{ccc}
        \toprule[1.5pt]
        错误名 & 错误类型 & 例子 \\
        \midrule
        Syntax Error & 语法错误 & 代码结构不是Python \\
        IndentationError & 缩进错误 & 缩进不一致或缺少缩进 \\
        TypeError & 对象类型错误 & 将字符串与数字相加 \\
        NameError & 命名错误 & 使用了尚未定义的变量或函数 \\
        Attribute & 属性错误 & 使用了一个class没有的属性 \\
        IndexError & 索引错误 & 访问超出list的索引位置 \\
        KeyError & 键错误 & 访问字典中不存在的键 \\
        ValueError & 值错误 & 将非数字字符串传给int() \\
        ImportError & 导入模块错误 & 模块不存在或路径错误 \\
        ArithmeticError & 数学错误 & 下面两个是他的子类 \\
        ZeroDivisionError & 除以零 & 1/0 \\
        OverflowError & 数值错误 & 算术结果超出数值类型范围 \\
        FileNotFoundError & 文件不存在 & 以只读模式打开不存在的文件 \\
        IOError~~OSError & 操作系统错误 & 文件操作出错 \\
        RuntimeError & 标准错误以外的错误 & `raise'语句触发 \\
        AssertionError & 断言错误 & assert语句触发 \\
        StopIteration & 迭代器错误 & 迭代器没有更多项目供迭代 \\
        \bottomrule[1.5pt]
      \end{tabular}
    \end{table}

    \begin{codeblock}[language=python, caption={try except}]
      try:
          f = open('testfile.txt'); val = bad_val
      except FileNotFoundError as e: #捕获文件打开错误,并执行以下部分
          print(e) #输出 No such file or directory: 'testfile.txt'
      except Exception: #捕获剩余的所有错误,exception只会执行碰到的第一个
          print('Sorry. Something went wrong.')
      else: #若try没有出现错误则执行
          print(f.read()); f.close()
      finally: #无论是否出错都会执行
          print('Executing Finally...')
    \end{codeblock}

    raise语句可以在特定情况下手动报出错误
    \begin{codeblock}[language=python, caption={Raise an error}]
      try:
          a = 2
          if a == 2:
              raise Exception
      except Exception as e:
          print(e)
    \end{codeblock}

    Python中的assert语句类似于C++assert断言,不需要导入库,可以与try,except结合使用。
    \begin{codeblock}[language=python, caption={assert in python}]
      def area_square(r):
          assert r > 0, 'A length must be positive'
          return r * r

      try:
          area_square(-1)
      except Exception as e:
          print(e)
    \end{codeblock}

  \subsection{Unit Testing}
    unittest是标准库中的一个module, 可以用于代码分块debug。 
    
    unittest 中有很多assert类型,参考\href{https://docs.python.org/3/library/unittest.html#module-unittest}{\underline{unittest文档}}
    \begin{codeblock}[language=python, caption={Unit Testing}]
      import unittest
      import app #假设app.py是待测的文件

      class TestApp(unittest.TestCase):

          @classmethod
          def setUpClass(cls): #setUpClass会在所有test之前首先运行
              print('setupClass')

          @classmethod
          def tearDownClass(cls): #tearDownClass会在所有test之后最后运行
              print('teardownClass')

          def setUp(self): #setUp会在每个test之前运行一次,注意此函数名不能自定义
              self.eg1 = 10 #可以创建实例,初始化一些变量以简化test操作

          def tearDown(self): #tearDown会在每个test之后运行一次,注意此函数名不能自定义
              pass #可以删除变量、实例以保证变量不影响下一个test

          def test_add(self): #检测add函数,这里函数名必须以test开头
              result = app.add(self.eg1, 5) #检测app.py中的add函数
              self.assertEqual(result, 15)

          def test_div(self): #检测div函数
              sefl.assertRaises(ValueError, app.div, 10, 0)
              #检测div(10, 0)运行时是否会有ValueError报错
              with self.assertRaises(ValueError):
                  app.div(10, 0) #这里利用文件管理器运行,与上一句效果相同

      if __name__ == '__main__':
          unittest.main() #运行所有的test
    \end{codeblock}

    用以下语句运行测试
    \begin{codeblock}[language=bash, caption={run unittest}]
      python -m unittest test_app.py
    \end{codeblock}

    对于网站申请类型的脚本,我们不希望test的通过与否取决于网站是否能够连接。
    为此,可以使用mock module。
    \begin{codeblock}[language=python, caption={mock module}]
      #\#\#main.py
      import unittest
      from unittest.mock import patch
      from app import get_conn #假设需要检验get\_conn函数

      class TestApp(unittest.TestCase):
        
          def test_get_conn(self):
              with patch('app.requests.get') as mocked_get: #若连接成功,会继续运行
                  mocked_get.return_value.ok = True
                  mocked_get.return_value.text = "Success!" #捏造一个成功连接,返回Success
                  result = get_conn() #试运行get\_conn
                  mocked_get.assert_called_with('http://company.com/Schafer/May') 
                      #检验url的正确性
                  self.assertEqual(result, 'Success') #检验get\_conn的返回值

      if __name__ == '__main__':
          unittest.main()

      #\#\#app.py
      import requests

      def get_conn():
          response = requests.get('http://company.com/Schafer/May')
          if response.ok:
              return response.text
          else:
              return "Fail to connect!"
    \end{codeblock}

  \subsection{jupyter notebook}
    jupyter notebook可以用于代码笔记,代码汇报等,提供代码实时运行的功能。可以在ubuntu中打开。
    \begin{codeblock}[language=bash, caption={run jupyter notebok}]
      jupyter notebook
    \end{codeblock}

  \subsection{virtualenv}
    virtualenv是指项目运行的module环境,这个环境可以不包含global环境中的所有module。
    \begin{codeblock}[language=bash, caption={setup virtualenv}]
      pip install virtualenv 
      mkdir Environments
      cd Environments
      virtualenv project1_env -p /usr/bin/python3.10 #创建virtualenv环境,python版本可选
      source project1_env/bin/activate #activate virtualenv
      pip install numpy #这里下载所需的module即可
      pip install -r requirements.txt #由文件下载所有module 
      pip freeze --local > requirements.txt #将virtualenv中的module整理成文档
      deactivate #退出virtualenv
      rm -rf project1_env #删除已创建的virtualenv环境
    \end{codeblock}

  \subsection{pipenv}
    管理环境是很有必要的,若所有项目都在base环境中运行,package的更新可能导致项目不可用。

    pipenv结合了pip和virtualenv,可以用来管理代码运行环境。
    pipenv由packages(生产环境运行必须有的包),dev-packages(开发环境所需的包)

    Pipfile列出了基本信息,Pipfile.lock是决定性的,列出了具体信息。
    \begin{codeblock}[language=python, caption={pipenv}]
      pipenv install requests #向package中添加module,环境保存在Pipfile中
      pipenv install pytest --dev #向dev中添加module
      pipenv uninstall requests #删除module

      pipenv install #安装Pipfile中的所有module
      pipenv install --ignore-pipfile #安装Pipfile.lock中的所有module
      pipenv install -r requirement.txt #利用文件加载环境
      pipenv graph #列出环境内容,包括依赖关系
      pipenv lock -r > requirements.txt #整理环境内容
      pipenv lock #更新pip.lock

      pipenv --python 3.6 #修改python版本,直接修改Pipfile不会修改环境
      pipenv --rm #删除环境,不会删除Pipfile
      pipenv install #由Pipfile重建环境
      pipenv --venv #查看虚拟环境目录
      pipenv check #检查是否能更新,Pipfile和环境是否对应

      pipenv shell #激活环境
      pipenv run python #在环境里运行指令(打开python)
      exit #退出环境 

      touch .env #这个文件可以设置独属于这个环境的环境变量
    \end{codeblock}

    \begin{codeblock}[language=python, caption={.env}]
      SECRET_KEY="MySuperSecretKey" #局部环境变量
    \end{codeblock}

  \subsection{anaconda/miniconda}
    anaconda也可以用来管理module和环境,其效果相当于pip+virtualenv。
    anaconda的优势在于可以安装不属于python的module,且可提供图形化管理功能。

    注意,anaconda和pipenv有冲突,不能一起使用。
    \begin{codeblock}[language=bash, caption={anaconda}]
      conda create --name my_app python=2.7 flask sqlalchemy
      #创建一个名为my\_app的项目,项目环境中带有flask和sqlalchemy,python版本为2.7
      conda activate my_app #激活环境,默认环境为base
      conda deactivate #退出环境
      conda env list #列出所有已创建的环境 
      conda remove --name my_app --all #删除已有的环境
      conda env export > environment.yaml #导出environment的内容
      conda env export create -f environment.yaml #通过文件创建环境
    \end{codeblock}

    有时我们需要记录环境与对应项目的目录,这是我们可以在环境目录(可以通过conda env list查看)
    etc/conda/activate.d/env\_vars.sh,etc/conda/deactivate.d/env\_vars.sh,这两个
    文件分别会在activate和deactivate时自动运行。
    \begin{codeblock}[language=bash, caption={activate.d/env\_vars.sh}]
      #!/bin/sh
      export DATABASE_URI="postgresql://user:pass@db_server:5432/test_db"
      #just add anything you need
    \end{codeblock}
    \begin{codeblock}[language=bash, caption={deactivate.d/env\_vars.sh}]
      #!/bin/sh
      unset DATABASE_URI
    \end{codeblock}

    可以通过\~{}/.bashrc中加入以下函数以自动启用文件夹中的environment.yaml。
    \begin{codeblock}[language=bash, caption={conda\_auto\_env}, mathescape=false]
      function conda_auto_env() {
        if [ -e "environment.yaml" ]; then
          ENV_NAME=$(head -n 1 environment.yaml | cut -f2 -d " ") 
          #Check if you are already in the environment
          if [[ $CONDA_PREFIX != *$ENV_NAME* ]]; then
              #Try to activate environment
              conda activate $ENV_NAME &>/dev/null
          fi
        fi
      }

      #export PROMPT\_COMMAND="conda\_auto\_env;\$PROMPT\_COMMAND"
      #\$PROMPT\_COMMAND每次按下enter时都会运行
      #若不需要,可以注释掉最后一行,并使用conda\_auto\_env手动调用环境
    \end{codeblock}

  \subsection{docker}
    docker也是一个环境管理工具,它不仅能打包python的module,还可以打包整个Linux环境或其指定部分,
    相当于一个小型的虚拟机。
    \begin{codeblock}[language=bash, caption={docker}]
      docker run -itd --name ubuntu-test ubuntu /bin/bash #以命令行模式新建容器
      docker stop ubuntu-test #停止容器
      docker start ubuntu-test #启动已经停止的容器
      docker restart ubuntu-test #重新启动停止或正在运行的容器
      docker rm ubuntu-test #删除容器,只能删除停止的容器
      docker rm -f ubuntu-test #强制删除,运行中也能删
      docker container prune #删除所有停止的容器
      docker ps #查看所有容器
      docker exec -it ubuntu-test /bin/bash #进入容器并运行其终端

      docker export ubuntu-test > ubuntu.tar #导出本地的容器,创建镜像文件
      cat ./ubuntu.tar | docker import - test/ubuntu:v1 #导入镜像
      docker images #查看所有本地镜像
      docker rmi test/ubuntu:v1 #删除本地镜像
      docker run -itd --name ubuntu-test test/ubuntu:v1 /bin/bash #由镜像创建容器
      docker commit -m="has update" -a="Author Jack" e218edb10161 test/ubuntu:v2
          #由容器创建镜像副本

      docker login #登录docker hub
      docker logout #退出docker hub
      docker search training #从docker hub上搜索相关镜像
      docker pull training/webapp #从docker hub上载入镜像
      docker run -d -P training/webapp python app.py #创建容器并运行web应用
      docker run -d -p 127.0.0.1:5001:5000 training/webapp python app.py #可以设置端口
      docker tag ubuntu:18.04 Ysong0603/ubuntu:18.04 #由一个镜像创建另一个镜像(有不同的标签)
      docker push Ysong0603/ubuntu:18.04 #将镜像提交到docker hub上
    \end{codeblock}

      使用Dockerfile可以将环境打包成镜像
      \begin{codeblock}[language=bash, caption={Dockerfile}]
        FROM python:3.12-slim
        WORKDIR /app
        COPY . /app
        RUN apt-get update && apt-get install -y python3
        CMD ["python3", "/app/main.py"]
      \end{codeblock}

  \subsection{git}
    git是一个版本控制工具,用于工作的管理。
    \begin{codeblock}[language=bash, caption={git command}]
      git init # 在本地创建一个git仓库
      git clone https://github.com/username/repo.git & cd repo # 从github上clone已有项目

      git checkout -b new-feature # 创建一个新的分支,并切换到该分支
      git checkout new-feature # 切换到目标分支
      git branch # 查看所有分支
      git branch -r # 查看remote分支
      git branch -d new-feature # 删除本地分支
      git merge main # 将main分支合并到当前分支
      git rebase main # 将当前分支变基到main分支

      git add main.py # 将修改过的文件添加到暂存区
      git add . # 添加所有修改过的文件
      git status # 查看哪些文件在暂存区中
      git diff # 查看工作区和暂存区的差异
      git rm main.py # 从暂存区和工作区中删除文件
      git rm --cached main.py #删除暂存区中的文件
      git commit -m "Create main python file." # 提交更改并添加提交信息
      git chechout . # 用暂存区的全部文件替换工作区的文件

      git pull origin main # 拉取远程仓库中的最新更新,相当于fetch+merge
      git push origin new-feature # 将本地的提交上传到仓库
      git log --graph # 查看git仓库中的提交历史记录,也可选--oneline, --reverse
      git log --author=Linus -n 5 # 查看Linus的最近5个提交
      git blame main.py # 显示文件每一行代码的修改记录
      git tag v1.0 # 为snapshot打上标签
    \end{codeblock}
