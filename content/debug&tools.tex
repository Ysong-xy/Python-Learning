\section{debug \& tools}

  \subsection{doctest}
    doctest可以检查函数的输出,在代码注释样例中给出一组输入输出,若结果错误会报错。
    \begin{codeblock}[language=python, caption={doctest hello.py}]
      from operator import floordiv, mod

      def divide_exact(n, d):
        """Return the quotient and remainder of  dividing N by D
        >>> q, r = divide_exact(2013, 10)
        >>> q
        201
        >>> r
        2
        """
        return floordiv(n, d), mod(n, d)
    \end{codeblock}

    \begin{codeblock}[language=bash, caption={doctest bash}]
      python3 -m doctest hello.py
      **********************************************************************
      File "/mnt/d/Desktop/python/program file/test/test_basic/Pythonproject
      1/hello.py", line 8, in hello.divide_exact
      Failed example:
          r
      Expected:
          2
      Got:
          3
      **********************************************************************
      1 items had failures:
        1 of   3 in hello.divide_exact
      ***Test Failed*** 1 failures.
    \end{codeblock}

  \subsection{Try \& Assert}
    try,except用来处理可能出现报错的情况,并提供捕获错误的功能。

    常见的错误类型如下,他们都有一个共同的父类Exception
    \begin{table}
      \centering  
      \caption{Types of Error}
      \label{tab: types of error}
      \begin{tabular}{ccc}
        \toprule[1.5pt]
        错误名 & 错误类型 & 例子 \\
        \midrule
        Syntax Error & 语法错误 & 代码结构不是Python \\
        IndentationError & 缩进错误 & 缩进不一致或缺少缩进 \\
        TypeError & 对象类型错误 & 将字符串与数字相加 \\
        NameError & 命名错误 & 使用了尚未定义的变量或函数 \\
        Attribute & 属性错误 & 使用了一个class没有的属性 \\
        IndexError & 索引错误 & 访问超出list的索引位置 \\
        KeyError & 键错误 & 访问字典中不存在的键 \\
        ValueError & 值错误 & 将非数字字符串传给int() \\
        ImportError & 导入模块错误 & 模块不存在或路径错误 \\
        ArithmeticError & 数学错误 & 下面两个是他的子类 \\
        ZeroDivisionError & 除以零 & 1/0 \\
        OverflowError & 数值错误 & 算术结果超出数值类型范围 \\
        FileNotFoundError & 文件不存在 & 以只读模式打开不存在的文件 \\
        IOError~~OSError & 操作系统错误 & 文件操作出错 \\
        RuntimeError & 标准错误以外的错误 & `raise'语句触发 \\
        AssertionError & 断言错误 & assert语句触发 \\
        StopIteration & 迭代器错误 & 迭代器没有更多项目供迭代 \\
        \bottomrule[1.5pt]
      \end{tabular}
    \end{table}

    \begin{codeblock}[language=python, caption={try except}]
      try:
          f = open('testfile.txt'); val = bad_val
      except FileNotFoundError as e: #捕获文件打开错误,并执行以下部分
          print(e) #输出 No such file or directory: 'testfile.txt'
      except Exception: #捕获剩余的所有错误,exception只会执行碰到的第一个
          print('Sorry. Something went wrong.')
      else: #若try没有出现错误则执行
          print(f.read()); f.close()
      finally: #无论是否出错都会执行
          print('Executing Finally...')
    \end{codeblock}

    raise语句可以在特定情况下手动报出错误
    \begin{codeblock}[language=python, caption={Raise an error}]
      try:
          a = 2
          if a == 2:
              raise Exception
      except Exception as e:
          print(e)
    \end{codeblock}

    Python中的assert语句类似于C++assert断言,不需要导入库,可以与try,except结合使用。
    \begin{codeblock}[language=python, caption={assert in python}]
      def area_square(r):
          assert r > 0, 'A length must be positive'
          return r * r

      try:
          area_square(-1)
      except Exception as e:
          print(e)
    \end{codeblock}

  \subsection{Unit Testing}
    unittest是标准库中的一个module,可以用于代码分块debug。unittest中有很多assert类型,
    参考\href{https://docs.python.org/3/library/unittest.html#module-unittest}{\underline{unittest文档}}
    \begin{codeblock}[language=python, caption={Unit Testing}]
      import unittest
      import app #假设app.py是待测的文件

      class TestApp(unittest.TestCase):

          @classmethod
          def setUpClass(cls): #setUpClass会在所有test之前首先运行
              print('setupClass')

          @classmethod
          def tearDownClass(cls): #tearDownClass会在所有test之后最后运行
              print('teardownClass')

          def setUp(self): #setUp会在每个test之前运行一次,注意此函数名不能自定义
              self.eg1 = 10 #可以创建实例,初始化一些变量以简化test操作

          def tearDown(self): #tearDown会在每个test之后运行一次,注意此函数名不能自定义
              pass #可以删除变量、实例以保证变量不影响下一个test

          def test_add(self): #检测add函数,这里函数名必须以test开头
              result = app.add(self.eg1, 5) #检测app.py中的add函数
              self.assertEqual(result, 15)

          def test_div(self): #检测div函数
              sefl.assertRaises(ValueError, app.div, 10, 0)
              #检测div(10, 0)运行时是否会有ValueError报错
              with self.assertRaises(ValueError):
                  app.div(10, 0) #这里利用文件管理器运行,与上一句效果相同

      if __name__ == '__main__':
          unittest.main() #运行所有的test
    \end{codeblock}

    用以下语句运行测试
    \begin{codeblock}[language=bash, caption={run unittest}]
      python -m unittest test_app.py
    \end{codeblock}

    对于网站申请类型的脚本,我们不希望test的通过与否取决于网站是否能够连接。
    为此,可以使用mock module。
    \begin{codeblock}[language=python, caption={mock module}]
      #\#\#main.py
      import unittest
      from unittest.mock import patch
      from app import get_conn #假设需要检验get\_conn函数

      class TestApp(unittest.TestCase):
        
          def test_get_conn(self):
              with patch('app.requests.get') as mocked_get: #若连接成功,会继续运行
                  mocked_get.return_value.ok = True
                  mocked_get.return_value.text = "Success!" #捏造一个成功连接,返回Success
                  result = get_conn() #试运行get\_conn
                  mocked_get.assert_called_with('http://company.com/Schafer/May') #检验url的正确性
                  self.assertEqual(result, 'Success') #检验get\_conn的返回值

      if __name__ == '__main__':
          unittest.main()

      #\#\#app.py
      import requests

      def get_conn():
          response = requests.get('http://company.com/Schafer/May')
          if response.ok:
              return response.text
          else:
              return "Fail to connect!"
    \end{codeblock}

  \subsection{jupyter notebook}
    jupyter notebook可以用于代码笔记,代码汇报等,提供代码实时运行的功能。可以在ubuntu中打开。
    \begin{codeblock}[language=bash, caption={run jupyter notebok}]
      jupyter notebook
    \end{codeblock}
