\section{visualization}

  \subsection{matplotlib}
    \subsubsection{画布预处理}
      \begin{codeblock}[language=python, caption={basic setup of plt}]
        from matplotlib import pyplot as plt

        plt.figure(figsize=(10,20),facecolor,edgecolor)
        plt.title("This is my title")
        plt.xlabel("x"); plt.ylabel("y")
        plt.xlim(2,22) #设置x的范围
        plt.xscale('log') #设置x为对数轴
        plt.xticks(ticks=[2*i+1 for i in range(10)], 
                   labels=[2*i+1 for i in range(10)]) #设置坐标轴位置和标签
        plt.legend(["line1", "line2"], loc="upper left") #显示图例,以加入图表的顺序为序
        plt.legend(loc=(0.03, 0.05)) #显示图例,利用图表绘制时的label,直接以横纵坐标设定位置
        plt.rcParams['font.sans-serif'] = ['SimHei'] #字体设置(默认字体不支持中文)
        plt.rcParams['axes.unicode_minus'] = False #正常显示负号

        plt.style.use("seaborn-v0_8") #利用预设样式
        plt.xkcd() #手绘样式
        plt.tight_layout() #布局更紧凑,可以避免label和ticks显示在画布范围外的问题

        plt.grid(True) #开启网格线
        plt.grid(axis=both, #axis=x or y or False
                linestyle="dashed", #or dotted or dashdot
                color="#FFFFFF"
        ) #添加网格线
        plt.axhline(5, #水平参考线
                    xmin=0.1, xmax=0.9, #\lbrack0, 1\rbrack 范围内i昂对于x轴的归一化范围
                    color="red", linestyle="--", linewidth=0.5 #参考线样式
        ) 
        plt.axvline(10) #垂直参考线
        plt.axhspan(5, 7, #水平参考区域
                    xmin=0.1, xmax=0.9, #\lbrack0, 1\rbrack 范围内i昂对于x轴的归一化范围
                    color="red", alpha=0.5, #填充样式,color会同时作用于facecolor和edgecolor
                    linestyle="--", linewidth=0.5 #区域边沿线样式
        ) 
        plt.axvspan(10, 12) #垂直参考区域

        plt.annotate(text,xy=(5,10), #待注释点坐标
                    xytext=(7,12), #注释文本位置
                    color="#FFFFFF",fontsize=16,
                    ha="center", #水平居中
                    va="bottom", #垂直对齐
                    arrowprops={"arrowstyle":"->", #or "-"
                                "color":"#FFFFFF"}
        ) #显示注释点
        plt.text(7,12,text) #显示无箭头注释

        plt.savefig('plot.png') #保存图片
        plt.show() #显示图片,注意没有这句不会显示
      \end{codeblock}

    \subsubsection{子图表}
      figure是一整张图片,plot是其中的图。一个figure可以有多个plot。
      \begin{codeblock}[language=python, caption={subplot}]
        ax = plt.gca() #返回当前坐标轴
        fig = plt.gcf() #返回当前figure

        fig, ax = plt.subplots( #创建figure和其中的subplot,ax fig是list
            nrows=2, ncols=1,
            sharex=True, #共用x轴(xticks只保留一个,xlabel不会共用)
        ) 

        ax[0].plot(x, y)
        ax[1].plot(x, y1)
        ax[0].legend()

        ax[0].set_title("Title of plot")
        ax[0].set_xlabel("x")
        ax[0].set_ylabel("y")

        plt.tight_layout()
        plt.show()
        fig.savefig('test.jpg')
      \end{codeblock}

    \subsubsection{数据获取}
      \begin{codeblock}[language=python, caption={fetch data}]
        import numpy as np
        import csv

        x = np.arange(5)
        y = np.array([3, 7, 8, 9, 13])
        y2 = np.array([2, 4, 7, 11, 16])

        with open('data.csv') as csv_file:
            csv_reader = csv.DictReader(csv_file)
            x = csv_reader.fieldnames
            y = [float(i) for i in next(csv_reader).value()]
            y1 = [float(i) for i in next(csv_reader).value()]
      \end{codeblock}

    \subsubsection{折线图}  
      \begin{codeblock}[language=python, caption={\href{https://matplotlib.org/stable/api/\_as\_gen/matplotlib.axes.Axes.plot.html\#matplotlib.axes.Axes.plot}
                                                       {\underline{line plot}}}]
        plt.plot(x, y, 'k--') #黑色虚线,第三个变量是format strings样式,见文档Notes part
        plt.plot(x, y,
                color="#5a7d9a",
                label="Line1",
                linestyle="dashed", #or dotted or dashdot
                    #也可以设置为-- -. - : None '' ' '
                linewidth=3,
                marker=".", #or "," "o" "+" "x"
                markersize=8,
                markerfacecolor="blue",
                markeredgecolor="cyan",
                markeredgewidth=2
        )
        #面积图,y轴是各组y数据的和
        plt.stackplot(x, y1, y2, y3,
                      color=["red","yellow","blue"],
                      labels=["player1", "player2", "player3"]
        ) 
        plt.fill_between(x, y, alpha=0.25) #填充x-y下方的区域
        plt.fill_between(x, y, 30, #填充x-y和y=30之间的区域
                         where=(y > 30), #仅填充y>30的部分
                         interpolate=True,
        )
        plt.fill_between(x, y, y1) #填充两条曲线之间的部分
      \end{codeblock}

    \subsubsection{柱状图}
      \begin{codeblock}[language=python, caption={bar chart}]
        plt.bar(x, y,
                width=0.8, #柱形宽度
                bottom=1, #柱形底部高度,可以是和x一样长的列表,用于bar的堆叠
                hatch="/") #or "l" "\" "\textbackslash\textbackslash" "//"
        plt.bar(['Jack', 'Ron', 'Jane'], [93, 98, 96]) #这里x轴可以不是数
        plt.barh(x,y) #横向柱状图,适用于bar较多的情况

        width = 0.25 #这是并列柱状图的画法
        plt.bar(x-width/2, y, width=width, color="#008fd5", label="y")
        plt.bar(x+width/2, y1, width=width, color="#e5ae38", label="y1")
        #直方图
        plt.hist(x, bins=5, #直方图,bins可以是整数(分组数量)或列表
                 edgecolor="black"
        ) 
        plt.hist(x, bins=[10, 20, 30, 40]) #列表设定各bins的边界
        plt.hist(x, bins=5,
                 log=True #y轴以10的幂次表示,用于数据差距比较大的情况
        )
      \end{codeblock}

    \subsubsection{饼图}
      \begin{codeblock}[language=python, caption={pie chart}]
        #普通饼图
        plt.pie(x, #不用换算成百分数,原数据即可
                labels=["一月","二月","三月"],
                colors=["red","blue","yellow"],
                wedgeprops={"edgecolor":"red"} #各区域之间的分割线
                autopct="%0.1f%%", #一位小数百分比,\%d\%\%整数百分比,\%0.1f一位小数,\%0.2f\%\%两位小数百分比
                explode=[0,0.1,0], #将第二块拉出0.5
                shadow=True, #加上阴影
                startangle=90, #开始方向,默认为右侧(0),逆时针为正方向
        ) 
        #圆环图
        plt.pie(x,colors=["red","blue","yellow"],
                shadow=True,labels=["一月","二月","三月"],
                wedgeprops={"width":0.6} #内圆半径,外圆半径为1
        ) 
      \end{codeblock}

    \subsubsection{散点图}
      \begin{codeblock}[language=python, caption={scatter plot}]
        plt.scatter(x, y, s=100, #散点图,s表示点的大小,可以是和x一样长的list
                    color="red", alpha=0.5, #填充样式
                    marker="0", #or "," "+" "x" "\^{}"
                    edgecolor="black", linewidth=2, linestyle="-.", #边缘线样式
        ) 
        plt.scatter(x, y, [i*50+100 for i in range(len(x))]) #气泡图

        plt.scatter(x, y, c=[7, 5, 9, 2, 3], cmap='Greens') #通过cmap设置颜色
        cbar = plt.colorbar(label='heat') #显示colorbar及其注释
        #可以通过print(plt.colormaps)查看所有的可选cmap
        #误差棒图
        plt.errorbar(x, y, #散点数组
                    yerr=[[0.1, 0.3, 0.2], [0.4, 0.1, 0.1]], #上置信度和置信度
                    xerr=[0.05, 0.04, 0.07], #上下置信度
                    fmt="o--", #散点样式和连线样式
                    color="red", #color of scatters
                    ecolor="blue", #color of the errorbars
                    elinewidth=3, #误差棒粗细
                    capsize=2, #误差棒横杠大小
                    capthick=2, #误差棒边界横杠厚度
                    uplims=False, #显示误差上界,False显示,True隐藏
                    lolims=[True, False, False], #只显示指定的下界
        ) #其余选项和plot,scatter类似
      \end{codeblock}

    \subsubsection{箱型图}
      \begin{codeblock}[language=python, caption={box plot}]
        #箱型图
        plt.boxplot(x, #x是一个数组
                    showmeans=True, #显示均值
                    patch=True, #显示缺口箱图,默认矩形箱图
                    vert=False, #False为横向,True为纵向
                    showfliers=True, #显示离群点,默认显示
                    whis=1.5, #箱须延长的百分比范围
                    width=1, #箱体总宽度
                    flierprops={ #离群点样式
                        "marker":"o", #or "," "+" "x"
                        "markerfacecolor":"red",
                        "markeredgecolor":"black",
                        "markersize":8
                    }, 
                    patch_artist=True, #自定义箱型
                    boxprops={ #箱型样式
                        "facecolor":"red",
                        "edgecolor":"yellow",
                        "linewidth":4
                    },
        ) 
        #绘制多个箱型
        plt.boxplot((x, y), #适用于x轴没有数值含义的时候
                    tick_labels=['x', 'y'], #均匀分布
        )
        plt.boxplot((x, y), #适用于x轴有数值
                    positions=(1, 2), 
                    manage_ticks-True #刻度会根据positions自动调整
        )
      \end{codeblock}

    \subsubsection{热力图}
      \begin{codeblock}[language=python, caption={heat map}]
        #绘制热力图
        plt.imshow(x, #x是个二维列表
                  cmap=plt.cm.cool #设置颜色
        ) 
        plt.colorbar(label="label") #显示图例
      \end{codeblock}
    
    \subsubsection{时间图像}
      \begin{codeblock}[language=python, caption={Plotting Time Series Data}]
        import pandas as pd
        from datetime import datetime, timedelta
        from matplotlib import pyplot as plt
        from matplotlib import dates as mpl_dates

        #获取时间
        dates = pd.Series([datetime(2024, 6, 30), datetime(2024, 7, 5), datetime(2024, 7, 11)])
        dates = pd.Series(["2024-6-30", "2024-7-5", "2024-7-11"])
        dates = pd.to_datetime(dates).sort_values() #注意需要排序
        y = [4, 6, 7]

        plt.plot(dates, y)
        plt.gcf().autofmt_xdate() #将日期斜方向写,gcf表示当前图像

        date_format = mpl_dates.DateFormatter('%b, %d %Y')
        plt.gca().xaxis.set_major_formatter(date_format) #设置日期格式
      \end{codeblock}

    \subsubsection{极坐标图表}
      \begin{codeblock}[language=python, caption={polar plots of plt}]
        plt.polar(theta,r) #雷达图
        plt.thetagrid(angles,labels) #角刻度标签
        plt.rgids(radii,rotation,labels) #r方向刻度标签

        ax=plt.axes(polar=True) #建立极坐标画布
        ax.bar(x=theta,height=data,width=0.4,color="rainbow") #绘制南丁格玫瑰图
        ax.bar(x=theta,height=100,width=0.4,color="white") #绘制中心空白
        ax.text(angle,height,text) #添加注释
        ax,grid(False)
        plt.thetagrids(angles=[],labels=[]) #刻度标签
        plt.rgrids(radii=[20],rotation,labels=['20'])
      \end{codeblock}

    \subsubsection{三维图表}
      \begin{codeblock}[language=python, caption={3D plots of plt}]
        from mpl_tookits.mplot3d import Axes3D
        fig=plt.figure()
        ax1=plt.axes(projection="3d")
        ax1.scatter3D(x,y,z,cmap="blue")
        ax1.plot3D(x,y,z,"gray")
        ax1.plot_surface(X,Y,Z,rstride=0.1,cstride=0.1) #步长越短越清晰
        ax1.contour(X,Y,Z,zdir='x',offset=-3,cmap="cold") #绘制等高线,投影在x=3平面上
        ax1.bar3d(X,Y,height,width,depth,Z,color="red",shade=True) #绘制柱状图,height为柱底高度
      \end{codeblock}

    \subsubsection{动态实时图像}
      \begin{codeblock}[language=python, caption={plotting live date}]
        from matplotlib.animation import FuncAnimation
        import itertools, random

        x_vals, y_vals, index = [], [], itertools.count()
        def animate(i):
            x_vals.append(next(index))
            y_vals.append(random.randint(0, 5))
            plt.cla() #先清除上一帧图像
            plt.plot(x_vals, y_vals)
            plt.tight_layout()

        ani = FuncAnimation(plt.gcf(), animate, interval=1000)

        plt.tight_layout()
        plt.show()
      \end{codeblock}

  \subsection{wordcloud}
    \begin{codeblock}[language=python, caption={wordcloud}]
      import matplotlib.pyplot as plt
      import wordcloud as wc

      text_data = """
      Python is a popular programming language.
      It is widely used for web development, data analysis, and artificial intelligence.
      Word clouds are fun visualizations of text data.
      Generate a word cloud using the wordcloud module.
      """

      # 生成词云对象
      wordcloud = wc.WordCloud(width=800, height=400, background_color='white').generate(text_data)

      # 显示词云图
      plt.figure(figsize=(10, 5))
      plt.imshow(wordcloud, interpolation='bilinear')
      plt.axis('off')
      plt.show()
    \end{codeblock}
