\section{Basic Module}

  \subsection{install and import module}
    \subsubsection{install and manage modules}
      可以通过pip安装和管理所需的module。
      \begin{codeblock}[language=bash, caption={download modules}]
        pip install pandas  #安装module
        pip uninstall pandas  #卸载安装的module
        pip list #列出所有已安装的module,可选-o查看是否是最新版本
        pip install -U pandas #更新module
        pip freeze --local | grep -v '^\-e' | cut -d = -f 1 | xargs -n1 pip install -U
        pip search pandas #搜索指定的module
        pip freeze > requirements.txt #将项目所需的module整理成文档
        pip install -r requirements.txt #将文件中的module全部安装下来
      \end{codeblock}

    \subsubsection{import modules}
      \begin{codeblock}[language=python, caption={import module}]
        import math
        import math as ma
        from math import sqrt
        from math import *
      \end{codeblock}

    \subsubsection{view the functions in modules}
      \begin{codeblock}[language=python, caption={view the functions in builtin module}]
        import builtins
        print(dir(builtins)) #这个查看方法对所有module都有效
      \end{codeblock}

    \subsubsection{circular import}
      circular import是一个常见的错误。当main.py import B.py的其中部分时,B.py整个文件都会被运行,
      若B.py同样import main,那么A.py也会从头开始运行整个文件,而这时main.py会找不到其import B.py中的部分,
      这时就触发了circular import。

  \subsection{builtins module}
    最基础的module,包含了print len range abs等常用函数,TypeError ValueError KeyError IndexError
    等常见异常,True False None等内置常量,int str list dict set等内置类型,不需要导入即可使用。

  \subsection{sys module}
    有关系统操作的module
    \begin{codeblock}[language=python, caption={sys module}]
      print(sys.path)
      #系统路径列表,可以通过append追加module所在地址
      #也可以追加在~/.bashrc的PYTHONPATH里面(作为环境变量)
      #如果需要import的module是一个文件夹,需要保证文件夹中有一个'\_\_init\_\_.py'文件
      #'\_\_init\_\_.py'文件会在import时运行,该文件可以为空,也可以导入文件夹中的子模块
      print(sys.executable) #输出python的文件位置
      print(sys.version) #输出python版本,用来检验编译器
    \end{codeblock}

  \subsection{os module}
    os module可以实现操作系统相关的功能。
    \begin{codeblock}[language=python, caption={os module}]
      #文件管理
      os.getcwd() #输出工作区目录
      os.chdir('/mnt/d/Desktop') #移动到指定目录
      os.listdir() #输出当前目录下的所有文件及文件夹
      os.mkdir('os_test') #在当前目录下新建,注意只能新建一层
      os.makedirs('os_test/my_profile') #新建目录,允许多层
      os.rmdir('os_test'); os.removedirs('os_test/myprofile') #删除目录
      os.rename('test.txt', 'demo.txt') #重命名文件
      print(os.stat('demo.txt')) #查看文件属性,如文件大小(st\_size),最后一次修改时间(st\_mtime)
      for dirpath, dirnames, filenames in os.walk(/mnt/d/Desktop/python) #递归的遍历目录下所有文件
          if f.endswith('.py'): #筛出所有python脚本文件
      #dirpath表示当前所在的目录,dirname表示当前目录下的文件夹,filenames表示当前目录下的文件名

      os.environ.get('HOME') #返回home的地址,这里还可以查询其他环境变量,如PATH
      os.path.join(os.environ.get('HOME"), 'test.txt') #拼接目录,直接字符串相加容易遗漏或多加slash
      os.path.exists('/mnt/d/test.txt') #返回一个bool值,即目录是否存在
      os.path.basename('/mnt/d/test.txt') #返回`test.txt'这里无论路径是否存在都不会报错
      os.path.dirname('/mnt/d/test.txt') #返回`/mnt/d'
      os.path.split('/mnt/d/test.txt') #返回(`/mnt/d', 'test.txt')
      os.path.splitext('/mnt/d/test.txt') #返回(`/mnt/d/test',`.txt')
    \end{codeblock}

  \subsection{subprocess module}
    subprocess可以用于在终端中执行任务。
    \begin{codeblock}[language=python, caption={subprocess module}]
      subprocess.run('ls') #在terminal中运行ls,若运行失败不会abort
      subprocess.run('ls', check=True) #在terminal中运行ls,若运行失败会abort
      subprocess.run('ls', stderr=subprocess.DEVNULL) #在terminal中运行ls,忽略所有error

      p1 = subprocess.run(['ls', '-la']) #捕获运行的结果
      p1.args; p1.returncode; p1.stderr #指令,返回值(0表示没有出错,1表示出错),标准错误流

      p1 = subprocess.run(['ls', '-la'], capture_output=True) #捕获运行的输出(字节流)
      print(p1.stdout.decode()) #返回标准输出
      p1 = subprocess.run(['ls', '-la'], capture_output=True, text=True) 
          #捕获运行的输出(字符串)
      print(p1.stdout); #返回标准输出

      with open('output.txt', 'w') as f:
          p1 = subprocess.run(['ls', '-la'], stdout=f, text=True) #输出到文件中

      p1 = subprocess.run(['cat', 'test.txt'], capture_output=True, text=True)
      p2 = subprocess.run(['grep', '-n', 'test'], capture_output-True, 
                          text-True, input=[p1.stdout])
      #传递输出和输入值,逐步完成process
    \end{codeblock}

  \subsection{time module}
    time module可以用于计时
    \begin{codeblock}[language=python, caption={time module}]
      time.sleep(1) #将进程挂起1s
      t = time.time() #返回linux时间戳,单位都是秒
      t1 = time.perf_counter(); t2 = time.perf_counter() #高精度时间计数器
      t3 = time.process_time(); t4 = time.process_time() #高精度的CPU时间
      print(t2 - t1) #计算时间差,会包含sleep()的时间
      print(t4 - t3) #计算时间差,不会包含sleep()的时间
    \end{codeblock}

    \begin{codeblock}[language=python, caption={Timer Class}]
      class Timer:  #@save
          """记录多次运行时间"""
          def __init__(self):
              self.times = []
              self.start()

          def start(self):
              """启动计时器"""
              self.tik = time.time()

          def stop(self):
              """停止计时器并将时间记录在列表中"""
              self.times.append(time.time() - self.tik)
              return self.times[-1]

          def avg(self):
              """返回平均时间"""
              return sum(self.times) / len(self.times)

          def sum(self):
              """返回时间总和"""
              return sum(self.times)

          def cumsum(self):
              """返回累计时间"""
              return np.array(self.times).cumsum().tolist()
    \end{codeblock}

  \subsection{threading module}
    \subsubsection{threading}
      threading module用于处理多线程。
      \begin{codeblock}[language=python, caption={threading module}]
        my_function(sec):
            time.sleep(sec)
            return "accomplished"
        t1 = threading.Thread(target=my_function, args=(1,)) #创建一个线程
        t2 = threading.Thread(target=my_function, args=(2,)) #创建一个线程
        t1.start(); t2.start(); #启动线程,同时运行余下部分(3个线程)
        t1.join(); t2.join(); #待t1,t2线程结束后再继续运行余下部分(2个线程)
      \end{codeblock}

    \subsubsection{concurrent.futures}
      \begin{codeblock}[language=python, caption={concurrent.futures module}]
        import concurrent.futures
        
        with concurrent.futures.ThreadPoolExecutor() as executor:
            f1 = executor.submit(my_function, 1) 
            f2 = executor.submit(my_function, 2) #这两个线程是并行的
            print(f1.result()); print(f2.result())

            results = [executor.submit(my_function, i) for i in range(10)]
            for f in concurrent.futures.as_completed(results):
                print(f.result()) #根据完成先后打印

            results = executor.map(my_function, [i for i in range(5)])
            for result in results:
                print(result) #根据提交先后打印
      \end{codeblock}

    \subsubsection{multiprocessing}
      \begin{codeblock}[language=python, caption={multiprocessing module}]
        processes = []
        for _ in range(i):
            p = multiprocessing.Process(target=my_function, args=(i,))
            p.start()
            processes.append(p)
        for process in processes:
            process.join()
      \end{codeblock}

  \subsection{logging module}
    logging module是python标准库中用于记录日志的模块,它可以在程序运行时输出各种级别的日志。
    \begin{codeblock}[language=python, caption={logging module}]
      #basicConfig用于设置日志的基本信息,这会设置root logger,影响所有的logger
      logging.basicConfig(filename='app.log', filemode='w', 
                  level=logging.DEBUG, #默认是WARNING
                  format='%(asctime)s - %(name)s - %(levelname)s - %(message)s', 
                  datefmt='%Y-%m-%d %H:%M:%S'
                      #format的更多选项见\href{https://docs.python.org/3/library/logging.html#logrecord-attributes}{\underline{LogRecord attributes}}
                  )
      #注意,多次设置root logger只有第一个生效,这在导入文件时要特别注意

      #可以使用handlers将日志同时输出到多个文件或输出到控制台
      logging.basicConfig(
          level=logging.INFO,
          format='%(asctime)s - %(levelname)s - %(message)s',
          handlers=[
              logging.FileHandler("logfile.log"),   # 文件处理器
              logging.StreamHandler()               # 控制台处理器
          ]
      )

      #为了让每个文件有单独的logging文件和等级,可以在每个文件里单独设置logger
      logger = logging.getLogger(__name__) #一般以\_\_name\_\_作为logger的命名
      logger.setLevel(logging.INFO) #设置logger的level
      file_handler = logging.FileHandler('test.log') #设置日志文件和格式
      file_handler.setFormatter(logging.Formatter('%(levelname)s:%(name)s:%(message)s'))
      logger.addHandler(file_handler) #将logger的内容输出到文件中
      string_handler = logging.StreamHandler()
      string_handler.setFormatter(logging.Formatter('%(levelname)s:%(name)s:%(message)s'))
      logger.addHandler(string_handler) #将logger的内容输出到终端

      #以下信息将会以上面设定的形式存到app.log文件中,只有level以上级别会显示出来
      logging.debug("这是调试信息")
      logging.info("这是一般信息")
      logging.warning("这是警告信息")
      logging.error("这是错误信息")
      logging.critical("这是严重错误信息")

      #若为每个文件单独设置了logger,则可以这样调用
      logger.info("这是一般信息")

      #error等级的错误可以添加报错信息,即traceback
      logging.exception("这是一条带有Traceback的错误信息")
    \end{codeblock}

  \subsection{argparse}
    argparse是Python标准库中的一个模块,用于解析命令行参数和选项,适合编写需要接受命令行输入的脚本。
    \begin{codeblock}[language=python, caption={argparse}]
      import argparse

      parser = argparse.ArgumentParser(description="一个简单的计算程序“)
      
      parser.add_argument("num1", type=float, help="第一个数字”) #添加参数(必选)
      parser.add_argument("num2", type=float, help="第二个数字”) #添加参数(有默认值)
      parser.add_argument("-o", "--operation", choices=["add", "sub", "nul", "div"],
          default="add", help="选择晕眩类型") #添加可选参数,需要有默认值

      args = parser.parse_args() #解析命令行参数

      if args.operation == "add":
         result = args.num1 + args.num2
    \end{codeblock}

  \subsection{re module}
  
    \subsubsection{regular expression}
      正则表达式由元字符,限定符,选择字符,排除字符等组成。

      元字符中的`.'`\$'`\^{}',限定符中的`?'`+'`*'以及`\textbackslash'若需要匹配,则应该使用转义字符。
      \begin{table}[htb]
        \centering
        \caption{meta-character}
        \label{tab:meta-character}
        \begin{tabular}{cc}
          \toprule[1.5pt]
          元字符 & 匹配对象 \\
          \midrule
          . & 换行符以外的任意字符\lbrack\^{}\textbackslash n\textbackslash r\rbrack \\
          \textbackslash w & 字母、数字、下划线或汉字 \\
          \textbackslash W & 非字母、数字、下划线、汉字 \\
          \textbackslash s & 任意空白字符(\lbrack\textbackslash f\textbackslash n\textbackslash r\textbackslash t\textbackslash v\rbrack) \\
          \textbackslash S & 任意非空白字符 \\
          \textbackslash d & 数字字符 \\
          \textbackslash D & 非数字字符 \\
          \textbackslash b & 单词的边界(开始或结束) \\
          \textbackslash B & 非单词的边界 \\
          \^{} & 字符串的开始 \\
          \$ & 匹配字符串的结束 \\
          \bottomrule[1.5pt]
        \end{tabular}
      \end{table}

      \begin{table}[htb]
        \centering
        \caption{qualifier}
        \label{tab:qualifier}
        \begin{tabular}{cc}
          \toprule[1.5pt]
          限定符 & 含义 \\
          \midrule
          ? & 匹配0-1次 \\
          + & 匹配至少1次,会贪婪匹配 \\
          * & 匹配0次或多次,会贪婪匹配 \\
          +?~~*? & 匹配规则同上,但是非贪婪 \\
          \{n\} & 匹配n次 \\
          \{n,\} & 匹配至少n次 \\
          \{n,m\} & 匹配至少n次,至多m次 \\
          | & 匹配左右其一 \\
          \bottomrule[1.5pt]
        \end{tabular}
      \end{table}

      \begin{itemize}
        \item 方括号表示字符集合,如\lbrack aeiou\rbrack 匹配元音字符,\lbrack a-z\rbrack 匹配小写字符。
          \lbrack\textbackslash u4e00-\textbackslash u9fa5\rbrack 匹配汉字。
        \item 方括号内的\^{}表示排除字符,如\lbrack\^{} aeiou\rbrack 匹配非元音字母的所有字符。
        \item 方括号中的无法使用限定符,限定符都判定为原来的字符。
        \item 小括号可以改变限定符的作用范围,如(thir|four)th相当于thirth|fourth,(\textbackslash .\lbrack 0-9\rbrack\{1,3\})\{3\}
          会将括号中的内容重复三次。
        \item 小括号还可以基于匹配模式从字符串中提取子字符串,组成一个元组。
      \end{itemize}

      Python中使用正则表达式需要将部分\textbackslash 转义,由于需要转义的\textbackslash 可能很多,可以使用模式字符串,即在字符串前加上r或R,
      例如:r`\textbackslash bm\textbackslash w*\textbackslash b'。

  \subsubsection{match string}
    Python中的re模块可以通过正则表达式处理字符串。

    下面是常用的匹配函数,这里匹配的字符串区间没有重叠的部分。
    \begin{codeblock}[language=python, caption={string matching}]
      import re #导入模块

      pattern=r'mr_\w+' #模式字符串
      string='MR_SHOP mr_shop' #待匹配字符串

      #match方法可以从字符串开始处开始匹配,若匹配失败,返回None
      match=re.match(pattern,string,re.I) #不区分大小写匹配字符串,返回一个match对象
      print(match.start());print(match.end()) #输出匹配字符串起始位置
      print(match.span()) #输出匹配字符串起止位置元组
      print(match.string) #输出匹配前的字符串,即string
      print(match.group()) #输出匹配的数据

      #search方法可以搜索字符串中第一个可以匹配的子串,返回一个match对象
      match=re.search(pattern,string,re.I) #不区分大小写匹配字符串,返回一个match对象

      #findall和finditer方法可以搜索字符串中所有可以匹配的子串
      match = re.findall(pattern,string,re.I) #不区分大小写匹配字符串,返回一个match对象
      print(match) #输出匹配子串的列表
      match = re.finditer(pattern, string, re.I) #返回一个由match对象组成的迭代器

      #上面的代码可以通过编译模式字符串进行简化
      pattern = re.compile(r'mr_\w+', re.I)
      match = pattern.match(string)
      match = pattern.match(string)
      match = pattern.findall(string)
      match = pattern.finditer(string)

      #捕获组的使用
      pattern = re.compile(r'(\w+)-(\d+)-(\w+)')
      text = "abc-123-def"
      match = pattern.match(text) #这里使用方法与上面相同
      #match.group(0)为'abc-123-def', match.group(i)为第i个捕获组捕获的内容
      match = pattern.findall(text) #match = \lbrack(`abc', `123', `def')\rbrack
    \end{codeblock}

    \begin{table}[htb]
      \centering
      \caption{matching pattern}
      \label{tab:matching pattern}
      \begin{tabular}{cc}
        \toprule[1.5pt]
        标志 & 含义 \\
        \midrule
        A~~ASCII & 只匹配ASCII范围内的字符 \\
        I~~IGNORECASE & 不区分大小写 \\
        M~~MULTILINE & 将\^{}和\$ 用于每一行的开头和结尾 \\
        S~~DOTALL & .匹配所有字符,包括换行符 \\
        X~~VERBOSE & 忽略未转义的空格和注释 \\
        \bottomrule[1.5pt]
      \end{tabular}
    \end{table}

  \subsubsection{substitute string}
    sub方法可以实现vim中批量搜索并替换字符串的操作。

    模板为:re.sub(pattern,repl,string,count,flags)
    \begin{itemize}
      \item pattern:模式字符串
      \item repl:替换后的子字符串
      \item string:原始字符串
      \item count:可选,最大替换次数,默认为0,表示替换所有的匹配
      \item flags:可选,见\autoref{tab:matching pattern}
    \end{itemize}

    \begin{codeblock}[language=python, caption={string substitution}]
      import re

      pattern = r'1[34578]\d{9}' #模式字符串
      string='电话号码是:13611111111'
      result=re.sub(pattern,'1xxxxxxxxxx',string)

      pattern = re.compile(r'(1[34578]\d)\d{8}')
      string = 'The phone numbers are:13611111111,15888888888,18333333333'
      result = pattern.sub(r'\1********', string) #这里\textbackslash 1表示第一个捕获组
      result = 'The phone numbers are:136********,158********,183********'
    \end{codeblock}

  \subsubsection{split string}
    re.split方法可以根据正则表达式切分字符串,匹配正则表达式的子串将被当作分隔符,并将分割结果以列表的形式返回。

    模板为:re.split(pattern,string,\lbrack maxsplit\rbrack,\lbrack flags\rbrack)
    \begin{itemize}
      \item pattern:模式字符串
      \item string:待切分的字符串
      \item maxsplit:可选,最大拆分次数
      \item flags:可选,见\autoref{tab:matching pattern}
    \end{itemize}
    \begin{codeblock}[language=python, caption={string segmentation}]
      import re

      pattern=r'[?&]'
      url='http://www.mingrisoft.com/login.jsp?username="mr"&pwd="mrsoft"'
      result=re.split(pattern,url)
      print(result)
    \end{codeblock}

  \subsection{date module}
    \subsubsection{datetime module}
      datetime用于时间和日期戳记录和转换
      \begin{codeblock}[language=python, caption={datetime \& pytz}]
        d = datetime.date(2020, 7, 24); print(d) #这里的day是一个date类型的变量
        td = datetime.date.today() #输出今天的日期
        print(d.year); print(d.month); print(d.day)
        print(d.weekday()); print(tday.isoweekday()) 
        #in weekday: Monday 0 Sunday 6, in isoweekday: Monday 1 Sunday 7
        tdelta = datetime.timedelta(days = 7) #这里tdelta是timedelta类型的,可以参与加减法
        date2 = date1 + timedelta; timedelta = date1 - date2
        tdelta.total_seconds() #时间差转秒

        t = datetime.time(9, 30, 45, 100000) #四个参数分别是h m s $\mu$s
        print(t.hour); print(t.minute); print(t.second); print(t.microsecond)
        print(t.iosformat()) #以国际标准形式输出时间
        print(t.strftime('%B %d, %Y')) #以自定义形式输出时间,这里可以参考\href{https://docs.python.org/3/library/datetime.html\#strftime-strptime-behavior}{\underline{strftime官方文档}}

        dt = datetime.datetime.strptime('July 26, 2016', '%B %d, %Y') 
            #根据模板由string转datetime
        dt = datetime.datetime(2020, 7, 8, 12, 10, 32, 100000, tz = pytz.UTC) #时区可选
        t = dt.time(); d = dt.date(); print(dt.year)
        sentence = f'Jack has a birthday on {birthday:%B %d, %Y}' #结合fstring使用
      \end{codeblock}

    \subsubsection{pytz module}
      pytz用于管理时区(time zone)相关的信息。
      \begin{codeblock}[language=python, caption={pytz module}]
        dt_today = datetime.datetime.today() #返回当前时区的时间
        dt_now = datetime.datetime.now(tz = pytz.UTC) #返回指定时区的时间,默认没有时区信息
        dt_utcnow = datetime.datetime.utcnow() #返回utc时间,不建议使用
        dt_mtn = dt_now.astimezone(pytz.timezone('Asia/Shanghai')) 
        #需要带有时区参数的datetime变量才可以进行时区转换
        for tz in pytz.all_timezones:
          print(tz)  #输出所有的时区名称
        dt_mtn = pytz.timezone('America/New_York').localize(dt_today) #为时间添加时区
        dt_time = datetime.datetime.fromtimestamp(mod_time) #将时间戳转换成datetime类型
      \end{codeblock}

    \subsubsection{calendar module}
      calendar用于计算天数,星期数,星期几等数据。
      \begin{codeblock}[language=python, caption={calendar module}]
        today = datetime.date.today()
        days_in_current_month = calendar.monthrange(today.year, today.month)
        #输出一个tuple,第一个元素表示今天星期几Monday0 Sunday6,第二个表示本月有几天
      \end{codeblock}

  \subsection{PIL module}
    PIL module可以处理图片,module中的具体内容见\href{https://pillow-cn.readthedocs.io/zh-cn/latest/}{\underline{PIL官方文档}}。
    \begin{codeblock}[language=python, caption={PIL}]
      from PIL import Image, ImageFilter

      image1 = Image.open('pup1.jpg')
      image1.save('pup1.png')
      image1.show() #显示图片

      image1.thumbnail((300, 300)) #压缩至300*300像素
      image1.rotate(90) #逆时针旋转$90^{\circ}$
      image1.convert(mode='L') #图片变黑白

      image1.filter(ImageFilter.GaussianBlur(15)) #模糊图片,模糊程度可选
    \end{codeblock}

  \subsection{random module}
    \begin{codeblock}[language=python, caption={random module}]
      value = random.random() #生成[0.0, 1.0)的随机浮点数
      value = random.uniform(1, 10) #生成[1.0, 10.0]的随机浮点数
      value = random.randint(1, 10) #生成[1, 10]的随机整数
      value = random.choice(['Red', 'Greed', 'Blue']) #从list中随机取出一个
      results = random.choices(['Red', 'Greed', 'Blue'], k=10) #随机取k遍,输出一个list
      results = random.choices(['Red', 'Greed', 'Blue'], weight=[18, 18, 2], k=10) 
      shuffled_list = list(range(2, 30)); random.shuffle(shuffled_list) #随机打乱list
      hand = random.sample(list(range(50)), k=5) #取5个样本,5个样本不会重复
      random.shuffle(my_list) #直接打乱列表,没有返回值
    \end{codeblock}

  \subsection{secrets module}
    secrets可以生成随机长密码。
    \begin{codeblock}[language=python, caption={secrets module}]
      import secrets
      print(secrets.token_hex(16)) #16Bytes
    \end{codeblock}

  \subsection{cn2an module}
    cn2an module可以实现中文数字到阿拉伯数字的转换。
    \begin{codeblock}[language=python, caption={cn2an}]
      def chinese_to_int(chinese_str):
          arabic_number = cn2an.cn2an(chinese_str, 'normal')
          return arabic_number
    \end{codeblock}

  \subsection{Progress Bar}
    \subsubsection{tqdm module}
    \begin{codeblock}[language=python, caption={tqdm module}]
      from tqdm import tqdm
      import time

      for i in tqdm(range(500), desc="Processing"):
          time.sleep(0.01)

      progress_bar = tqdm(total=2000, desc="Processing")
      for i in range(200):
          for j in range(10):
              progress_bar.update(1)
    \end{codeblock}
