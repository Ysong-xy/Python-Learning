\section{Class}

  \subsection{定义类模板}
    \subsubsection{基本结构}
      \begin{codeblock}[language=python, caption={define a class}]
        class User: 
            '''这是一个学生用户类''' #类的说明
            #以下是静态成员变量,Python中成为类的属性
            student_user = 'Student User' 
            student_number = 0
        
            def __init__(self, name, number): #构造函数
                self.__name=name #这里名字设置为私有成员,这里变量是非静态成员,python称之为实例属性
                self.number = number; 
                User.student_number += 1
                #这里的name,number是非静态成员变量,Python中成为实例的属性
        
            def __str__(self): #输出运算的重载
                return f'{self.number} {self.__name}'
        
            @property
            def name(self):
                return self.__name
        
            @name.setter
            def name(self, name):
                self.__name = name

            def introduce(self): #self类似*this指针
                print("I'm ", self.__name, ", my number is ", self.number, sep='')

            @staticmethod
            def greet(input_student): #静态成员函数
                print(f"Hello, my name's {self.__name}!")

            @classmethod 
            def get_info(cls): #类方法,其第一个变量是类,常用于类型转换,类属性操作等
                print(f"There are {cls.student_number} {cls.student_user} in total.")
      \end{codeblock}

    \subsubsection{访问限制}
      Python没有对属性和方法的访问权限进行限制。为了保证类内部某些属性不被外部访问,可以在属性或方法名前(或前后)加上双下划线。
      \begin{itemize}
        \item \_\_foo\_\_:首尾双下划线表示定义特殊方法,一般是系统定义名字。
        \item \_\_foo:双下划线表示私有成员,只允许所在的类调用。
        \item \_foo:单下划线表示保护成员,即C++中的protected成员。
      \end{itemize}

    \subsubsection{属性}
      Python中,数据成员被称为属性。可以通过@property将一个方法转换为属性,转换后可以直接通过方法名调用该方法,无需添加()。
      这样做可以简化代码,也为属性添加安全保护,即添加了@property的属性是只读的(这是由于return时经过了复制传递)。

      \begin{codeblock}[language=python, caption={property of class}]
        class User: 
            ''' 用户类 ''' #类的说明
            @property
            def name(self);
                return self.__name #这个name属性被设定为只读的
      \end{codeblock}

      @name.setter可以在给属性name赋值时运行
      \begin{codeblock}[language=python, caption={setter of class}]
        @name.setter #这里name是属性名称
        def name(self, name)
            if len(name) < 20:
                self.__name=name
      \end{codeblock}

      @name.deleter属性可以定义一个在删除name属性时执行的函数。
      \begin{codeblock}[language=python, caption={deleter}]
        @name.deleter
        def name(self):
            print('Delete Name!')
            self.__name = None
        del student1.__name
      \end{codeblock}

    \subsubsection{类的特殊成员(魔术方法)}
      \begin{codeblock}[language=python, caption={Special members for class}]
        print(User.__class__) #类的名称
        print(User.__bases__) #类的基类
        print(User.__dict__) #类的数据成员及对应的值,输出一个字典

        def __new__(cls, *args, **kwargs): #创建类的实例
            print(f"Run new with: cls={cls}, args={args}, kwargs={kwargs}")
            return super().__new__(cls) #调用父类的\_\_new\_\_方法,并传入当前类作为参数

        def __del__(self): #对象删除时调用的方法

        def __init__(self, name, number): #接收并初始化实例
            self.__name=name #这里名字设置为私有成员,这里的变量是非静态成员,python中称之为实例属性
            self.number = number; 
            User.student_number += 1
            #这里的name,number是非静态成员变量,Python中成为实例的属性
            #Python中,一个类函数只能有一个\_\_init\_\_函数,若需要多个构造函数,可以用classmethod实现
        #在实例创建时,会先调用\_\_new\_\_在调用\_\_init\_\_

        def __str__(self): #print()和str()的重载
            return f'{self.number} {self.__name}'
        def __repr__(self): #返回一个Python合法的字符串,使之可以用eval重建
            return(f"User({self.__name}, {self.number})")
            #The goal of \_\_repr\_\_ is to be unambiguous
            #The goal of \_\_str\_\_ is to be readable
        student1 = User("Amy", 11); repr_str = repr(student1)
        copy_student1 = eval(repr_str)
        def __format__(self, format_spec): #重载str.format()和f-string

        def __bool__(self): #bool()的重载

        def __add__(self, other): #重载+
        def __iadd__(self, other): #重载+=
        def __sub__(self, other): #重载-
        def __isub__(self, other): #重载-=
        def __mul__(self, other): #重载*
        def __imul__(self, other): #重载*=
        def __truediv__(self, other): #重载/
        def __itruediv__(self, other): #重载/=
        def __floordiv__(self, other): #重载//
        def __mod__(self, other): #重载\%
        def __pow__(self, other): #重载\*\*
        def __eq__(self, other): #重载==
        def __ne__(self, other): #重载!=
        def __lt__(self, other): #重载<
        def __le__(self, other): #重载<=
        def __gt__(self, other): #重载>
        def __ge__(self, other): #重载>=

        def __len__(self): #重载len()
        def __getitem__(self, key): #重载\lbrack\rbrack,读取值
        def __setitem__(self, key, val): #重载\lbrack\rbrack=,修改值
        def __delitem__(self, key): #重载del \lbrack\rbrack,删除值
        def __iter__(self): #返回迭代器,重载iter()
        def __next__(self): #返回迭代器的下一个值,重载next()

        def __call__(self, x): #允许实例像function一样被调用
            return self.__name + x
        #既然class可以像function一样被调用,那class也可以成为\hyperref[subsubsec:decorator]{\underline{decorator}}
        class decorator_class:
            def __init__(self, original_function):
                self.original_function = original_function

            def __call__(self, *args, **kwargs):
                print('call method executed before {}'.format(self.original_function.__name__))
                return self.original_function(*args, **kwargs)
        @decorator_class
        def display():
            print('display function ran')

        def __enter__(self): #进入with语句时自动调用
        def __exit__(self, exc_type, exc_value, traceback): 
        #离开with语句时自动调用(正常和非正常退出都会调用)
        #三个参数分别是异常类型、异常值和异常回溯信息,返回True会一直异常,否则异常会继续传播 
      \end{codeblock}
        
  \subsection{类的使用}
    \subsubsection{创建类的实例}
      \begin{codeblock}[language=python, caption={Create an instance of class}]
        student1 = User('Amy', 11) 
      \end{codeblock}

    \subsubsection{类成员的使用}
      \begin{codeblock}[language=python, caption={Using class members}]
        print(student.student_number); print(User.student_number) #都输出1
        print(student.number); student1.number = 1 #这里number无法修改

        student1.introduce() #输出I'm Amy, my number is 11
        User.introduce(student1) #效果与上一句相同

        User.greet(student1) #输出Hello, my name's Amy!
        student1.greet(student1) #效果与上一句相同

        User.get_info() #输出There are 1 Student User in total.
        student1.get_info() #效果与上一句相同
        print(student1) #这里会调用\_\_str\_\_函数,输出11 Amy
      \end{codeblock}

  \subsection{继承}
    Python的继承默认是公有继承。继承的内容有类的属性,实例方法,类方法,静态方法,特殊方法,
    私有成员不会被继承。类方法在继承时,会以子类作为第一个参数。

    Python中的所有类都继承自builtins.object类。
    \begin{codeblock}[language=python, caption={Class inheritance}]
      class Student(User)
          def __init__(self, grade, name, number):
              super().__init__(name, number) #调用基类的构造函数
              self.__grade = grade
    \end{codeblock}

    子类可以重写父类中的方法,这与C++,Java没什么区别,只是Python没有虚函数机制。

  \subsection{有关类的内置函数}
    由于python中的类在访问和添加成员的操作上限制较少,可以调用一些内置函数来检查这些
    类的合理性。
    \begin{codeblock}[language=python, caption={built-in functions about class}]
      isinstance(object, classinfo) #检查object的类别是不是classinfo
      isinstance(student1, User) #返回True

      issubclass(cls, classinfo) #检查cls类是不是classinfo的子类
      issubclass(Student, User) #返回True

      hasattr(object, name) #检查object是否有name属性
      hasattr(student1, "number") #返回True
      
      callable(object) #检查object是否可以像函数一样被调用
      callable(student1.name) #返回False
      callable(student1.greet) #返回True
    \end{codeblock}

  \subsection{类的典型例子}
    \subsubsection{文件管理类}
      文件管理类可以实现打开创建文件,读写文件,文件异常处理等工作
      \begin{codeblock}[language=python, caption={File management class}]
        class Open_File():

            def __init__(self, filepath, mode):
                self.filepath = filepath
                self.mode = mode

            def __enter__(self):
                print(f"Entering the file: {self.filepath}")
                self.file = open(self.filepath, self.mode)
                return self.file

            def __exit__(self, exc_type, exc_val, traceback)
                print(f"Exiting the file: {self.filepath}")
                if exc_type:
                    print(f"An exception occurred: {exc_type}, {exc_value}")
                self.file.close()

        with Open_File('test.txt', 'w') as f:
            f.write('Testing')
      \end{codeblock}
