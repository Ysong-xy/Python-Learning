\section{Website}

  \subsection{请求连接}
    Python自带的urllib和urllib3模块可以实现一些网络爬虫的常用操作。外接库中的requests库则更为常用。
    \subsubsection{urllib module}
      利用request模块可以实现get请求方式获取网页内容。
      \begin{codeblock}[language=python, caption={Fetch web content using GET method}]
        import urllib.request #网络请求子模块

        response = urllib.request.urlopen('http://www.baidu.com') #打开网页
        html = response.read() #读取网页代码
        response.close()

        #也可以用with语句简化
        with urllib.request.urlopen('http://www.baidu.com') as response:
            html = response.read()
        print(html)
      \end{codeblock}

      也可以实现post请求方式获取网页内容。
      \begin{codeblock}[language=python, caption={Fetch web content using POST method}]
        import urllib.parse #url解析和引用模块
        import urllib.request

        #使用urlencode方法对数据进行处理,并将处理后的数据设置为utf-8编码
        data = bytes(urllib.parse.urlencode({'word':'hello'}), encoding='utf8')
        with urllib.request.urlopen('http://httpbin.org/post', data=data) as response#打开网页
            html = response.read()
        print(html)
      \end{codeblock}

    \subsubsection{urllib3 module}
      urllib3是一个更强大的Python库。
      \begin{codeblock}[language=python, caption={urllib3 module}]
        import urllib3

        #用GET的方式连接
        http = urllib3.PoolManager() #创建对象,用于处理连接和安全等细节
        response = http.request('GET','https://www.baidu.com/') #连接网站
        print(response.data) #输出读取内容

        #用POST的方式连接
        http = urllib3.PoolManager() #创建对象,用于处理连接和安全等细节
        response = http.request('POST', 'http://httpbin.org/post', fields={'word':'hello'})
        print(response.data) #输出读取内容
      \end{codeblock}

    \subsubsection{requests module}
      另外有个更加人性化的第三方库requests,这个库更加常用。
      \begin{codeblock}[language=python, caption={requests module}]
        import requests

        data={'word':'hello'}
        response = requests.get('http://www.baidu.com',params=data) #get方法访问
        response = requests.post('http://httpbin.org/post',data=data) #post方法访问
        if response.ok: #检验连接是否成功
            print(response.content) #以字节流形式输出网页源码(没有换行符和缩进,可读性差)
            print(response.text) #以文本形式输出网页源码(带有换行符和缩进,可读性好)

        #下载文件
        import zipfile
        r = requests.get('<url>') #zip file的url
        with open('data.zip', 'wb') as f: #建一个data.zip用于保存
            f.write(r.content)
      \end{codeblock}

      为了绕开反爬设计,我们可以请求header。
      \begin{codeblock}[language=python, caption={Handling request headers}]
        import requests

        url = 'http://www.bilibili.com/'
        headers = {'User-Agent':'Mozilla/5.0 (Windows NT 10.0; Win64; x64) 
            AppleWebKit/537.36 (KHTML, like Gecko) Chrome/121.0.0.0 Safari/537.36'}
        #这里User-Agent的内容要从网络监视器中复制过来
        response = requests.get(url,headers=headers)
        print(response.content)
      \end{codeblock}

  \subsection{处理html文件}
    BeautifulSoup module用于处理html文件
    \begin{codeblock}[language=python, caption={Handling HTML}]
      from bs4 import BeautifulSoup

      with open('test.html') as html_file: #打开本地的html文件
          soup = BeautifulSoup(html_file, 'lxml') #用lxml解析器解析html文件

      source = requests.get('http://bilibili.com').text
      soup = BeautifulSoup(source, 'lxml') #从网页上获取文件

      print(soup.prettify()) #转换成易读的html文件
      match = soup.title.text #得到文件中的某个部分
      match_div = soup.find('div', class_='footer') #查找文件中第一个对应的内容
      match_all = soup.find_all('div') #查找文件中所有对应的内容,返回一个列表
    \end{codeblock}

  \subsection{Flask}
    Flask是一个常见的网站开发框架。
    \subsubsection{create a website}
      \begin{codeblock}[language=python, caption={create a web site}]
        #my\_app.py
        from flask import Flask, abort
        app = Flask(__name__)

        @app.route("/") #root page
        @app.route("/home") #两个网址共用一个网页
        def hello():
            return "<h1>Hello World!</h1>"

        @app.route("/error") #两个网址共用一个网页
        def hello():
            abort(403) #返回403错误(forbidden)

        if __name__ == '__main__':
            app.run(debug=True)
      \end{codeblock}

      \begin{codeblock}[language=bash, caption={start a web site}]
        export FLASK_APP=my_app.py
        flask run #运行网站,每次修改python文件需要重新启动
        export FLASK_DEBUG=1
        flask run #运行网站,每次修改都会立即出现在网站上

        python my_app.py #直接运行网站
      \end{codeblock}

    \subsubsection{template}
      template就是每个url的html模板,建立template可以避免将html语言写到python文件中造成混乱。

      在下面的例子中,我们将为root page创建一个template/home.html文件,注意这个文件必须放在template文件夹下。
      \begin{codeblock}[language=HTMLwithJinja, caption={template/home.html}]
        <!DOCTYPE html>
        <html>
        <head>
            <meta charset="UTF-8"> <!--提供文档的元数据-->
            <title>网页标题</title>
            <style> <!--定义内部CSS样式-->
                .custom-text {
                    font-family: "Times New Roman", serif;
                    color: green;
                }
            </style> <!--更好的做法是将要用到的style都写到一个.css文件中
        </head>
        <body>
            <header>
                <h1>欢迎来到我的网站</h1>
                <nav>
                    <ul>
                        <li><a href="{{ url_for('home') }}">主页</a></li>
                        <li><a href="#about">关于我们</a></li>
                        <li><a href="/contact">联系我们</a></li>
                    </ul>
                </nav>
            </header>
            <main> <!--main part of body-->
                <h1>这是一个<span style="color:red;">标题<\span></h1> <!--h1至h6表示标题的等级-->
                <p style="custom-text">这是一个段落。</p></br> <!--br为空格-->

                <strong>这是重要的内容。</strong>
                <small>这是一个较小的文本。</small>
                <em>这是一个需要强调的文本。</em>
                <mark>这是一个被标记的文本。</mark>
                <p>H<sub>2</sub>O 表示水。</p> <!--下标文本-->
                <p>2<sup>2</sup> = 4</p> <!--上标文本-->
                <p>这是一个 <code>print("Hello, World!")</code> 示例。</p>

                <a href="https://www.example.com">访问示例网站</a>
                <a href="#">点击这里回到顶部</a>
                <img src="image.jpg" alt="描述文本">
            </main>

            <h2>无序列表</h2>
            <ul> <!--ol有序无标记列表,li有序有标记列表-->
                <li>苹果</li>
                <li>香蕉</li>
                <li>橙子</li>
            </ul>

            <div style="border: 1px solid black; padding: 10px;">
                <h2>新闻标题</h2>
                <p style="font-family: Arial, sans-serif; color: blue;">
                    这是新闻的内容部分。该部分使用`&lt;div&rt;`标签包裹,以便进行样式处理和布局控制。
                </p>
            </div>
            <div>
                <button>Click me!</button> <!--按钮-->
                <input type="text" name="username" placeholder="Enter name"> <!--文本输入框-->
                <input type="password" name="password" placeholder="Password"> <!--密码框-->
                <input type="submit" value="Submit"> <!--提交按钮-->
                <input type="radio" name="gender" value="male"> <!--单选按钮-->
                <input type="radio" name="gender" value="female"> <!--单选按钮-->
                <input type="checkbox" name="agree" value="yes"> <!--复选框-->
            </div> 
        </body>
        </html>
      \end{codeblock}

      html文件中有一些特殊字符需要转义:
        \begin{itemize}
          \item 大于号 小于号<~~>: \&lt;~~\&rt;
          \item 书名号《~~》: \&Lt;~~\&Rt;
          \item 与号 \&: \&amp;
          \item 双引号 ": \&quot;
          \item 单引号 ': \&\#39;
          \item 空格 '': \&nbsp;
          \item 乘号 $\times$: \&times;
          \item 除号 $\div$: \&divide;
        \end{itemize}

      \begin{codeblock}[language=python, caption={using template in my\_app.py}]
        @app.route("/")
        def home():
            return render_template('home.html')

        @app.route("/post/<int:post_id>") #route可以带有变量,这里设置其为int类型
        def post(post_id):
            post = Post.query.get_or_404(post_id) #如果网页存在就返回post\_id,否则返回404
            return render_template('post.html', post=post)
      \end{codeblock}

    \subsubsection{template inheritance}
      同一个网站中template往往有很多重合部分,template inheritance可以实现统一设置模板。

      下面,我们先创建一个layout.html,其中包含网页的公共部分。 
      \begin{codeblock}[language=HTMLwithJinja, caption={layout.html}]
        <head>
            
                <title>My app - {{ title }}</title>
            
                <title>My app</title>
            
        </head>
        <body>
             <!--需要替换的部分-->
        <\body>
      \end{codeblock}

      在home.html中使用layout.html。
      \begin{codeblock}[language=HTMLwithJinja, caption={Using template inheritance}]
         
        
           #读出posts的内容
              <h1>{{ post.name }}</h1> #两个大括号表示代码结果要输出到网页上
              <p>with number of {{ post.number }}</p>
           #结束读取
        
      \end{codeblock}

    \subsubsection{bootstrap}
      bootstrap定义了一些方法来优化网页,参考\href{https://getbootstrap.com/docs/5.3/getting-started/introduction/}{bootstrap官方文档}。
      \begin{codeblock}[language=HTMLwithJinja, caption={bootstrap}]
        <!doctype html>
        <html lang="en">
        <head>
            <meta charset="utf-8">
            <meta name="viewport" content="width=device-width, initial-scale=1">
            <link 
                href="https://cdn.jsdelivr.net/npm/bootstrap@5.3.3/dist/css/
                    bootstrap.min.css" 
                rel="stylesheet" 
                integrity="sha384-QWTKZyjpPEjISv5WaRU9OFeRpok6YctnYmD
                    r5pNlyT2bRjXh0JMhjY6hW+ALEwIH" 
                crossorigin="anonymous">

            
                <title>My app - {{ title }}</title>
            
                <title>My app</title>
            
        </head>
        <body>
            <div class="container" mt-1>  <!--利用bootstrap中的设定-->
                <!--外间距:mt上间距,mb下间距,my上下间距-->
                <!--外间距:ml左间距,mr右间距,mx左右间距-->
                <!--内间距:pt左间距,pr右间距,px左右间距-->
                <!--内间距:pl左间距,pr右间距,px左右间距-->
                 <!--需要替换的部分-->
            </div>

            <script 
                src="https://cdn.jsdelivr.net/npm/bootstrap@5.3.3/dist/js/
                    bootstrap.bundle.min.js" 
                integrity="sha384-YvpcrYf0tY3lHB60NNkmXc5s9fDVZLESaAA55NDz
                    Oxhy9GkcIdslK1eN7N6jIeHz" 
                crossorigin="anonymous">
            </script>
        </body>
        </html>
      \end{codeblock}
    
    \subsubsection{External CSS}
      External CSS(外部样式表)将网页中文本、图片等元素所需样式都写在static/main.css文件里备用。
      main.css中可以为每个板块类定义样式,这个样式会自动生效于网页中所有对应的类。
      
      下面这个样例表示在`article-metadata'类的元素内的链接(<a>标签)当鼠标悬停(hover)时的样式。
      \begin{codeblock}[language=HTMLwithJinja, caption={static/main.css}]
        .article-metadata a:hover {
          color: #333;
          text-decoration: none;
        }
      \end{codeblock}

      导入External CSS需要用到url\_for module
      \begin{codeblock}[language=python, caption={url\_for}]
        from flask import Flask, render_template, url_for

        url_for('home') #返回home函数对应的url
        url_for('static', filename='picture1.jpg') #在static文件夹下picture1.jpg的地址
        url_for('post', post_id=1) #返回url:~~/post/1,用于decorator里有变量的情况
      \end{codeblock}

      在layout.html中调用url\_for导入样式表。
      \begin{codeblock}[language=python, caption={layout.html}]
        <head>
            <link rel="stylesheet" type="text/css" 
                  href="{{ url_for('static', filename='main.css') }}">
            <!--static为默认目录,main.css只要在这个目录下面,url_for都会自动找到该文件-->
        </head>
      \end{codeblock}

    \subsubsection{post data}
      template中可以通过post传递变量,以实现信息交换。
      \begin{codeblock}[language=python, caption={post data in flask}]
        posts = [ #posts是一个dict组成的list
            {'name': 'Amy', 'number': 11},
            {'name': 'Jack', 'number': 28}
        ]

        def home():
            return render_template('home.html', posts=posts, title="Home page")
      \end{codeblock}

      \begin{codeblock}[language=HTMLwithJinja, caption={use posts in html}]
        <head>
            
                <title>My app - {{ title }}</title>
            
                <title>My app</title>
            
        </head>
        <body>
             #读出posts的内容
                <h1>{{ post.name }}</h1> #两个大括号表示代码结果要输出到网页上
                <p>with number of {{ post.number }}</p>
             #结束读取
        <\body>
      \end{codeblock}

    \subsubsection{form \& field}
      form表示网页上的表单,field表示提供用户输入的区域。一个form可以有多个输入。
      flask-wtf module可用于处理表单。其中wtforms提供了各种表单区域,详见\href{https://wtforms.readthedocs.io/en/3.1.x/fields/}{\underline{wtf field}}。
      wtforms.validators提供了检验输入法的方法,详见\href{https://wtforms.readthedocs.io/en/3.1.x/validators/}{\underline{wtf validators}}。

      下面,我们建立一个forms.py文件来处理表单
      \begin{codeblock}[language=python, caption={form.py}]
        from flask_wtf import FlaskForm
        from wtforms import StringField, PasswordField, SubmitField, BooleanField
            #这里常用的还有还有TextAreaField文本编辑区域
        from wtforms.validators import DataRequired, Length, Email, EqualTo, ValidationError
        from models import User #models将在下一目建立

        class RegistrationForm(FlaskForm):
            username = StringField('Username', 
                validators=[DataRequired(), Length=(min=2, max=20)])
            email = StringField('Email', #这里的第一个变量就是下面html文件中的label
                validators=[DataRequired(), Email()])
            password = PasswordField('Password', 
                validators=[DataRequired()])
            confirm_password = PasswordField('Confirm Password', 
                validators=[DataRequired(), EqualTo('password')])
            remember = BooleanField('Remember Me')
            submit = SubmitField('Sign Up')

            def validate_username(self, username): #Validate函数会在表单提交时自动调用
                user = User.query.filter_by(username=username.data).first()
                if user: #用户名已存在
                    raise ValidationError('That username is taken.')

            def validate_email(self, email):
                user = User.query.filter_by(email=email.data).first()
                if user: #用户名已存在
                    raise ValidationError('That email is taken.')
      \end{codeblock}

      创建register.html显示表单。<form>用于建立表单,<fieldset>用于将表单中的内容进行分组。
      \begin{codeblock}[language=HTMLwithJinja, caption={register.html}]
         
        
            <div class="content-section">
                <form method="POST" action=""> <!--action=""表示将表单数据提交到当前URL-->
                    {{ form.hideen_tag() }} <!--防止跨站请求伪造攻击-->
                    <fieldset class="form-group">
                        <legend class="border-bottom mb-4">Join Today</legend>
                        <!--legend 一般用于fieldset内部,作为注释-->
                        <div class="form-group">
                            {{ form.username.label(class="form-control-label") }}
                            
                                {{ 
                                    form.username(
                                        class="form-control form-control-lg is invalid"
                                    ) 
                                }}
                                <div class="invalid-feedback">
                                    
                                        <span>{{ error }}</span>
                                    
                                </div>
                            
                                {{ form.username(class="form-control form-control-lg") }}
                            
                        </div>
                        <div class="form-group">
                            {{ form.email.label(class="form-control-label") }}
                            
                                {{ 
                                    form.email(
                                        class="form-control form-control-lg is invalid"
                                    ) 
                                }}
                                <div class="invalid-feedback">
                                    
                                        <span>{{ error }}</span>
                                    
                                </div>
                            
                                {{ form.email(class="form-control form-control-lg") }}
                            
                        </div>
                        <div class="form-group">
                            {{ form.password.label(class="form-control-label") }}
                            
                                {{ 
                                    form.password(
                                        class="form-control form-control-lg is invalid"
                                    )
                                }}
                                <div class="invalid-feedback">
                                    
                                        <span>{{ error }}</span>
                                    
                                </div>
                            
                                {{ form.password(class="form-control form-control-lg") }}
                            
                        </div>
                        <div class="form-group">
                            {{ form.confirm_password.label(class="form-control-label") }}
                            
                                {{ 
                                    form.confirm_password(
                                        class="form-control form-control-lg is invalid"
                                    ) 
                                }}
                                <div class="invalid-feedback">
                                    
                                        <span>{{ error }}</span>
                                    
                                </div>
                            
                                {{ form.confirm_password(class="form-control 
                                    form-control-lg") }}
                            
                        </div>
                    </fieldset>
                    <div class="form-group">
                        {{ form.submit(class="btn btn-outlne-info") }}
                    </div>
                </form>
            </div>
            <div class="border-top pt-3">
                <small class="text-muted">
                    Already Have An Account? 
                    <a class="ml-2" href="{{ url_for('login') }}">Sign In</a>
                </small>
            </div>
        
      \end{codeblock}
        
      在app.py中设置密码(防止用户信息泄露),调用表单,接收用户提交的表单,并在处理表单后重定向。
      \begin{codeblock}[language=python, caption={Using forms in main.py}]
        from forms import RegistrationForm
        from flask import Flask, render_template, url_For, redirect

        app.config['SECRET_KEY'] = '114514' #密码可以由secrets module生成

        @app.route("/register", methods=['GET', 'POST']) #接收GET和POST表单
        def register():
            form = RegistrationForm()
            if form.validate_on_submit(): #判断提交表单是否合理
                flash(f'Account created for {form.username.data}!', 'success') #显示一次性消息
                    #消息类别有"error""danger""warning""success"等
                return redirect(url_for('home'))
            return render_template('register.html', title='Register', form=form)
      \end{codeblock}

      在layout.html中显示flash的内容并完成重定向。
      \begin{codeblock}[language=HTMLwithJinja, caption={layout.html}]
         
            <!--success就是true-->
            
                
                    <div class="alert alert-{{ category }}">
                        {{ messages }}
                    </div>
                
            
        
      \end{codeblock}

    \subsubsection{flask-sqlalchemy}
      flask-sqlalchemy可以用于存储网站运行产生的数据。

      创建一个models.py文件,用于处理数据库。
      \begin{codeblock}[language=python, caption={Using sqlalchemy in models.py}]
        from flask_sqlalchemy import SQLAlchemy
        from datetime import datetime
        from app import db

        app.config['SQLALCHEMY_DATABASE_URI'] = 'sqlite:///site.db' #///表示相对路径
        db = SQLAlchemy(app)

        class User(db.Model): #创建一个User表
            id = db.Column(db.Integer, primary_key=True) #不重复的key,会自动生成,从1开始
            username = db.Column(db.String(20), unique=True, nullable=False) 
                #用户名不重复且不为空
            email = db.Column(db.String(120), unique=True, nullable=False) 
            image_file = db.Column(db.String(20), nullable=False, default='default.jpg') 
                #图片用hash值记录
            password = db.Column(db.String(60), nullable=False) #密码用hash值记录
            posts = db.relationship('Post', backref='auther', lazy=True) 
                #记录Post和User的关系
                #这里Post指代下面的Post类型,lazy=True会使实例直接带有relationship信息

            def __repr__(self): 
                return f"User('{self.username}', '{self.email]', '{self.image_file}')"

        class Post(db.Model): #创建一个Post表
            id = db.Column(db.Integer, primary_key=True)
            title = db.Column(db.String(100), nullable=False)
            date_posted = db.Column(db.DateTime, nullable=False, default=datetime.utcnow)
                #注意这里utcnow是一个函数而不是函数的返回值,否则default会立即调用,生成固定的时间
            content = db.Column(db.Text, nullable=False) #长文本类型
            user_id = db.Column(db.Integer, db.ForeignKey('user.id'), nullable=False)
                #这里user.id指代user表中的id列(表名是类名的小写)
                #表示user\_id是user表id列的一个外键

            def __repr__(self):
                return f"Post('{self.title}', '{self.date_posted]')"
      \end{codeblock}

      在app.py文件中使用models.py
      \begin{codeblock}[language=python, caption={Using models in main.py}]
        from models import User, Post

        app.config['SQLALCHEMY_DATABASE_URI'] = 'sqlite:///site.db' #///表示相对路径
        db = SQLAlchemy(app) #database和app绑定,再次运行不会生成新的database 
      \end{codeblock}

      在shell创建数据库文件,并尝试加入数据。
      \begin{codeblock}[language=python, caption={create database file in terminal}]
        >>> from app import db
        >>> db.create_all()

        >>> from app import User, Post
        >>> user_1 = User(username='Jack', email='J@demo.com', password='password')
        >>> db.session.add(user_1)
        >>> db.session.commit()

        >>> User.query.all() #输出所有User
        >>> User.query.first() #输出第一个User
        >>> User.query.filter_by(username='Jack').all()
        >>> user = User.quety.get(1) #按照id查找

        >>> post_1 = Post(title='Blog 1', content='First Post Content!', user_id=user.id)
        >>> db.session.add(post_1)
        >>> db.session.commit()
        >>> user.posts #此时可以输出用户的posts
        >>> db.query.first().author #利用backref查询作者

        >>> db.drop_all() #删除数据库中的所有内容
      \end{codeblock}

      将website产生的用户数据存入database,这里用到了hash生成,见下一目。
      \begin{codeblock}
        @app.route("/register", methods=['GET', 'POST']) 
        def register():
            form = RegistrationForm()
            if form.validate_on_submit(): 
                hash_password = bcrypt.generate_password_hash(
                    form.password.data).decode('utf-8')
                user = User(username=form.username.data, 
                            email=form.email.data, 
                            password=hashed_password)
                db.session.add(user)
                db.session.commit()
                flash(f'Account created for {form.username.data}! You can log in now', 
                    'success') 
                return redirect(url_for('login'))
            return render_template('register.html', title='Register', form=form)
      \end{codeblock}

    \subsubsection{Bcript \& LoginManager}
      flask-bcrypt.Bcrypt可用于生成hash值。flask\_login.LoginManager用于控制用户登录
      \begin{codeblock}[language=python, caption={Using Bcript in app.py}]
        from flask import request #这将帮助我们得到GET的内容,以便重定向
        from flask_bcrypt import Bcrypt
        from flask_login import LoginManager, UserMixin, login_user, 
                                current_user, logout_user, login_required
            #这些变量可以在html中直接访问,无需通过render\_template传递

        bcrypt = Bcrypt(); bcrypt.init_app(app)
        bcrypt = Bcrypt(app) #上面两句是一样的
        login_manager = LoginManager(app)
        login_manager.login_view = 'login' #指向login函数
        #当没有login而访问需要login的url时,会被重定向到login,同时通过GET传递重定向前的route
        login_manager.login_message_catagory = 'info' #info是一个内置类
        #这一句会改变重定向时提示的样式

        pwd_hash = bcrypt.generate_password_hash('password').decode('utf-8') 
            #这个函数重复运行会生成不同的hash值
        bcrypt.check_password_hash(pwd_hash, 'password') #返回True
        bcrypt.check_password_hash(pwd_hash, 'Testing') #返回False

        @login_manager.user_loader
        def load_user(user_id):
            return User.query.get(int())

        class User(db.Model, UserMixin):
            #defination of id etc.

        @app.route("/login")
        def login():
          user = User() #add name email etc.  
          if current_user.is_authenticated:
              return redirect(url_for('home'))
          else:
              login_user(user) #user会载入current\_user,后面可访问,如current\_user.username
              next_page = request.args.get('next') #得到重定向前的route
              return redirect(next_page) if next_page else redirect(url_for('home'))

        @app.route("/account")
        @login_required #需要login才能访问
        def account():
            form = UpdateAccontForm() #修改用户信息的表单
            if form.validate_on_submit():
                current_user.username = form.username.data #form要加data
                current_user.email = form.email.data
                db.session.commit() #直接修改current\_user即可
                flash('Your account has been updated!', 'success')
                return redirect(url_for('account'))
            elif request.method == 'GET':
                form.username.data = current_user.username
                form.email.data = current_user.email
            image_file = url_for('static', 
                filename='profile_pics/' + current_user.image_file)
            return render_template('account.html', form=form, image_file=image_file)
      \end{codeblock}

    \subsubsection{FileField}
      FileField允许网页传输文件(图片、文本文件等)。FileAllowed设置了允许传输的文件类型。
      \begin{codeblock}[language=python, caption={FileField}]
        from flask_wtf.file import FileField, FileAllowed
        import secrets

        def save_picture(form_picture):
            random_hex = secrets.token_hex(8) #生成随机文件名,防止同名文件bug
            _, f_ext = os.path.splitext(form_picture.filename)
            picture_fn = random_hex + f_ext
            picture_path = os.path.join(app.root_path, 'static/profile_pics', picture_fn)
            form_picture.save(picture_path)
            return picture_fn

        class UpdateAccountForm(FlaskForm):
            picture = FileField('Update Profile Picture', 
                                validators=[FileAllowed(['jpg', 'png'])])

        @app.route("/account", methods=['GET', 'POST'])
            if form.validate_on_submit():
                if form.picture.data:
                    current_user.picture = save_picture(form.picture.data)
      \end{codeblock}

      在html文件中调用FileField。
      \begin{codeblock}[language=HTMLwithJinja, caption={Using FileField in html}]
        <form method="POST" action="" enctype="multipart/form-data">
        <!--这里必须给出编码方式enctype-->
          <fieldset>
            <div class="form-group">
                {{ form.picture.label() }}
                {{ form.picture(class="form-control-file") }}
                
                    
                        <span class="text-danger">{{ error }}</span></br>
                    
                
            </div>
          </fieldset>
        </form>
      \end{codeblock}

    \subsubsection{Pagination}
      Pagination可以减少post的内容,提高网站运行速度,同时避免页面过长,增加可读性。
      为了在分页时网页的内容不混乱,我们还需要对post的内容进行排序。
      \begin{codeblock}[language=python, caption={paginate}]
        @app.route("/")
        def home():
            page = request.args.get('page', default=1, type=int)
            posts = Post.query.order_by(Post.data)paginate(per_page=10, page=page) 
            #以10个为1页,访问第page页
            return render_template('home.html', posts=posts)
      \end{codeblock}

      在html文件中,可以设置一个导航栏,导航到指定页码,方便随机访问。
      \begin{codeblock}[language=HTMLwithJinja, caption={Using pagination in home.html}]
        
            <!--处理当前page的post内容-->
        

        <!--导航栏-->
        
        <!--这是一个预设,仅开头结尾和当前page前后的page_num会显示,其他page_num会缩成一个None-->
        <!--right_current包含current page,故要比left_current多1-->
            
                 <!--把当前page的button设计的不一样-->
                    <a 
                        class="btn btn-info mb-4" 
                        href="{{ url_for('home', page=page_num) }}">
                        {{ page_num }}
                    </a>
                
                    <a 
                        class="btn btn-outline-info mb-4" 
                        href="{{ url_for('home', page=page_num) }}">
                        {{ page_num }}
                    </a>
                
            
                ... <!--缩成None的page_num显示成...-->
            
        
      \end{codeblock}
