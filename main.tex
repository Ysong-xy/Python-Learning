\documentclass{article}
\usepackage{xeCJKfntef}
\usepackage[english]{babel}
\usepackage{fancyhdr}
\usepackage{booktabs}
\usepackage{longtable}
\usepackage{threeparttable}
\usepackage{threeparttablex}
\usepackage{colortbl}
\usepackage[table]{xcolor}
\usepackage{multirow}
\usepackage{makecell}
\usepackage{ctex}
\usepackage{amsmath}
\usepackage{float}
\usepackage{enumerate}
\usepackage{listings}
\usepackage{chngcntr}
\usepackage{geometry}
\usepackage[hidelinks=true]{hyperref}

\geometry{
  left=3.5cm, % 设置页边距
  right=3.5cm,
  top=5cm,
  bottom=5cm,
  headheight=15pt, % 对页眉页脚有特殊要求时设定
  headsep=20pt, % 页眉和正文之间的间距
  footskip=20pt % 页脚和正文之间的间距
}

\linespread{1.0}

\setmainfont{Times New Roman} %.正文字体
\setsansfont{Arial} %.斜体字体 
\setmonofont{Consolas} %.等宽字体
\setCJKmainfont{KaiTi}
\setCJKsansfont{SimHei}
\setCJKmonofont{SimHei}

\lstdefinestyle{lstStyleCode}{
  tabsize=4,
  frame=shadowbox, %把代码用带有阴影的框圈起来
  framerule=0.15pt,
  numbers=left,
  numberstyle=\tiny\color{black},
  rulesepcolor=\color{red!20!green!20!blue!20},
  aboveskip         = \medskipamount,
  belowskip         = \medskipamount,
  basicstyle        = \ttfamily\zihao{6},
  commentstyle      = \slshape\color{black!60},
  stringstyle       = \color{green!40!black!100},
  keywordstyle      = \bfseries\color{blue!50!black},
  extendedchars     = false,
  upquote           = true,
  showstringspaces  = false,
  xleftmargin       = 1em,
  xrightmargin      = 1em,
  breaklines        = false,
  framexleftmargin  = 1em,
  framexrightmargin = 1em,
  backgroundcolor   = \color{gray!6},
  columns           = flexible,
  keepspaces        = true,
  texcl             = true,
  mathescape        = true
}
\lstnewenvironment{codeblock}[1][]{%
  \lstset{style=lstStyleCode,#1}}{}

%. 设置计数器格式,以章节进行计数。
\renewcommand{\thesection}{\arabic{section}}
\renewcommand{\thesubsection}{\thesection.\arabic{subsection}}
\renewcommand{\thesubsubsection}{\thesubsection.\arabic{subsubsection}}
\renewcommand{\thetable}{\arabic{table}}
\renewcommand{\thefigure}{\arabic{figure}}
\counterwithin{table}{section}
\counterwithin{figure}{section}
%. 更改表格名称为Table,图片更名为Figure
\renewcommand{\tablename}{Table}
\renewcommand{\figurename}{Figure}
\addto\captionsenglish{\renewcommand{\contentsname}{目录}} %. 设置目录大标题名称


\pagestyle{fancy}
\fancyhf{}
\rhead{\leftmark}
\lhead{\hyperlink{contents}{Python学习笔记}}
\cfoot{\textbf{\thepage}}
\renewcommand{\footrulewidth}{0.4pt} %默认为0,即没有分割线

\title{\huge Python学习笔记}
\author{\large Yusong}
\date{}

\begin{document}
  \counterwithin{lstlisting}{section}

  \maketitle
  \thispagestyle{empty}

  \newpage
  \hypertarget{contents}{}
  \setcounter{page}{1}
  \pagenumbering{Roman}
  \tableofcontents

  \newpage
  \setcounter{page}{1}
  \pagenumbering{arabic}

  \section{Basic grammar}

    \subsection{type}
      \begin{codeblock}[language=python, caption={variable type}]
        >>> str = "python"
        >>> dir(str) #获取相关用法
        >>> type(str) #获取变量类型
        >>> help(str) #获取guide
        >>> isinstance(str, string) #若str为string类型,则返回True,否则返回False
      \end{codeblock}

      \begin{codeblock}[language=python, caption={basic functions}]
        print('Hello World!')
        print('Hello', 'World!', sep=', ', end='') #各字符串间用`, '分隔,结尾没有换行符
        print(round(3.75, 1)) #四舍五入保留一位小数
        float_number = 3.3; int_number = int(float_number)
      \end{codeblock}

    \subsection{condition statements}
      \begin{codeblock}[language=python, caption={condition statement}]
        def absolute_value(x):
          """Return the absolute value of x."""
          if x < 0:
            return -x
          elif x == 0:
            return 0
          else:
            return x

        def absolute_value(x):
          return -x if x < 0 else x
      \end{codeblock}

      \begin{itemize}
        \item False values in Python: False, 0, ' ', None
        \item True values in Python: Anything else
      \end{itemize}

    \subsection{iteration}
      \begin{codeblock}[language=python, caption={iteration}]
        i, total = 0, 0
        while i < 3:
          total += i
          i += 1

        for i in range(3):
          total += i
      \end{codeblock}

    \subsection{function}
      函数用于处理需要多次重复运行的同一任务。python的函数默认返回None。
      \subsubsection{定义函数}
        \begin{codeblock}[language=python, caption={definition of functions}]
          def my_func(string1,string2):
            print(string1)
            print(string2)

          def my_func1(string1='Hello',string2): #可以指定默认值
            ''' 功能:打招呼
                string1:默认hello
                string2:对象名字 '''
            print(string1)
            print(string2)
            return 1

          def my_info(*args, **kwargs): #这里*args会创建一个元组,kwargs会创建一个字典
            print(args)
            print(kwargs)

          #可以使用yield函数返回,yield允许函数在返回一个值的同时保存状态,并在下一次继续执行
          def count_up_to(n):
              counter = 1
              while counter <= n:
                  yield counter
                  counter += 1

          #调用带有yield关键字的函数时,它会返回一个生成器对象
          gene = count_up_to(5)
          print(next(gene))
          for number in gene:
              print(number)
          #当生成器函数不再遇到yield语句时,生成会终止
        \end{codeblock}

      \subsubsection{调用函数}
        以下是几种正确的调用方法
        \begin{codeblock}[language=python, caption={use of function}]
          my_func('Hello','world')
          str1='Hello';str2='world'
          my_func(str1,str2)
          my_func(string1=str1,string2=str2) #写明形式参数名称为关键字参数
          my_func(str1,string2=str2) #允许前面的形参不写名称,后面的写明
          my_func(string2=str2,string1=str1) #写明形式参数名时可以交换顺序
          info_value = ['Math', 'Art']; dic_value = {'name': 'John', 'age': 22}
          my_info('Math', 'Art', name='John', age=22)
          my_info(*info_value, **dic_value)
        \end{codeblock}

        以下是几种错误的调用方法
        \begin{codeblock}[language=python, caption={mistakes in using functions}]
          my_func(string1=str1,str2) #前面写明形参名称,后面必须写
          my_func(string2=str2,str1)
        \end{codeblock}

      \subsubsection{可变参数}
        可变参数是Python特有的设计,可变参数也被称为不定长参数,即传入函数中的实际参数可以是0个、1个或多个。
        \begin{codeblock}[language=python, caption={variable parameters}]
          def greet(*names): #这里会将输入接受并放在一个元组中
            print("Hello",end=' ')
            for item in names:
              print(item,end=' ')

          if __name__=='__main__':
            greet('Tom','Jerry')

            param=['Tom','Jerry'] #也可以调用列表
            greet(*param) #这里不能写形参名
        \end{codeblock}

        另一种方法会将输入以``形参名:变量名''的形式存为一个字典。
        \begin{codeblock}[language=python, caption={other variable parameters}]
          def greet(**names): #输入转为字典
            print('Hello')
            for key,value in names.items():
              if value=='male':
                print('Mr',end=' ')
              else:
                print('Miss',end=' ')
              print(key)

          if __name__=='__main__':
            greet(Tom='male',Jerry='female') #直接调用

            dict={'Tom':'male','Jerry':'female'} #也可以用字典调用
            greet(**dict) #这里也不能用关键字参数
        \end{codeblock}

      \subsubsection{静态类型函数}
        Python的函数默认是动态类型的,可以接受所有类型的输入。当然也可以将函数定为静态类型的(和C++类似)。
        \begin{codeblock}[language=python, caption={static functions}]
          def square(number:int|float)->int|float:
            return num**2
        \end{codeblock}

      \subsubsection{变量的作用域}
        \begin{itemize}
          \item 在函数内的定义的变量是局部变量,外部不能调用。
          \item 函数外部定义的变量是全局变量,可以在任意位置调用该变量。
          \item 函数内定义的变量可以通过global关键字定为全局变量。
          \item environment就是frame的顺序,当调用某变量时,会从里向外逐层寻找该变量。
          \item 在一个函数中,一个变量名只能始终为一个全局变量,或者始终是一个局部变量。
          \item 在内部函数中的变量名前加上nonlocal,则内部函数的该变量与外部函数的变量是同一个。
        \end{itemize}

        \begin{codeblock}[language=python, caption={local and global}]
          message='Hello'
          def my_func():
            global message
            cnt = 0
            def in_func():
              nonlocal cnt
            message='World'
            cnt++
          print(message) #输出'World'

          #下面这段代码会报错
          x = 'global x'

          def test():
            print(x)
            x = 'local x'
            print(x)
          
          test()
          #解决这个报错的方法有两种,即函数内声明全局或不使用全局
          #可以在第一个print(x)前面加上global x
          #也可以删去第一个print(x)
        \end{codeblock}

      \subsubsection{High-order function}
        High-order function是一类返回函数的函数,用于处理同类型的任务。
        \begin{codeblock}[language=python, caption={High-order function}]
          def make_adder(n):
            """Return a function that takes K and return K + N.

            >>> add_three = make_adder(3)
            >>> add_three(4)
            """
            def adder(k):
              return k + n
            return adder

          add_three = make_adder(3)
          print(make_adder(100)(4))
        \end{codeblock}

        High-order function在音频领域经常被使用,用于制作波形。
        \begin{codeblock}[language=python, caption={mario}]
          from wave import open
          from struct import Struct
          from math import floor

          frame_rate = 11025

          def encode(x):
              i = int((1 << 14) * x)
              return Struct('h').pack(i)

          def play(sampler, name='song.wav', seconds=2):
              out = open(name, 'wb')
              out.setnchannels(1)
              out.setsampwidth(2)
              out.setframerate(frame_rate)
              t = 0
              while t < seconds * frame_rate:
                  sample = sampler(t)
                  out.writeframes(encode(sample))
                  t += 1
              out.close()

          def tri(frequency, amplitude=0.3):
              period = frame_rate // frequency
              def sampler(t):
                  saw_wave = t / period - floor(t / period + 0.5)
                  tri_wave = 2 * abs(2 * saw_wave) - 1
                  return amplitude * tri_wave
              return sampler

          c_freq, e_freq, g_freq = 261.63, 329.63, 392.00

          def both(f, g):
              return lambda t: f(t) + g(t)

          def note(f, start, end, fade=0.01):
              def sampler(t):
                  seconds = t / frame_rate
                  if seconds < start:
                      return 0
                  elif seconds > end:
                      return 0
                  elif seconds > end - fade:
                      return (end - seconds) / fade * f(t)
                  elif seconds < start + fade:
                      return (seconds - start) / fade * f(t)
                  else:
                      return f(t)
              return sampler

          c, e, g = tri(c_freq), tri(e_freq), tri(g_freq)
          g_low = tri(g_freq / 2)
          z = 0
          song = note(e, z, z + 1/8)
          z += 1/8
          song = both(song, note(e, z, z + 1/8))
          z += 1/4
          song = both(song, note(e, z, z + 1/8))
          z += 1/4
          song = both(song, note(c, z, z + 1/8))
          z += 1/8
          song = both(song, note(e, z, z + 1/8))
          z += 1/4
          song = both(song, note(g, z, z + 1/8))
          z += 1/2
          song = both(song, note(g_low, z, z + 1/8))
          z += 1/2

          play(song)
        \end{codeblock}

      \subsubsection{self-reference}
        self-reference function会返回自身或者与自身相关的函数,这样可以递归的执行不定长度的任务。
        \begin{codeblock}[language=python, caption={self-reference}]
        def print_all(x):
          print(x)
          return print_all
        print_all(1)(2)(3) #这里会全部输出来
        \end{codeblock}

      \subsubsection{decorator}
        \label{subsubsec:decorator}
        decorator是一种用于修改函数或方法行为的工具,它接受另一个函数作为输入,并返回一个新的函数。decorator可以叠加多层
        \begin{codeblock}[language=python, caption={decorator}]
          def decorator_function(original_function):
              def wrapper_function():
                  print('wrapper executed before {}'.format(original_function.__name__))
                  return original_function()
              return wrapper_function

          @decorator_function #将装饰器应用到display函数上,使display自动被装饰器的逻辑包裹
          def display():
              print('display function ran')

          #display = decorator\_function(display)装饰器的效果和这一句相同

          display() #上面的两句都会打印
          
          #若待装饰的function是有参数的,则需要给装饰器的返回值也添加上参数
          def decorator_function(original_function)
              def wrapper_function(*args, **kwargs)
                  print('wrapper executed before {}'.format(original_function.__name))
                  return original_function(*args, **kwargs)
              return wrapper_function

          #decorator也可以带有别的参数
          def prefix_decorator(prefix):
            def decorator_function(original_function):
                def wrapper_function():
                    print(prefix, 'wrapper executed before {}'.format(original_function.__name__))
                    return original_function()
                return wrapper_function
            return decorator_function

          @prefix_decorator('TESTING:')
          def display():
              print('display function ran')
        \end{codeblock}

      \subsubsection{function currying}
        function currying将一个接受多个参数的函数分解为一系列每个只接受一个参数的函数,即一个函数链。
        \begin{codeblock}[language=python, caption={function currying}]
          def curry2(f):
            def g(x):
              def h(y):
                return f(x, y)
              return h
            return g
            
          from operator import add
          m = curry2(add)
          add_three = m(3)
          print(add_three(2))
        \end{codeblock}
    
      \subsubsection{匿名函数}
        Python的匿名函数也就是lambda表达式。匿名函数可以是另一个函数的参数,如sort函数的key变量。

        \begin{itemize}
          \item 只有def得到的函数会有一个本征名,在shell中输入变量名即可看到这一差异。
          \item def函数可以有多步运行过程,而lambda表达式只能有一个表达式。
        \end{itemize}

        语法为result=lambda \lbrack arg1\lbrack arg2,$\cdots$,argn\rbrack \rbrack:expression
        \begin{codeblock}[language=python, caption={lambda function}]
          result = lambda r:math.pi*r*r
          area = (lambda r: math.pi * r * r)(3)
          print(result(5))

          def area(r):
            return r * r * math.pi
          result = area
        \end{codeblock}

    \subsection{input and output}
      \subsubsection{打开关闭文件}
        \begin{codeblock}[language=python, caption={open and close files}]
          file = open(<filename>[,mode]) #打开文件
          print(file.name); print(file.mode) #输出文件名和打开方式
          file.close() #关闭文件
          print(file.closed) #检查文件是否已经关闭,已关闭会返回True

          #用with语句打开文件可以确保文件被正确关闭,且可以避免文件打开错误引起的异常
          with open(<filename>[,mode]) as file:
        \end{codeblock}

        \begin{table}[H]
          \centering
          \caption{Methods of Opening a File}
          \label{tab:Methods of Opening a File}
          \begin{tabular}{ccc}
            \toprule[1.5pt]
            符号 & 说明 & 注意 \\
            \midrule
            r & 只读模式,指针放在开头 & \multirow{4}*{文件必须存在} \\
            rb & 二进制只读模式 & ~ \\
            r+ & 可以读取,也可以从文件的开头覆盖 & ~ \\
            rb+ & 二进制读写模式 & ~ \\
            w & 只写模式 & \multirow{4}*{\makecell{若文件存在,\\则覆盖,否则\\创建新文件}} \\
            wb & 二进制只写模式 & ~ \\
            w+ & 只读模式打开文件并清空 & ~ \\
            wb+ & 二进制只读模式并清空文件 & ~\\
            a & 追加模式 & \multirow{4}*{\makecell{若文件存在,\\则指针放在末尾,\\否则创建文件}} \\
            ab & 二进制追加模式 & ~ \\
            a+ & 追加且可读(注意指针位置) & ~ \\
            ab+ & 二进制追加且可读 & ~ \\
            \bottomrule[1.5pt]
          \end{tabular}
        \end{table}

      \subsubsection{读取写入文件}
        \begin{codeblock}[language=python, caption={read and write files}]
          #读取文件
          with open('test1.txt', 'r', encoding='utf-8') as file:
              f_contents = file.read(); f_contents = file.read(100) #读取文件前100Byte的内容
              f_list = file.readlines()) #按行切分并组成列表
              f_list = file.readline() #只读一行,结尾带有换行符
              for line in file:
                  print(line, end='')遍历每一行
              file.tell() #返回cursor在文件中的位置
              file.seek(7) #移动cursor到第7Byte的位置
          #写入文件
          with open('test2.txt', 'w') as f:
              f.write('Test') 
              f.seek(0); f.write('Hello World!') #这里的内容会覆盖原有的内容

          #读写非文本文件,可以用二进制读写,下面以创建一个图片副本为例
          with open('puppy.jpg', 'rb') as rf:
              with open('puppy.jpg', 'wb') as wf:
                  chunk_size = 4096
                  rf_chunk = rf.read(chunk_size)
                  while len(rf_chunk) > 0:
                      wf.write(rf_chunk)
                      rf_chunk = rf.read(chunk_size)
        \end{codeblock}

        read()和write()函数还有以下可选参数:
        \begin{itemize}
          \item offset:移动字符个数
          \item whence:指定从什么位置开始计算,0表示开头,1表示当前位置,2表示文件末尾(只能在二进制模式下使用)
        \end{itemize}

      \subsubsection{上下文管理器}
        上下文管理器就是可以在with语句中被调用的特殊函数,可以自动管理资源的分配和释放。
        利用contextmanager decorator可以将普通的函数转换成上下文管理器。
        \begin{codeblock}[language=python, caption={contextmanager}]
          import contextlib

          @contextlib.contextmanager
          def open_file(file, mode):
              try:
                  f = open(file, mode) #yield之前的语句将在进入with语块时执行
                  yield f
              finally:
                  f.close() #yield之后的语句将在退出with语块时执行

          with open_file('test.txt', 'w):
              f.write('Hello, world!')
        \end{codeblock}

      \subsubsection{处理.csv文件}
        csv是一种表格类型的文件,其每行各单元之间有分隔符,默认分隔符是`,'。若单元格内部也有域分隔符相同的字符,
        该单元格会带引号。
        \begin{codeblock}[language=python, caption={read and write .csv files}]
          import csv

          with open('test.csv', 'r') as csv_file:
            csv_reader = csv.reader(csv_file) #csv\_reader是一个迭代器
            next(csv_reader) #跳过第一行
            for line in csv_reader:
              print(line) #这里每一行是一个list

            dic_reader = csv.DictReader(csv_file) #以字典形式读取
            #若行数为n,则得到一个长度为n-1的迭代器,第1行是key,其余n-1行为value

          with open('new_names.csv', 'w') as new_file:
            csv_writer = csv.writer(new_file, delimiter='\t') #这里设定了分隔符
            csv_writer.writerow(list(range(5)))

            dic_writer = csv.DictWriter(new_file, fieldnames=['number', 'name', 'score'])
            #以字典形式写入,fieldnames为colomn名,即第一行
            csv_writer.writeheader() #写入第一行的fieldnames
            csv_writer.writerow({'number': 1, 'name': 'Jack', 'score': 98})
        \end{codeblock}

      \subsubsection{处理.json文件}
        json库可以用于处理.json文件(JavaScript Object Notation)。json文件中的object会被处理称dict,array会被处理生list,
        其他类型的变量也会对应处理。
        \begin{codeblock}[language=python, caption={json module}]
          import json

          data = json.loads(json_string) #将json文件的数据类型转换为python中的数据类型
          new_json_string = json.dumps(data, indent=2, sort_key=true) #将python语句转换称json语句

          with open('test.json') as f:
              data = json.load(f) #读取json文件的内容
          with open('new_test.json', 'w') as wf:
              json.dump(data, f) #写入json文件
          #可选:indent设置缩进为2,sort\_key设置排序所有键
        \end{codeblock}

        
  \section{Container}

    \subsection{iterator}
      iterator和generator(带yield的函数)有些类似,区别在于内存的使用,iterator会生成所有的值并存在内存里,
      generator则按需生成值,节省内存。
      \begin{codeblock}[language=python, caption={iterator}]
        ite = map(lambda x: x * x, range(10), [i for i in range(20)])
        ite = filter(lambda x: x % 2 == false, range(10))
        #这两个函数里第二个及更后面的参数可以是iterator,range,tuple,list,string等可以迭代的内容
        #map会对遍历过程中的每个元素进行function操作,filter只会留下condition为True的元素
        ite = zip('abcd', range(4), [i*i for i in range(4)])
        #zip返回的迭代器是各可循环容器的元素组成的元组,遍历直到有容器被遍历完
        print(next(result)); print(next(result)) #逐项查找
        for value in ite:
          print(value) #遍历迭代器
        #注意,迭代器在遍历后会变为空,这一点与list,range,tuple不同。
        #迭代器转为list等其他变量类型时会将其遍历一遍,因此同样会变空。
      \end{codeblock}

    \subsection{list}
      \begin{codeblock}[language=python, caption={basic operations of list}]
        #创建列表
        empty_list = []; empty_list = list()
        student1=["Hermione","Harry","Ron"]
        student2=["Draco","Padma"]
        student_score=list(range(10)) #[range(10)]会创建一个只有range(10)一个元素的列表
        student_score=[(letter, num) for letter in 'abcd' for num in range(4)]
        score = [(letter, num) for letter, num in zip('abcd', range(4))] #注意这两个是不同的
        student_score = list(map(lambda n: n*n, range(10)))
        #对列表元素进行操作
        student="Potter"
        student1.append(student) #在列表的末尾追加
        student1.insert(2, student) #在指定index处插入
        student2.extend(student1) #列表的拼接,这样可以比+更快
        del student[-1] #删除最后一个元素
        student_popped=student1.pop(1) #删除指定index处的元素,默认删除最后一个
        student_popped=student1.remove('Padma') #寻找第一个指定元素并删除,找不到会报错
        student_reversed=student1.reverse() #反转元素顺序
        seq_Ron=student.index("Ron") #查找第一个匹配的元素并输出,找不到会报错
        #列表拼接、扩展、排序、遍历
        student1=student1+student2
        student3=sorted(student1); student1.sort() #sorted是一个function,sort是一个method
        student3=sorted(student1, key=str.lower(), reverse=True) #不区分大小写倒序排序
        #sorted函数会对每个元素进行key的操作,按照返回值的顺序,对原数据进行排序
        student1.sort(reverse=True) #True stands for down
        if "Hermione" in student1:
          print("Hermione is in student1")
        for a in student1: #遍历列表
        for index, item in enumerate(student1):
        for index, student in enumerate(student1, start=1): #这里index可以从1开始编号
        students = ', '.join(student1) #将list中的元素合并成一个字符串,以逗号为分隔符
        #列表统计
        number=student1.count("Padma") #统计元素出现的数量
        ind=student2.index("Padma") #找到元素首次出现的位置
        min_one, max_one = min(student_score), max(student_score)
        total=sum(student_score) #列表求和
      \end{codeblock}

    \subsection{tuple}
      元组是不可变的序列,但可以重新赋值
      \begin{codeblock}[language=python, caption={basic operation of tuple}]
        #元组创建
        empty_tuple = (); empty_tuple = tuple() #创建空元组
        a=tuple(rage(10,20,3)); a=(); a=(1,2,3)
        a_tup = sorted(a) #tuple没有sort method
      \end{codeblock}

    \subsection{set}
      Python的set是无序集合,这与C++不同。
      \begin{codeblock}[language=python, caption={basic operation of set}]
        students=[
          {"name":"Hermione","house":"Gryffindor"},
          {"name":"Harry","house":"Gryffindor"},
          {"name":"Ron","house":"Gryffindor"},
          {"name":"Draco","house":"Slytherin"},
          {"name":"Padma","house":"Ravenclaw"},
        ]
        houses=set() #创建空集合,注意houses={}会创建一个字典
        houses.pop() #删除第一个元素,注意顺序是随机的
        houses.clear() #清空集合
        for student in students:
          houses.add(student["house"])
        for house in sorted(houses):
          print(house)
        if "Slytherin" in houses:
          print("Slytherin is in houses.")
      \end{codeblock}

    \subsection{dict}
      \begin{codeblock}[language=python, caption={basic operation of dictionary}]
        #创建字典
        students1={
          "Harry":"Gryffindor",
          "Ron":"Gryffindor",
          "Draco":"Slytherin",
          "Padma":"Ravenclaw",
        }
        name=["Harry","Ron","Draco","Padma"]
        house=["Gryffindor","Gryffindor","Slytherin","Ravenclaw"]
        student1=dict(zip(name,sign)) 
        student1=dict(n: s for n, s in zip(name,sign))
        student1=dict((("Harry","Gryffindor"),("Ron","Gryffindor"),
            ("Draco","Slytherin"),("Padma","Ravenclaw"))) #这三句的效果相同
        student1=dict(Harry="Gryffindor",Ron="Gryffindor",Draco="Slytherin",Padma="Ravenclaw")
        student_empty=dict.fromkeys(name) #创建值为空的字典
        #访问,排序,删除,加入,遍历
        ron_house=student1["Ron"] #访问字典,找不到会报错
        ron_house=student1.get("Ron") #访问字典,找不到会返回None
        david_house=student1.get("David", "Not Found") #访问字典,找不到返回Not Found
        s_house = sorted(student1) #根据key来排序
        ron_house=student1.pop("Ron") #删除元素并返回值
        del student1["Ron"] #删除元素
        student1["Hermione"]="Gryffindor" #加入元素
        student1.update({"Hermione":"Gryffindor"}) #合并两字典
        for key,value in student1.items() #遍历字典
        for key in student1.keys(); for key in student1 #遍历键数组
        for value in student1.values() #遍历值数组
      \end{codeblock}

    \subsection{String}
      \subsubsection{编码解码}
        \begin{codeblock}[language=python, caption={create a string}]
          raw_string = r'\tTab' #这里的\textbackslash t不会被替换
          name="Harry"; text='Hello world!'
          byte=name.encode('GBK') #编码,返回编码的16进制值
          name1=byte.decode('GBK') #解码
        \end{codeblock}

      \subsubsection{常用操作}
        \begin{codeblock}[language=python, caption={basic operations of string}]
          length=len(name) #返回字符串长度
          size=len(name.encode()) #返回内存大小
          text.split(' ') #切分字符串
          strnew=string.join(house) #合并字符串
          sub='ab'
          str_replace=sub.replace('a', 'c') #字符串替换
          num=string.count(sub) #检索子字符串出现次数
          start=string.find(sub) #子字符串首次出现的索引,若没有出现,返回-1
          str_lower=str.lower() #转为小写,不会改变原字符串
          str_upper=str.upper() #转为大写
          str_strip=str.strip(['@']) #删去字符串左右两侧的' ',\textbackslash t,\textbackslash r,\textbackslash n等(默认),以及'@'(可选)
          str_lstrip=str.lstrip(['@']) #删去字符串左侧的特定字符
          str_rstrip=str.rstrip(['@']) #删去字符串右侧的特定字符
          str_filled = str.ljust(10, fillchar=' ') #字符串左对齐,右侧填充至10个字符
          str_filled = str.rjust(10, fillchar=' ') #字符串右对齐
          str_filled = str.center(10, fillchar='*') #字符串中间对齐
        \end{codeblock}

      \subsubsection{格式化字符串}
        格式化字符串也就是先制定一个模板,并在模板中以占位符为变量留下位置。一种实现方式是与C语言相似的占位符,
        更适合python的做法是使用format函数或者f字符串。
        \begin{codeblock}[language=python, caption={formatting string1}]
          '%[-][+][0][m][.n]格式字符'%exp #这是模板
          template = '编号:%09d\t 公司名称: %s\t官网: http://www.%s.com'
          print(template%(7, '百度', 'baidu'))
        \end{codeblock}

        可选项如下
        \begin{itemize}
          \item -:左对齐,正数前面无负号,负数前面有负号
          \item +:右对齐,正数前面加正号,负数前面加负号
          \item 0:右对齐,正数前面无负号,负数前面加负号,用0补足位数
          \item m:占有宽度
          \item .n:小数位数
        \end{itemize}

        \begin{table}[H]
          \centering
          \caption{placeholder}
          \label{tab:placeholder}
          \begin{tabular}{cccc}
            \toprule[1.5pt]
            格式字符 & 说明 & 格式字符 & 说明 \\
            \midrule
            \%s & 字符串 & \%c & 单个字符 \\
            \%x & 十六进制整数 & \%o & 八进制整数 \\
            \%f~~\%F & 浮点数 & \%d~~\%i & 十进制整数 \\
            \%\% & 字符\% & \%e & 指数(基底为e) \\
            \bottomrule[1.5pt]
          \end{tabular}
        \end{table}
    
        format函数使用模板
        \begin{codeblock}[language=python, caption={format string}]
          '{[index][:[fill][align][sign][#][width][.precision][type]]}'.format(item1, item2)
        \end{codeblock}

        \begin{itemize}
          \item index:可选,表示该位置填入的是format的那个参数(0-base)
          \item fill:可选,填充字符
          \item align:可选,<左对齐,>右对齐,\^{}内容居中
          \item sign:可选,+-,用法与占位符选项相同
          \item \#:可选,进制前缀显示,也可以选`,'表示添千位分隔符
          \item width:可选,字段宽度
          \item .precision:可选,小数位数
          \item type:可选,指定类型
        \end{itemize}

        \begin{table}[H]
          \centering
          \caption{type options}
          \label{tab:type options}
          \begin{tabular}{cccc}
            \toprule[1.5pt]
            格式字符 & 说明 & 格式字符 & 说明 \\
            \midrule
            S & 字符串类型(默认) & b & 十进制整数转为二进制 \\
            D & 十进制整数 & o & 十进制整数转为八进制 \\
            C & 按ASCII码转为字符 & x~~X & 十进制整数转为十六进制 \\
            e~~E & 转为科学计数法 & f~~F & 转为浮点数,默认6位小数 \\
            g~~G & 自动切换 & \% & 转为百分数 \\
            \bottomrule[1.5pt]
          \end{tabular}
        \end{table}

        下面给出一些例子
        \begin{codeblock}[language=python, caption={examples of format string}]
          number, name, url = 7, "百度", "baidu"
          template='编号:{:0>9s}\t公司名称:{:s}\t官网:http://www.{:s}.com'
          context=template.format(number, name, url)
          context_f=f'编号:{number}\t公司名称:{name}\t官网:http://www.{url}.com'

          f_template='编号:{f_number}\t公司名称:{f_name}\t官网:http://www.{f_url}.com'
          f_context = f_template.format(f_number = number, f_name = name, f_url = url)

          context='编号:{2}\t公司名称:{0}\t官网:http://www.{1}.com'.format(name, url, number)

          #下面假设有company变量,他有三个成员,分别是name、url和number
          context='编号:{0.number}\t公司名称:{0.name}\t官网:http://www.{0.url}.com'.format(company)
        
          company = {'name': 'baidu', 'number': 7, 'url': 'baidu'}
          context = '编号:{number}\t公司名称:{name}\t官网:http://www.{url}.com'.format(**company)
          context = 'This function ran with args: {}, and kwargs: {}'.format(args, kwargs)
          #这里\{\}中的内容可以是表达式或其他类型的数据(如list,dict)
        \end{codeblock}

  \section{regular expression}
    
    \subsection{grammar}
      正则表达式由元字符,限定符,选择字符,排除字符等组成。
      \subsubsection{元字符}
        元字符中的`.'`\$'`\^{}',限定符中的`?'`+'`*'以及`\textbackslash'若需要匹配,则应该使用转义字符。
        \begin{table}[H]
          \centering
          \caption{meta-character}
          \label{tab:meta-character}
          \begin{tabular}{cc}
            \toprule[1.5pt]
            元字符 & 匹配对象 \\
            \midrule
            . & 换行符以外的任意字符\lbrack\^{}\textbackslash n\textbackslash r\rbrack \\
            \textbackslash w & 字母、数字、下划线或汉字 \\
            \textbackslash W & 非字母、数字、下划线、汉字 \\
            \textbackslash s & 任意空白字符(\lbrack\textbackslash f\textbackslash n\textbackslash r\textbackslash t\textbackslash v\rbrack) \\
            \textbackslash S & 任意非空白字符 \\
            \textbackslash d & 数字字符 \\
            \textbackslash D & 非数字字符 \\
            \textbackslash b & 单词的边界(开始或结束) \\
            \textbackslash B & 非单词的边界 \\
            \^{} & 字符串的开始 \\
            \$ & 匹配字符串的结束 \\
            \bottomrule[1.5pt]
          \end{tabular}
        \end{table}

      \subsubsection{限定符}
        \begin{table}[H]
          \centering
          \caption{qualifier}
          \label{tab:qualifier}
          \begin{tabular}{cc}
            \toprule[1.5pt]
            限定符 & 含义 \\
            \midrule
            ? & 匹配0-1次 \\
            + & 匹配至少1次,会贪婪匹配 \\
            * & 匹配0次或多次,会贪婪匹配 \\
            +?~~*? & 匹配规则同上,但是非贪婪 \\
            \{n\} & 匹配n次 \\
            \{n,\} & 匹配至少n次 \\
            \{n,m\} & 匹配至少n次,至多m次 \\
            | & 匹配左右其一 \\
            \bottomrule[1.5pt]
          \end{tabular}
        \end{table}

      \subsubsection{brackets}
        \begin{itemize}
          \item 方括号表示字符集合,如\lbrack aeiou\rbrack 匹配元音字符,\lbrack a-z\rbrack 匹配小写字符。
            \lbrack\textbackslash u4e00-\textbackslash u9fa5\rbrack 匹配汉字。
          \item 方括号内的\^{}表示排除字符,如\lbrack\^{} aeiou\rbrack 匹配非元音字母的所有字符。
          \item 方括号中的无法使用限定符,限定符都判定为原来的字符。
          \item 小括号可以改变限定符的作用范围,如(thir|four)th相当于thirth|fourth,(\textbackslash .\lbrack 0-9\rbrack\{1,3\})\{3\}
            会将括号中的内容重复三次。
          \item 小括号还可以基于匹配模式从字符串中提取子字符串,组成一个元组。
        \end{itemize}

      \subsubsection{notice}
        Python中使用正则表达式需要将部分\textbackslash 转义,由于需要转义的\textbackslash 可能很多,可以使用模式字符串,即在字符串前加上r或R,
        例如:r`\textbackslash bm\textbackslash w*\textbackslash b'。

    \subsection{re module}
      Python中的re模块可以通过正则表达式处理字符串。
      \subsubsection{匹配模式}
        \begin{table}[H]
          \centering
          \caption{matching pattern}
          \label{tab:matching pattern}
          \begin{tabular}{cc}
            \toprule[1.5pt]
            标志 & 含义 \\
            \midrule
            A~~ASCII & 只匹配ASCII范围内的字符 \\
            I~~IGNORECASE & 不区分大小写 \\
            M~~MULTILINE & 将\^{}和\$ 用于每一行的开头和结尾 \\
            S~~DOTALL & .匹配所有字符,包括换行符 \\
            X~~VERBOSE & 忽略未转义的空格和注释 \\
            \bottomrule[1.5pt]
          \end{tabular}
        \end{table}

      \subsubsection{匹配函数}
        这里匹配的字符串区间没有重叠的部分。
        \begin{codeblock}[language=python, caption={string matching}]
          import re #导入模块

          pattern=r'mr_\w+' #模式字符串
          string='MR_SHOP mr_shop' #待匹配字符串

          #match方法可以从字符串开始处开始匹配,若匹配失败,返回None
          match=re.match(pattern,string,re.I) #不区分大小写匹配字符串,返回一个match对象
          print(match.start());print(match.end()) #输出匹配字符串起始位置
          print(match.span()) #输出匹配字符串起止位置元组
          print(match.string) #输出匹配前的字符串,即string
          print(match.group()) #输出匹配的数据

          #search方法可以搜索字符串中第一个可以匹配的子串,返回一个match对象
          match=re.search(pattern,string,re.I) #不区分大小写匹配字符串,返回一个match对象

          #findall和finditer方法可以搜索字符串中所有可以匹配的子串
          match = re.findall(pattern,string,re.I) #不区分大小写匹配字符串,返回一个match对象
          print(match) #输出匹配子串的列表
          match = re.finditer(pattern, string, re.I) #返回一个由match对象组成的迭代器

          #上面的代码可以通过编译模式字符串进行简化
          pattern = re.compile(r'mr_\w+', re.I)
          match = pattern.match(string)
          match = pattern.match(string)
          match = pattern.findall(string)
          match = pattern.finditer(string)

          #捕获组的使用
          pattern = re.compile(r'(\w+)-(\d+)-(\w+)')
          text = "abc-123-def"
          match = pattern.match(text) #这里使用方法与上面相同
          #match.group(0)为'abc-123-def', match.group(i)为第i个捕获组捕获的内容
          match = pattern.findall(text) #match = \lbrack(`abc', `123', `def')\rbrack
        \end{codeblock}

      \subsubsection{替换字符串}
        sub方法可以实现vim中批量搜索并替换字符串的操作。

        模板为:re.sub(pattern,repl,string,count,flags)
        \begin{itemize}
          \item pattern:模式字符串
          \item repl:替换后的子字符串
          \item string:原始字符串
          \item count:可选,最大替换次数,默认为0,表示替换所有的匹配
          \item flags:可选,见\autoref{tab:matching pattern}
        \end{itemize}

        \begin{codeblock}[language=python, caption={string substitution}]
          import re

          pattern = r'1[34578]\d{9}' #模式字符串
          string='电话号码是:13611111111'
          result=re.sub(pattern,'1xxxxxxxxxx',string)

          pattern = re.compile(r'(1[34578]\d)\d{8}')
          string = 'The phone numbers are:13611111111,15888888888,18333333333'
          result = pattern.sub(r'\1********', string) #这里\textbackslash 1表示第一个捕获组
          result = 'The phone numbers are:136********,158********,183********'
        \end{codeblock}

      \subsubsection{切分字符串}
        re.split方法可以根据正则表达式切分字符串,匹配正则表达式的子串将被当作分隔符,并将分割结果以列表的形式返回。

        模板为:re.split(pattern,string,\lbrack maxsplit\rbrack,\lbrack flags\rbrack)
        \begin{itemize}
          \item pattern:模式字符串
          \item string:待切分的字符串
          \item maxsplit:可选,最大拆分次数
          \item flags:可选,见\autoref{tab:matching pattern}
        \end{itemize}
        \begin{codeblock}[language=python, caption={string segmentation}]
          import re

          pattern=r'[?&]'
          url='http://www.mingrisoft.com/login.jsp?username="mr"&pwd="mrsoft"'
          result=re.split(pattern,url)
          print(result)
        \end{codeblock}

  \section{Class}

    \subsection{定义类模板}
      \subsubsection{基本结构}
        \begin{codeblock}[language=python, caption={define a class}]
          class User: 
              '''这是一个学生用户类''' #类的说明
              #以下是静态成员变量,Python中成为类的属性
              student_user = 'Student User' 
              student_number = 0
          
              def __init__(self, name, number): #构造函数
                  self.__name=name #这里名字设置为私有成员,这里变量是非静态成员,python称之为实例属性
                  self.number = number; 
                  User.student_number += 1
                  #这里的name,number是非静态成员变量,Python中成为实例的属性
          
              def __str__(self): #输出运算的重载
                  return f'{self.number} {self.__name}'
          
              @property
              def name(self):
                  return self.__name
          
              @name.setter
              def name(self, name):
                  self.__name = name

              def introduce(self): #self类似*this指针
                  print("I'm ", self.__name, ", my number is ", self.number, sep='')

              @staticmethod
              def greet(input_student): #静态成员函数
                  print(f"Hello, my name's {self.__name}!")

              @classmethod 
              def get_info(cls): #类方法,其第一个变量是类,常用于类型转换,类属性操作等
                  print(f"There are {cls.student_number} {cls.student_user} in total.")
        \end{codeblock}

      \subsubsection{访问限制}
        Python没有对属性和方法的访问权限进行限制。为了保证类内部某些属性不被外部访问,可以在属性或方法名前(或前后)加上双下划线。
        \begin{itemize}
          \item \_\_foo\_\_:首尾双下划线表示定义特殊方法,一般是系统定义名字。
          \item \_\_foo:双下划线表示私有成员,只允许所在的类调用。
          \item \_foo:单下划线表示保护成员,即C++中的protected成员。
        \end{itemize}

      \subsubsection{属性}
        Python中,数据成员被称为属性。可以通过@property将一个方法转换为属性,转换后可以直接通过方法名调用该方法,无需添加()。
        这样做可以简化代码,也为属性添加安全保护,即添加了@property的属性是只读的(这是由于return时经过了复制传递)。

        \begin{codeblock}[language=python, caption={property of class}]
          class User: 
              ''' 用户类 ''' #类的说明
              @property
              def name(self);
                  return self.__name #这个name属性被设定为只读的
        \end{codeblock}

        @name.setter可以在给属性name赋值时运行
        \begin{codeblock}[language=python, caption={setter of class}]
          @name.setter #这里name是属性名称
          def name(self, name)
              if len(name) < 20:
                  self.__name=name
        \end{codeblock}

        @name.deleter属性可以定义一个在删除name属性时执行的函数。
        \begin{codeblock}[language=python, caption={deleter}]
          @name.deleter
          def name(self):
              print('Delete Name!')
              self.__name = None
          del student1.__name
        \end{codeblock}

      \subsubsection{类的特殊成员(魔术方法)}
        \begin{codeblock}[language=python, caption={Special members for class}]
          print(User.__class__) #类的名称
          print(User.__bases__) #类的基类
          print(User.__dict__) #类的数据成员及对应的值,输出一个字典

          def __new__(cls, *args, **kwargs): #创建类的实例
              print(f"Run new with: cls={cls}, args={args}, kwargs={kwargs}")
              return super().__new__(cls) #调用父类的\_\_new\_\_方法,并传入当前类作为参数

          def __del__(self): #对象删除时调用的方法

          def __init__(self, name, number): #接收并初始化实例
              self.__name=name #这里名字设置为私有成员,这里的变量是非静态成员,python中称之为实例属性
              self.number = number; 
              User.student_number += 1
              #这里的name,number是非静态成员变量,Python中成为实例的属性
              #Python中,一个类函数只能有一个\_\_init\_\_函数,若需要多个构造函数,可以用classmethod实现
          #在实例创建时,会先调用\_\_new\_\_在调用\_\_init\_\_

          def __str__(self): #print()和str()的重载
              return f'{self.number} {self.__name}'
          def __repr__(self): #返回一个Python合法的字符串,使之可以用eval重建
              return(f"User({self.__name}, {self.number})")
          student1 = User("Amy", 11); repr_str = repr(student1)
          copy_student1 = eval(repr_str)
          def __format__(self, format_spec): #重载str.format()和f-string

          def __bool__(self): #bool()的重载

          def __add__(self, other): #重载+
          def __iadd__(self, other): #重载+=
          def __sub__(self, other): #重载-
          def __isub__(self, other): #重载-=
          def __mul__(self, other): #重载*
          def __imul__(self, other): #重载*=
          def __truediv__(self, other): #重载/
          def __itruediv__(self, other): #重载/=
          def __floordiv__(self, other): #重载//
          def __mod__(self, other): #重载\%
          def __pow__(self, other): #重载\*\*
          def __eq__(self, other): #重载==
          def __ne__(self, other): #重载!=
          def __lt__(self, other): #重载<
          def __le__(self, other): #重载<=
          def __gt__(self, other): #重载>
          def __ge__(self, other): #重载>=

          def __len__(self): #重载len()
          def __getitem__(self, key): #重载\lbrack\rbrack,读取值
          def __setitem__(self, key, val): #重载\lbrack\rbrack=,修改值
          def __delitem__(self, key): #重载del \lbrack\rbrack,删除值
          def __iter__(self): #返回迭代器,重载iter()
          def __next__(self): #返回迭代器的下一个值,重载next()

          def __call__(self, x): #允许实例像function一样被调用
              return self.__name + x
          #既然class可以像function一样被调用,那class也可以成为\hyperref[subsubsec:decorator]{\underline{decorator}}
          class decorator_class:
              def __init__(self, original_function):
                  self.original_function = original_function

              def __call__(self, *args, **kwargs):
                  print('call method executed before {}'.format(self.original_function.__name__))
                  return self.original_function(*args, **kwargs)
          @decorator_class
          def display():
              print('display function ran')

          def __enter__(self): #进入with语句时自动调用
          def __exit__(self, exc_type, exc_value, traceback): 
          #离开with语句时自动调用(正常和非正常退出都会调用)
          #三个参数分别是异常类型、异常值和异常回溯信息,返回True会一直异常,否则异常会继续传播 
        \end{codeblock}
          
    \subsection{类的使用}
      \subsubsection{创建类的实例}
        \begin{codeblock}[language=python, caption={Create an instance of class}]
          student1 = User('Amy', 11) 
        \end{codeblock}

      \subsubsection{类成员的使用}
        \begin{codeblock}[language=python, caption={Using class members}]
          print(student.student_number); print(User.student_number) #都输出1
          print(student.number); student1.number = 1 #这里number无法修改

          student1.introduce() #输出I'm Amy, my number is 11
          User.introduce(student1) #效果与上一句相同

          User.greet(student1) #输出Hello, my name's Amy!
          student1.greet(student1) #效果与上一句相同

          User.get_info() #输出There are 1 Student User in total.
          student1.get_info() #效果与上一句相同
          print(student1) #这里会调用\_\_str\_\_函数,输出11 Amy
        \end{codeblock}

    \subsection{继承}
      Python的继承默认是公有继承。继承的内容有类的属性,实例方法,类方法,静态方法,特殊方法,
      私有成员不会被继承。类方法在继承时,会以子类作为第一个参数。

      Python中的所有类都继承自builtins.object类。
      \begin{codeblock}[language=python, caption={Class inheritance}]
        class Student(User)
            def __init__(self, grade, name, number):
                super().__init__(name, number) #调用基类的构造函数
                self.__grade = grade
      \end{codeblock}

      子类可以重写父类中的方法,这与C++,Java没什么区别,只是Python没有虚函数机制。

    \subsection{有关类的内置函数}
      由于python中的类在访问和添加成员的操作上限制较少,可以调用一些内置函数来检查这些
      类的合理性。
      \begin{codeblock}[language=python, caption={built-in functions about class}]
        isinstance(object, classinfo) #检查object的类别是不是classinfo
        isinstance(student1, User) #返回True

        issubclass(cls, classinfo) #检查cls类是不是classinfo的子类
        issubclass(Student, User) #返回True

        hasattr(object, name) #检查object是否有name属性
        hasattr(student1, "number") #返回True
        
        callable(object) #检查object是否可以像函数一样被调用
        callable(student1.name) #返回False
        callable(student1.greet) #返回True
      \end{codeblock}

    \subsection{类的典型例子}
      \subsubsection{文件管理类}
        文件管理类可以实现打开创建文件,读写文件,文件异常处理等工作
        \begin{codeblock}[language=python, caption={File management class}]
          class Open_File():

              def __init__(self, filepath, mode):
                  self.filepath = filepath
                  self.mode = mode

              def __enter__(self):
                  print(f"Entering the file: {self.filepath}")
                  self.file = open(self.filepath, self.mode)
                  return self.file

              def __exit__(self, exc_type, exc_val, traceback)
                  print(f"Exiting the file: {self.filepath}")
                  if exc_type:
                      print(f"An exception occurred: {exc_type}, {exc_value}")
                  self.file.close()

          with Open_File('test.txt', 'w') as f:
              f.write('Testing')
        \end{codeblock}

  \section{basic module}

    \subsection{install and import module}
      \subsubsection{install and manage modules}
        可以通过pip安装和管理所需的module。
        \begin{codeblock}[language=bash, caption={download modules}]
          pip install pandas  #安装module
          pip uninstall pandas  #卸载安装的module
          pip list #列出所有已安装的module,可选-o查看是否是最新版本
          pip install -U pandas #更新module
          pip freeze --local | grep -v '^\-e' | cut -d = -f 1 | xargs -n1 pip install -U
          pip search pandas #搜索指定的module
          pip freeze > requirements.txt #将项目所需的module整理成文档
          pip install -r requirements.txt #将文件中的module全部安装下来
        \end{codeblock}

      \subsubsection{virtualenv}
        virtualenv是指项目运行的module环境,这个环境可以不包含global环境中的所有module。
        \begin{codeblock}[language=bash, caption={setup virtualenv}]
          pip install virtualenv 
          mkdir Environments
          cd Environments
          virtualenv project1_env -p /usr/bin/python3.10 #创建virtualenv环境,python版本可选
          source project1_env/bin/activate #activate virtualenv
          pip install numpy #这里下载所需的module即可
          pip freeze --local > requirements.txt #将virtualenv中的module整理成文档
          deactivate #退出virtualenv
          rm -rf project1_env #删除已创建的virtualenv环境
        \end{codeblock}

      \subsubsection{anaconda}
        anaconda也可以用来管理module和环境,其效果相当于pip+virtualenv。
        anaconda的优势在于可以安装不属于python的module,且可提供图形化管理功能。
        \begin{codeblock}[language=bash, caption={anaconda}]
          conda create --name my_app python=2.7 flask sqlalchemy
          #创建一个名为my\_app的项目,项目环境中带有flask和sqlalchemy,python版本为2.7
          conda activate my_app #激活环境,默认环境为base
          conda deactivate #退出环境
          conda env list #列出所有已创建的环境 
          conda remove --name my_app --all #删除已有的环境
          conda env export > environment.yaml #导出environment的内容
          conda env export create -f environment.yaml #通过文件创建环境
        \end{codeblock}

        有时我们需要记录环境与对应项目的目录,这是我们可以在环境目录(可以通过conda env list查看)
        etc/conda/activate.d/env\_vars.sh,etc/conda/deactivate.d/env\_vars.sh,这两个
        文件分别会在activate和deactivate时自动运行。
        \begin{codeblock}[language=bash, caption={activate.d/env\_vars.sh}]
          #!/bin/sh
          export DATABASE_URI="postgresql://user:pass@db_server:5432/test_db"
          #just add anything you need
        \end{codeblock}
        \begin{codeblock}[language=bash, caption={deactivate.d/env\_vars.sh}]
          #!/bin/sh
          unset DATABASE_URI
        \end{codeblock}

        可以通过\~{}/.bashrc中加入以下函数以自动启用文件夹中的environment.yaml。
        \begin{codeblock}[language=bash, caption={conda\_auto\_env}, mathescape=false]
          function conda_auto_env() {
            if [ -e "environment.yaml" ]; then
              ENV_NAME=$(head -n 1 environment.yaml | cut -f2 -d " ") 
              #Check if you are already in the environment
              if [[ $CONDA_PREFIX != *$ENV_NAME* ]]; then
                  #Try to activate environment
                  conda activate $ENV_NAME &>/dev/null
              fi
            fi
          }

          #export PROMPT\_COMMAND="conda\_auto\_env;\$PROMPT\_COMMAND"
          #\$PROMPT\_COMMAND每次按下enter时都会运行,若不需要,可以注释掉最后一行,并使用conda\_auto\_env手动调用环境
        \end{codeblock}

      \subsubsection{import modules}
        \begin{codeblock}[language=python, caption={import module}]
          import math
          import math as ma
          from math import sqrt
          from math import *
        \end{codeblock}

      \subsubsection{view the functions in modules}
        \begin{codeblock}[language=python, caption={view the functions in builtin module}]
          import builtins
          print(dir(builtins)) #这个查看方法对所有module都有效
        \end{codeblock}

    \subsection{builtins}
      最基础的module,包含了print len range abs等常用函数,TypeError ValueError KeyError IndexError
      等常见异常,True False None等内置常量,int str list dict set等内置类型,不需要导入即可使用。

    \subsection{sys}
      有关系统操作的module
      \begin{codeblock}[language=python, caption={sys module}]
        print(sys.path)
        #系统路径列表,可以通过append追加module所在地址
        #也可以追加在~/.bashrc的PYTHONPATH里面(作为环境变量)
        #如果需要import的module是一个文件夹,需要保证文件夹中有一个'\_\_init\_\_.py'文件
        #'\_\_init\_\_.py'文件会在import时运行,该文件可以为空,也可以导入文件夹中的子模块
        print(sys.executable) #输出python的文件位置
        print(sys.version) #输出python版本,用来检验编译器
      \end{codeblock}

    \subsection{os}
      os module可以实现操作系统相关的功能。
      \begin{codeblock}[language=python, caption={os module}]
        #文件管理
        os.getcwd() #输出工作区目录
        os.chdir('/mnt/d/Desktop') #移动到指定目录
        os.listdir() #输出当前目录下的所有文件及文件夹
        os.mkdir('os_test') #在当前目录下新建,注意只能新建一层
        os.makedirs('os_test/my_profile') #新建目录,允许多层
        os.rmdir('os_test'); os.removedirs('os_test/myprofile') #删除目录
        os.rename('test.txt', 'demo.txt') #重命名文件
        print(os.stat('demo.txt')) #查看文件属性,如文件大小(st\_size),最后一次修改时间(st\_mtime)
        for dirpath, dirnames, filenames in os.walk(/mnt/d/Desktop/python) #递归的遍历目录下所有文件
        #dirpath表示当前所在的目录,dirname表示当前目录下的文件夹,filenames表示当前目录下的文件名
        os.environ.get('HOME') #返回home的地址
        os.path.join(os.environ.get('HOME"), 'test.txt') #拼接目录,直接字符串相加容易遗漏或多加slash
        os.path.exists('/mnt/d/test.txt') #返回一个bool值,即目录是否存在
        os.path.basename('/mnt/d/test.txt') #返回`test.txt'这里无论路径是否存在都不会报错
        os.path.dirname('/mnt/d/test.txt') #返回`/mnt/d'
        os.path.split('/mnt/d/test.txt') #返回(`/mnt/d', 'test.txt')
        os.path.splitext('/mnt/d/test.txt') #返回(`/mnt/d/test',`.txt')
      \end{codeblock}

    \subsection{logging}
      logging module是python标准库中用于记录日志的模块,它可以在程序运行时输出各种级别的日志。
      \begin{codeblock}[language=python, caption={logging module}]
        logging.basicConfig(filename='app.log', filemode='w', level=logging.DEBUG,
                    format='%(asctime)s - %(name)s - %(levelname)s - %(message)s',
                    datefmt='%Y-%m-%d %H:%M:%S'
                    )
        #以下信息将会以上面设定的形式存到app.log文件中,只有DEBUG以上级别会显示出来
        logging.debug("这是调试信息")
        logging.info("这是一般信息")
        logging.warning("这是警告信息")
        logging.error("这是错误信息")
        logging.critical("这是严重错误信息")
        #可以使用handlers将日志同时输出到多个文件或输出到控制台
        logging.basicConfig(
            level=logging.INFO,
            format='%(asctime)s - %(levelname)s - %(message)s',
            handlers=[
                logging.FileHandler("logfile.log"),   # 文件处理器
                logging.StreamHandler()               # 控制台处理器
            ]
        )
      \end{codeblock}

    \subsection{date}
      \subsubsection{datetime module}
        datetime用于时间和日期戳记录和转换
        \begin{codeblock}[language=python, caption={datetime \& pytz}]
          d = datetime.date(2020, 7, 24); print(d) #这里的day是一个date类型的变量
          td = datetime.date.today() #输出今天的日期
          print(d.year); print(d.month); print(d.day)
          print(d.weekday()); print(tday.isoweekday()) 
          #in weekday: Monday 0 Sunday 6, in isoweekday: Monday 1 Sunday 7
          tdelta = datetime.timedelta(days = 7) #这里tdelta是timedelta类型的,可以参与加减法
          date2 = date1 + timedelta; timedelta = date1 - date2
          tdelta.total_seconds() #时间差转秒

          t = datetime.time(9, 30, 45, 100000) #四个参数分别是h m s $\mu$s
          print(t.hour); print(t.minute); print(t.second); print(t.microsecond)
          print(t.iosformat()) #以国际标准形式输出时间
          print(t.strftime('%B %d, %Y')) #以自定义形式输出时间,这里可以参考\href{https://docs.python.org/3/library/datetime.html\#strftime-strptime-behavior}{\underline{strftime官方文档}}

          dt = datetime.datetime.strptime('July 26, 2016', '%B %d, %Y') #根据模板由string转datetime
          dt = datetime.datetime(2020, 7, 8, 12, 10, 32, 100000, tz = pytz.UTC) #时区可选
          t = dt.time(); d = dt.date(); print(dt.year)
          sentence = f'Jack has a birthday on {birthday:%B %d, %Y}' #结合fstring使用
        \end{codeblock}

      \subsubsection{pytz module}
        pytz用于管理时区(time zone)相关的信息。
        \begin{codeblock}[language=python, caption={pytz module}]
          dt_today = datetime.datetime.today() #返回当前时区的时间
          dt_now = datetime.datetime.now(tz = pytz.UTC) #返回指定时区的时间,默认没有时区信息
          dt_utcnow = datetime.datetime.utcnow() #返回utc时间,不建议使用
          dt_mtn = dt_now.astimezone(pytz.timezone('Asia/Shanghai')) 
          #需要带有时区参数的datetime变量才可以进行时区转换
          for tz in pytz.all_timezones:
            print(tz)  #输出所有的时区名称
          dt_mtn = pytz.timezone('America/New_York').localize(dt_today) #为时间添加时区
          dt_time = datetime.datetime.fromtimestamp(mod_time) #将时间戳转换成datetime类型
        \end{codeblock}

      \subsubsection{calendar}
        calendar用于计算天数,星期数,星期几等数据。
        \begin{codeblock}[language=python, caption={calendar module}]
          today = datetime.date.today()
          days_in_current_month = calendar.monthrange(today.year, today.month)
          #输出一个tuple,第一个元素表示今天星期几Monday0 Sunday6,第二个表示本月有几天
        \end{codeblock}

    \subsection{calendar}

    \subsection{time}
      time module可以用于计时
      \begin{codeblock}[language=python, caption={time module}]
        t = time.time() #返回linux时间戳,单位都是秒
        t1 = time.perf_counter(); t2 = time.perf_counter() #高精度时间计数器
        t3 = time.process_time(); t4 = time.process_time() #高精度的CPU时间
        print(t2 - t1) #计算时间差,会包含sleep()的时间
        print(t4 - t3) #计算时间差,不会包含sleep()的时间
      \end{codeblock}

  \subsection{collection}
    collection包含了dict,list,str等Container,同时提供了一些额外的数据结构,相当于C++标准库。
    \begin{codeblock}[language=python, caption={namedtuple}]
      import collection

      #namedtuple可以通过给tuple中每个变量加名称增加代码的可读性。
      Color = collection.namedtuple('Color', ['red', 'green', 'blue'])
      color = Color(55, 155, 255); print(color[0], color.red)

      #deque双端队列,可以从两端添加和删除元素
      collection.deque([1, 2, 3])

      #Counter计数器,计算可迭代对象中元素的频率
      collection.Counter(['a', 'b', 'c', 'a', 'b', 'b'])

      #OrderedDict有序字典,记住元素插入顺序
      collection.OrderedDict([('a', 1), ('b', 2)])
      
      #defaultdict可以为字典中没有的元素添加默认值,默认值类型为int
      dd = collection.defaultdict(int); print(dd['key']) #输出0 
      
      #heapq是优先级队列,用最小堆实现
      heap = collection.heapq.heapify([10, 17, 50, 7, 30, 24, 27, 45, 15, 5, 36, 21])
      collection.heappush(heap, 13) #插入元素,heap会以满二叉堆形式存储
      print(heapq.heappop(heap)) #弹出堆顶元素并返回
    \end{codeblock}

    \subsection{random}
      \begin{codeblock}[language=python, caption={random module}]
        value = random.random() #生成[0.0, 1.0)的随机浮点数
        value = random.uniform(1, 10) #生成[1.0, 10.0]的随机浮点数
        value = random.randint(1, 10) #生成[1, 10]的随机整数
        value = random.choice(['Red', 'Greed', 'Blue']) #从list中随机取出一个
        results = random.choices(['Red', 'Greed', 'Blue'], k=10) #随机取k遍,输出一个list
        results = random.choices(['Red', 'Greed', 'Blue'], weight=[18, 18, 2], k=10) 
        shuffled_list = list(range(2, 30)); random.shuffle(shuffled_list) #随机打乱list
        hand = random.sample(list(range(50)), k=5) #取5个样本,5个样本不会重复
      \end{codeblock}

    \subsection{cn2an}
      cn2an module可以实现中文数字到阿拉伯数字的转换。
      \begin{codeblock}[language=python, caption={cn2an}]
        def chinese_to_int(chinese_str):
            arabic_number = cn2an.cn2an(chinese_str, 'normal')
            return arabic_number
      \end{codeblock}

  \section{Maths and Statistics}

    \subsection{math}
      math module提供了最基础的数学函数。
      \begin{codeblock}[language=python, caption={math module}]
        math.ceil(val) #向上取整
        math.floor(val) #向下取整
        math.trunc(val) #向零取整
        math.fabs(val) #取绝对值,只用于浮点数,abs通用于整数、浮点数、复数等

        math.sqrt(val) #开平方根
        math.pow(x, y) #计算`x'的`y'次方
        math.log(x, base); math.log10(x); math.log2(x) #计算对数,默认base为e
        math.e; math.exp(x) #得到e和e的乘方
        math.sin(x); math.asin(x) #三角函数和反三角函数
        math.sinh(x); math.asinh(x) #双曲函数和反双曲函数

        math.factorial(x) #返回x的阶乘
        math.gcd(x, y) #返回`x'`y'的最大公约数
        math.degrees(x); math.radians(x) #弧度和角度转换
      \end{codeblock}

    \subsection{numpy}
      \subsubsection{定义矩阵}
        \begin{codeblock}[language=python, caption={Define matrices using numpy}]
          import numpy as np

          a=np.array([0.1*i for i in range(100)])
          a=np.array([[i+5*j for i in range(5)] for j in range(5)])
          a=np.arange(25).reshape(5,5) #reshape可以重新规定矩阵型号
          a=np.linspace(0,10,100) #第三个参数是生成的列表长度
          a=np.arange(0,10,0.1) #第三个参数是步长
          a=np.logspace(0.9.10) #生成10的0-9次幂
          a=np.eye(3) #生成三维单位阵
          a=np.diag([1,2,3,4,5]) #生成对角阵
          a=np.random.rand(2,3) #生成2*3随机矩阵,0-1均匀分布
          a=np.random.random((2,3)) #生成2*3随机矩阵,用元组表示大小
          a=np.random.randint(low,high,size=(2,3)) #生成2*3随机整数矩阵
        \end{codeblock}

      \subsubsection{特殊函数}  
        \begin{codeblock}[language=python, caption={Advanced operations in numpy}]
          X,Y=np.meshgrid(x,y) #将x,y扩展为一个矩阵
          x,y=np.outer(x,y) #得到矩阵$x^{t}y$
          x=np.append(x1,x2) #拼接array
          a=X[1] #取出矩阵的一行
          a=X[:,1] #取出矩阵的一列
          X[:,0]=a #更改矩阵的一列
          (n,m)=np.where(X>1) #查找满足条件的元素坐标
          a=X[n,m] #取出满足条件的元素
          l=np.argwhere(X>1) #n为坐标组成的二维array
        \end{codeblock}

      \subsubsection{矩阵运算}
        \begin{codeblock}[language=python, caption={calculation of matrix}]
          c=np.dot(x,y) #矩阵乘法
          c=x*y #对应元素相乘
          c=np.dot(a,np.linalg.inv(b)) #矩阵右除
          c=np.dot(a,np.linalg.inv(a),(b)) #矩阵左除
          c=np.transpose(a);c=a.T #矩阵转置
          result=np.linalg.inv(a) #求逆矩阵
          result=np.linalg.det(a) #求行列式
          result=np.linalg.matrix_rank(a) #求矩阵的秩
          matrix.sum(axis=0) #求和,1行0列
        \end{codeblock}

    \subsection{Pandas}
      \subsubsection{Series类型}
        Series类型是一维数组,由index和value组成。
        \begin{codeblock}[language=python, caption={Series in Pandas}]
          import pandas

          data=['A','B','C']
          index=['a','b','c']
          
          series = pandas.Series(data) #默认index从0开始编号
          series_with_index = pandas.Series(data,index=index) #规定index
          print(series.index,series.value) #会输出数组
          print(series_with_index['a']) #调用Series的元素
        \end{codeblock}

      \subsubsection{DataFrame类型}
        DataFrame的每列的名称为键,每个键对应一个数组,这个数组为值。
        \begin{codeblock}[language=python, caption={Create a dataframe}]
          import pandas

          data = {'a':[1,2,3,4,5],'b':[6,7,8,9,10],'c':[11,12,13,14,15]}
          index = ['A','B','C','D','E']
          data_frame = pandas.DataFrame(data) #创建DataFrame对象,默认index从0开始编号
          data_frame = pandas.DataFrame(data,index=index) #规定index
          data_frame = pandas.DataFrame(data,columns=['a','b']) #指定列
          print(data_frame)
        \end{codeblock}

      \subsubsection{读写数据}
        Pandas模块可以将csv或excel文件转为DataFrame变量,也可以将DataFrame变量写入csv或excel文件。
        \begin{codeblock}[language=python, caption={Read and write files using Pandas}]
          import pandas

          data = pandas.read_csv(<filename>)
          data = pandas.read_excel(<filename>)
          data.to_csv(<new_filename>,columns=['A','B'],index=False) #不写入行索引
          data.to_excel(<new_filename>,columns=['A','B'],index=False) #写入excel文件
        \end{codeblock}

      \subsubsection{基本操作}
        \begin{codeblock}[language=python, caption={Basic functions about DataFrame}]
          data_frame['d']=[50,60,70,80,90] #增添数据

          data_frame.drop([0,1],inplace=True) #按index删除,inplace表示对原数据删除不返回删除后的对象
          data_frame.drop(labels='a',axis=1,inplace=True) #按column删除,axis=1表示列,0表示行

          data_frame['a'][1] = numpy.nan; data_frame['b']=[7,8,9,10,11] #修改数据
          data_frame.a[1] = numpy.nan; data_frame.b = [7,8,9,10,11] #这与上面是等价的
        \end{codeblock}

      \subsubsection{统计操作}
        \begin{codeblock}[language=python, caption={Perform statistical operations using DataFrame}]
          #数据预处理
          null_num = data_frame.isnull().sum() #统计空缺值数量,isnull在空缺值返回True,否则返回False
          not_null_num = data_frame.nornull().sum() #统计非空缺值数量
          data_frame.dropna(axis,inplace=True) #删除包含空缺值的整行数据
          data_frame.fillna(0,inplace=True) #修改空缺值
          data_frame.fillna({'A':0,'B':1,'C':2},inplace=True) #每列空缺值用指定的值代替

          #常用的统计函数
          average = data_frame.mean() #求每列的平均值,输出一个Series
          score = data_frame.a+data_frame.b-data_frame.c #可以直接做向量运算
          data_frame.sort_values(['a'],axis=0,ascending=False,inplace=True) #排序,ascending为升序
        \end{codeblock}

  \section{tools}

    \subsection{doctest}
      doctest可以检查函数的输出,在代码注释样例中给出一组输入输出,若结果错误会报错。
      \begin{codeblock}[language=python, caption={doctest hello.py}]
        from operator import floordiv, mod

        def divide_exact(n, d):
          """Return the quotient and remainder of  dividing N by D
          >>> q, r = divide_exact(2013, 10)
          >>> q
          201
          >>> r
          2
          """
          return floordiv(n, d), mod(n, d)
      \end{codeblock}

      \begin{codeblock}[language=bash, caption={doctest bash}]
        python3 -m doctest hello.py
        **********************************************************************
        File "/mnt/d/Desktop/python/program file/test/test_basic/Pythonproject
        1/hello.py", line 8, in hello.divide_exact
        Failed example:
            r
        Expected:
            2
        Got:
            3
        **********************************************************************
        1 items had failures:
          1 of   3 in hello.divide_exact
        ***Test Failed*** 1 failures.
      \end{codeblock}

    \subsection{Try \& Assert}
      try,except用来处理可能出现报错的情况,并提供捕获错误的功能。

      常见的错误类型如下,他们都有一个共同的父类Exception
      \begin{table}
        \centering  
        \caption{Types of Error}
        \label{tab: types of error}
        \begin{tabular}{ccc}
          \toprule[1.5pt]
          错误名 & 错误类型 & 例子 \\
          \midrule
          Syntax Error & 语法错误 & 代码结构不是Python \\
          IndentationError & 缩进错误 & 缩进不一致或缺少缩进 \\
          TypeError & 对象类型错误 & 将字符串与数字相加 \\
          NameError & 命名错误 & 使用了尚未定义的变量或函数 \\
          Attribute & 属性错误 & 使用了一个class没有的属性 \\
          IndexError & 索引错误 & 访问超出list的索引位置 \\
          KeyError & 键错误 & 访问字典中不存在的键 \\
          ValueError & 值错误 & 将非数字字符串传给int() \\
          ImportError & 导入模块错误 & 模块不存在或路径错误 \\
          ArithmeticError & 数学错误 & 下面两个是他的子类 \\
          ZeroDivisionError & 除以零 & 1/0 \\
          OverflowError & 数值错误 & 算术结果超出数值类型范围 \\
          FileNotFoundError & 文件不存在 & 以只读模式打开不存在的文件 \\
          IOError~~OSError & 操作系统错误 & 文件操作出错 \\
          RuntimeError & 标准错误以外的错误 & `raise'语句触发 \\
          AssertionError & 断言错误 & assert语句触发 \\
          StopIteration & 迭代器错误 & 迭代器没有更多项目供迭代 \\
          \bottomrule[1.5pt]
        \end{tabular}
      \end{table}

      \begin{codeblock}[language=python, caption={try except}]
        try:
            f = open('testfile.txt'); val = bad_val
        except FileNotFoundError as e: #捕获文件打开错误,并执行以下部分
            print(e) #输出 No such file or directory: 'testfile.txt'
        except Exception: #捕获剩余的所有错误,exception只会执行碰到的第一个
            print('Sorry. Something went wrong.')
        else: #若try没有出现错误则执行
            print(f.read()); f.close()
        finally: #无论是否出错都会执行
            print('Executing Finally...')
      \end{codeblock}

      raise语句可以在特定情况下手动报出错误
      \begin{codeblock}[language=python, caption={Raise an error}]
        try:
            a = 2
            if a == 2:
                raise Exception
        except Exception as e:
            print(e)
      \end{codeblock}

      Python中的assert语句类似于C++assert断言,不需要导入库,可以与try,except结合使用。
      \begin{codeblock}[language=python, caption={assert in python}]
        def area_square(r):
            assert r > 0, 'A length must be positive'
            return r * r

        try:
            area_square(-1)
        except Exception as e:
            print(e)
      \end{codeblock}

    \subsection{jupyter notebook}
      jupyter notebook可以用于代码笔记,代码汇报等,提供代码实时运行的功能。可以在ubuntu中打开。
      \begin{codeblock}[language=bash, caption={run jupyter notebok}]
        jupyter notebook
      \end{codeblock}

  \section{visualizations}

    \subsection{matplotlib}
      \subsubsection{画布预处理}
        \begin{codeblock}[language=python, caption={basic setup of plt}]
          from matplotlib import pyplot as plt
          import numpy as np

          plt.figure(figsize=(10,20),facecolor,edgecolor)
          plt.title("title")
          plt.xlabel("x");plt.ylabel("y")
          plt.style.use("seaborn-v0_8")
          plt.legend() #显示图例
          plt.xticks(ticks=[2*i+1 for i in range(10)],labels=[2*i+1 for i in range(10)])
          plt.xlim(2,22)
          plt.grid(axis=both, #axis=x or y or False
                  linestyle="dashed", #or dotted or dashdot
                  color="#FFFFFF"
          ) #添加网格线
          plt.axhline(5,color,linestyle,linewidth) #水平参考线
          plt.axvline(10,color,linestyle,linewidth) #垂直参考线
          plt.axhspan(5,7,color,linestyle,linewidth) #水平参考区域
          plt.axvspan(10,12,color,linestyle,linewidth) #垂直参考区域
          plt.annotate(text,xy=(5,10), #待注释点坐标
                      xytext=(7,12), #注释文本位置
                      color="#FFFFFF",fontsize=16,
                      ha="center", #水平居中
                      va="bottom", #垂直对齐
                      arrowprops={"arrowstyle":"->", #or "-"
                                  "color":"#FFFFFF"}
          ) #显示注释点
          plt.text(7,12,text) #显示无箭头注释
        \end{codeblock}

      \subsubsection{基本图表}  
        \begin{codeblock}[language=python, caption={plots of plt}]
          plt.plot(x,y,color="red",
                  linestyle="dashed", #or dotted or dashdot
                  linewidth=3,
                  marker=".", #or "," "o" "+" "x"
                  markersize=8,
                  markerfacecolor="blue",
                  markeredgecolor="cyan"
          )
          plt.bar(x,y,width,bottom=3, #柱形底部高度
                  hatch="/") #or "l" "\" "\textbackslash\textbackslash" "//"
          plt.barh(x,y)
          plt.hist(x,bins) #直方图,bins可以是整数(条数)或列表
          plt.scatter(x,y,s) #s is a list, which stands for the size of the dots
          plt.pie(x,colors=["red","blue","yellow"],
                  autopct=%1.1f%%, #整数部分一位,小数部分一位
                  explode=[0,0.5,0], #将第二块拉出0.5
                  shadow=True,labels=["一月","二月","三月"],
          ) #饼图
          plt.pie(x,colors=["red","blue","yellow"],
                  autopct=%1.1f%%, #整数部分一位,小数部分一位
                  explode=[0,0.5,0], #将第二块拉出0.5
                  shadow=True,labels=["一月","二月","三月"],
                  radius=1.0,wedgeprops={"width":0.6} #内外圆半径
          ) #圆环图
          plt.boxplot(x,showmeans=True, #显示均值
                      flierprops={"marker":"o", #or "," "+" "x"
                                  "markerfacecolor":"red",
                                  "markeredgecolor":"black",
                                  "markersize":8
                      }, #异常点样式
                      patch_artist=True, #自定义箱型
                      boxprops={"facecolor":"red",
                                "edgecolor":"yellow"
                      } #箱型样式
          ) #箱型图
          plt.stackplot(x,y1,y2,y3,color=["red","yellow","blue"]) 
          #面积图(可堆叠)
          plt.errorbar(x,y,yerr=[lower_errors,upper_errors],
                      ecolor=blue, #color of the errorbars
                      elinewidth=3, #width of the errorbars
                      capsize=2 #横杠大小
          )
          plt.imshow(x, #x是个二维列表
                    cmap=plt.cm.cool #设置颜色
          ) #绘制热力图
          plt.colorbar() #显示图例
        \end{codeblock}

      \subsubsection{极坐标图表}
        \begin{codeblock}[language=python, caption={polar plots of plt}]
          plt.polar(theta,r) #雷达图
          plt.thetagrid(angles,labels) #角刻度标签
          plt.rgids(radii,rotation,labels) #r方向刻度标签

          ax=plt.axes(polar=True) #建立极坐标画布
          ax.bar(x=theta,height=data,width=0.4,color="rainbow") #绘制南丁格玫瑰图
          ax.bar(x=theta,height=100,width=0.4,color="white") #绘制中心空白
          ax.text(angle,height,text) #添加注释
          ax,grid(False)
          plt.thetagrids(angles=[],labels=[]) #刻度标签
          plt.rgrids(radii=[20],rotation,labels=['20'])
        \end{codeblock}

      \subsubsection{三维图表}
        \begin{codeblock}[language=python, caption={3D plots of plt}]
          from mpl_tookits.mplot3d import Axes3D
          fig=plt.figure()
          ax1=plt.axes(projection="3d")
          ax1.scatter3D(x,y,z,cmap="blue")
          ax1.plot3D(x,y,z,"gray")
          ax1.plot_surface(X,Y,Z,rstride=0.1,cstride=0.1) #步长越短越清晰
          ax1.contour(X,Y,Z,zdir='x',offset=-3,cmap="cold") #绘制等高线,投影在x=3平面上
          ax1.bar3d(X,Y,height,width,depth,Z,color="red",shade=True) #绘制柱状图,height为柱底高度
        \end{codeblock}

    \subsection{wordcloud}
      \begin{codeblock}[language=python, caption={wordcloud}]
        import matplotlib.pyplot as plt
        import wordcloud as wc

        text_data = """
        Python is a popular programming language.
        It is widely used for web development, data analysis, and artificial intelligence.
        Word clouds are fun visualizations of text data.
        Generate a word cloud using the wordcloud module.
        """

        # 生成词云对象
        wordcloud = wc.WordCloud(width=800, height=400, background_color='white').generate(text_data)

        # 显示词云图
        plt.figure(figsize=(10, 5))
        plt.imshow(wordcloud, interpolation='bilinear')
        plt.axis('off')
        plt.show()
      \end{codeblock}

  \section{web spider}

    \subsection{请求连接}
      Python自带的urllib和urllib3模块可以实现一些网络爬虫的常用操作。外接库中的requests库则更为常用。
      \subsubsection{urllib module}
        利用request模块可以实现get请求方式获取网页内容。
        \begin{codeblock}[language=python, caption={Fetch web content using GET method}]
          import urllib.request #网络请求子模块

          response = urllib.request.urlopen('http://www.baidu.com') #打开网页
          html = response.read() #读取网页代码
          response.close()

          #也可以用with语句简化
          with urllib.request.urlopen('http://www.baidu.com') as response:
              html = response.read()
          print(html)
        \end{codeblock}

        也可以实现post请求方式获取网页内容。
        \begin{codeblock}[language=python, caption={Fetch web content using POST method}]
          import urllib.parse #url解析和引用模块
          import urllib.request

          #使用urlencode方法对数据进行处理,并将处理后的数据设置为utf-8编码
          data = bytes(urllib.parse.urlencode({'word':'hello'}), encoding='utf8')
          with urllib.request.urlopen('http://httpbin.org/post', data=data) as response#打开网页
              html = response.read()
          print(html)
        \end{codeblock}

      \subsubsection{urllib3 module}
        urllib3是一个更强大的Python库。
        \begin{codeblock}[language=python, caption={urllib3 module}]
          import urllib3

          #用GET的方式连接
          http = urllib3.PoolManager() #创建对象,用于处理连接和安全等细节
          response = http.request('GET','https://www.baidu.com/') #连接网站
          print(response.data) #输出读取内容

          #用POST的方式连接
          http = urllib3.PoolManager() #创建对象,用于处理连接和安全等细节
          response = http.request('POST', 'http://httpbin.org/post', fields={'word':'hello'})
          print(response.data) #输出读取内容
        \end{codeblock}

      \subsubsection{requests module}
        另外有个更加人性化的第三方库requests,这个库更加常用。
        \begin{codeblock}[language=python, caption={requests module}]
          import requests

          data={'word':'hello'}
          response = requests.get('http://www.baidu.com',params=data) #get方法访问
          response = requests.post('http://httpbin.org/post',data=data) #post方法访问
          print(response.content) #以字节流形式输出网页源码(没有换行符和缩进,可读性差)
          print(response.text) #以文本形式输出网页源码(带有换行符和缩进,可读性好)
        \end{codeblock}

        为了绕开反爬设计,我们可以请求header。
        \begin{codeblock}[language=python, caption={Handling request headers}]
          import requests

          url = 'http://www.bilibili.com/'
          headers = {'User-Agent':'Mozilla/5.0 (Windows NT 10.0; Win64; x64) AppleWebKit/537.36 
              (KHTML, like Gecko) Chrome/121.0.0.0 Safari/537.36'}
          #这里User-Agent的内容要从网络监视器中复制过来
          response = requests.get(url,headers=headers)
          print(response.content)
        \end{codeblock}

    \subsection{处理html文件}
      BeautifulSoup module用于处理html文件
      \begin{codeblock}[language=python, caption={Handling HTML}]
        from bs4 import BeautifulSoup

        with open('test.html') as html_file: #打开本地的html文件
            soup = BeautifulSoup(html_file, 'lxml') #用lxml解析器解析html文件

        source = requests.get('http://bilibili.com').text
        soup = BeautifulSoup(source, 'lxml') #从网页上获取文件

        print(soup.prettify()) #转换成易读的html文件
        match = soup.title.text #得到文件中的某个部分
        match_div = soup.find('div', class_='footer') #查找文件中第一个对应的内容
        match_all = soup.find_all('div') #查找文件中所有对应的内容,返回一个列表
      \end{codeblock}

\end{document}