% 基本设置和字体设置
\usepackage[english]{babel} %.设置文档语言为英语
\usepackage{fontspec}
\usepackage[UTF8, punct=kaiming, heading,linespread=1.2]{ctex}           %.使用中文环境

% 页面样式设置
\usepackage{fancyhdr}        % 页眉和页脚设置
\usepackage{titlesec}        % 标题页设置
\usepackage{geometry}        % 页面布局控制

% 表格相关宏包
\usepackage{booktabs}        %.优雅的表格横线
\usepackage{longtable}       %.长表格支持
\usepackage{threeparttable}  %.三部分表格
\usepackage{threeparttablex} %.三部分表格扩展
\usepackage{colortbl}        %.表格颜色
\usepackage{xcolor}          %.表格背景色
\usepackage{multirow}        %.表格单元格多行合并
\usepackage{multicol}        %.表格单元格多列合并
\usepackage{makecell}        %.制作更复杂的表头和表格单元格

% 浮动体工具
\usepackage{float}           %.控制浮动对象位置
\usepackage{graphicx}        %.插入图形
\usepackage[font=small,labelfont=bf]{caption}

% 数学和公式
\usepackage{amsmath}         %.数学公式支持
\usepackage{enumerate}       %.列表环境
\usepackage{listings}        %.代码高亮显示
\usepackage{chngcntr}        %.修改计数器
\usepackage{siunitx}         %.科学计数法和单位

% 引用和参考文献
\usepackage[hidelinks]{hyperref} % 超链接,隐藏颜色边框
\usepackage{csquotes}           % 语言相关引号样式处理,建议与babel搭配
\usepackage[backend=biber, style=authoryear]{biblatex}
\addbibresource{ref.bib}
\graphicspath{{./figure/}}

\geometry{
  left=2.5cm,    % 设置左边距
  right=2.5cm,   % 设置右边距
  top=3cm,       % 设置上边距
  bottom=2.5cm,  % 设置下边距
  headheight=15pt, % 设置页眉高度
  headsep=15pt,     % 设置页眉与正文之间的间距
  footskip=15pt    % 设置页脚与正文之间的间距
}

\setmainfont{Times New Roman}
\setsansfont{Arial} %.斜体字体 
\setmonofont{Fira Code} %.等宽字体

\lstdefinestyle{lstStyleCode}{
  tabsize=4,
  frame=shadowbox, %把代码用带有阴影的框圈起来
  framerule=0.15pt,
  numbers=left,
  numberstyle=\tiny\color{black},
  rulesepcolor=\color{red!20!green!20!blue!20},
  aboveskip         = \medskipamount,
  belowskip         = \medskipamount,
  basicstyle        = \zihao{6}\ttfamily,
  commentstyle      = \slshape\color{black!60},
  stringstyle       = \color{green!40!black!100},
  keywordstyle      = \bfseries\color{blue!50!black},
  extendedchars     = false,
  upquote           = true,
  showstringspaces  = false,
  xleftmargin       = 1em,
  xrightmargin      = 1em,
  breaklines        = false,
  framexleftmargin  = 1em,
  framexrightmargin = 1em,
  backgroundcolor   = \color{gray!6},
  columns           = flexible,
  keepspaces        = true,
  texcl             = true,
  mathescape        = true
}
\lstnewenvironment{codeblock}[1][]{%
  \lstset{style=lstStyleCode,#1}}{}

\lstdefinelanguage{HTMLwithJinja}{
    language=HTML,
    morekeywords={if, else, endif, for, endfor, block, endblock, extends, include},
    alsoletter={<>},
    morekeywords={\{, \}},
    sensitive=true,
}

\definecolor{lightgray}{rgb}{0.9,0.9,0.9}
\lstdefinestyle{customjs}{
    language=HTML, % 或者其他适合的语言
    backgroundcolor=\color{lightgray}, % 设置背景色
    commentstyle=\color{green}, % 注释颜色
    stringstyle=\color{red}, % 字符串颜色
    keywordstyle=\color{blue}, % 关键字颜色
    morecomment=[l]{\{\%}, % 处理 {% 的注释
    morecomment=[l]{\%\}}, % 处理 %} 的注释
    basicstyle=\ttfamily, % 基本字体
    showstringspaces=false, % 不显示字符串中的空格
}

%. 设置计数器格式,以章节进行计数。
\renewcommand{\thesection}{\arabic{section}}
\renewcommand{\thesubsection}{\thesection.\arabic{subsection}}
\renewcommand{\thesubsubsection}{\thesubsection.\arabic{subsubsection}}
\renewcommand{\thetable}{\arabic{table}}
\renewcommand{\thefigure}{\arabic{figure}}
\counterwithin{table}{section}
\counterwithin{figure}{section}
%. 更改表格名称为Table,图片更名为Figure
\renewcommand{\tablename}{Table}
\renewcommand{\figurename}{Figure}
