\usepackage{xeCJKfntef}
\usepackage[english]{babel}
\usepackage{fancyhdr}
\usepackage{booktabs}
\usepackage{longtable}
\usepackage{threeparttable}
\usepackage{threeparttablex}
\usepackage{colortbl}
\usepackage[table]{xcolor}
\usepackage{multirow}
\usepackage{multicol}
\usepackage{makecell}
\usepackage{ctex}
\usepackage{amsmath}
\usepackage{float}
\usepackage{enumerate}
\usepackage{listings}
\usepackage{chngcntr}
\usepackage{geometry}
\usepackage{siunitx} %.物理单位和符号
\usepackage[hidelinks=true]{hyperref}

\geometry{
  left=3.17cm, % 设置页边距
  right=3.17cm,
  top=2.54cm,
  bottom=2.54cm,
  headheight=15pt, % 对页眉页脚有特殊要求时设定
  headsep=20pt, % 页眉和正文之间的间距
  footskip=20pt % 页脚和正文之间的间距
}

\linespread{1.0}

\setmainfont{Times New Roman} %.正文字体
\setsansfont{Arial} %.斜体字体 
\setmonofont{Consolas} %.等宽字体
\setCJKmainfont{KaiTi}
\setCJKsansfont{SimHei}
\setCJKmonofont{SimHei}

\lstdefinestyle{lstStyleCode}{
  tabsize=4,
  frame=shadowbox, %把代码用带有阴影的框圈起来
  framerule=0.15pt,
  numbers=left,
  numberstyle=\tiny\color{black},
  rulesepcolor=\color{red!20!green!20!blue!20},
  aboveskip         = \medskipamount,
  belowskip         = \medskipamount,
  basicstyle        = \ttfamily\zihao{6},
  commentstyle      = \slshape\color{black!60},
  stringstyle       = \color{green!40!black!100},
  keywordstyle      = \bfseries\color{blue!50!black},
  extendedchars     = false,
  upquote           = true,
  showstringspaces  = false,
  xleftmargin       = 1em,
  xrightmargin      = 1em,
  breaklines        = false,
  framexleftmargin  = 1em,
  framexrightmargin = 1em,
  backgroundcolor   = \color{gray!6},
  columns           = flexible,
  keepspaces        = true,
  texcl             = true,
  mathescape        = true
}
\lstnewenvironment{codeblock}[1][]{%
  \lstset{style=lstStyleCode,#1}}{}

%. 设置计数器格式,以章节进行计数。
\renewcommand{\thesection}{\arabic{section}}
\renewcommand{\thesubsection}{\thesection.\arabic{subsection}}
\renewcommand{\thesubsubsection}{\thesubsection.\arabic{subsubsection}}
\renewcommand{\thetable}{\arabic{table}}
\renewcommand{\thefigure}{\arabic{figure}}
\counterwithin{table}{section}
\counterwithin{figure}{section}
%. 更改表格名称为Table,图片更名为Figure
\renewcommand{\tablename}{Table}
\renewcommand{\figurename}{Figure}
\addto\captionsenglish{\renewcommand{\contentsname}{目录}} %. 设置目录大标题名称
